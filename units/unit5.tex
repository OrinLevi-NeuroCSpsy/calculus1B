% יחידה 5: אינטגרלים לא אמיתיים ויישומים
\section{אינטגרלים לא אמיתיים ויישומים}

יחידה זו עוסקת באינטגרלים לא אמיתיים (אינטגרלים עם גבולות אינסופיים או עם נקודות סינגולריות), מבחני התכנסות, ויישומים גאומטריים.

% ===================================
\subsection{הגדרת אינטגרל לא אמיתי מסוג ראשון}
% ===================================

\begin{defbox}
\textbf{הגדרה 11.1 (אינטגרל לא אמיתי — גבול אינסופי).}
תהי $f : [a, +\infty) \to \R$ פונקציה \textbf{אינטגרבילית מקומית} ב־$[a, +\infty)$.

נגדיר $F : [a, +\infty) \to \R$ על ידי:
\[
F(x) = \int_a^x f(t)\dt \quad \text{לכל } x \in [a, +\infty)
\]

נאמר כי \textbf{האינטגרל הלא אמיתי} של $f$ בקטע $[a, +\infty)$ \textbf{מתכנס} כאשר קיים וסופי הגבול $\limx{+\infty} F(x)$.

במקרה זה נסמן:
\[
\boxed{\int_a^{+\infty} f(x)\dx = \limx{+\infty} \int_a^x f(t)\dt}
\]

אם הגבול אינו קיים או אינו סופי — נאמר שהאינטגרל \textbf{מתבדר} (או לא מתכנס).
\end{defbox}

\begin{notebox}
\textbf{הערה 11.2.}
באופן דומה מגדירים אינטגרל לא אמיתי על קטעים מהצורה $(-\infty, a]$:
\[
\int_{-\infty}^a f(x)\dx = \limx{-\infty} \int_x^a f(t)\dt
\]
\end{notebox}

% ===================================
\subsection{דוגמאות בסיסיות}
% ===================================

\begin{exbox}
\textbf{דוגמה 1.} $\int_0^{+\infty} \cos x\dx$ — \textbf{לא מתכנס}.

\textbf{פתרון:}
\[
\int_0^x \cos t\dt = \sin x
\]
הגבול $\limx{+\infty} \sin x$ אינו קיים (מתנדנד בין $-1$ ל־$1$).
\end{exbox}

\begin{exbox}
\textbf{דוגמה 2.} $\int_0^{+\infty} e^{\alpha x}\dx$ עבור $\alpha \in \R \setminus \{0\}$.

\textbf{פתרון:}
\[
\int_0^x e^{\alpha t}\dt = \frac{1}{\alpha}(e^{\alpha x} - 1)
\]

הגבול ב־$x \to +\infty$ קיים וסופי \textbf{אם ורק אם $\alpha < 0$}.

במקרה זה:
\[
\int_0^{+\infty} e^{\alpha x}\dx = \frac{1}{\alpha}(0 - 1) = -\frac{1}{\alpha} = \frac{1}{|\alpha|}
\]
\end{exbox}

\begin{exbox}
\textbf{דוגמה 3.} $\int_0^{+\infty} \frac{1}{1+x^2}\dx = \frac{\pi}{2}$.

\textbf{פתרון:}
\[
\int_0^x \frac{1}{1+t^2}\dt = \arctan x
\]
\[
\limx{+\infty} \arctan x = \frac{\pi}{2}
\]
\end{exbox}

\begin{thmbox}
\textbf{דוגמה חשובה — אינטגרל $\int_1^{+\infty} \frac{1}{x^\alpha}\dx$.}

\begin{itemize}
    \item עבור $\alpha = 1$: $\int_1^x \frac{1}{t}\dt = \ln x \to +\infty$ — \textbf{מתבדר}.
    \item עבור $\alpha \neq 1$: $\int_1^x \frac{1}{t^\alpha}\dt = \frac{x^{1-\alpha} - 1}{1-\alpha}$. הגבול קיים וסופי \textbf{אם ורק אם $\alpha > 1$}.
\end{itemize}

\textbf{סיכום:}
\[
\boxed{\int_1^{+\infty} \frac{1}{x^\alpha}\dx \text{ מתכנס } \Leftrightarrow \alpha > 1}
\]
\end{thmbox}

% ===================================
\subsection{אינטגרל לא אמיתי מסוג שני — סינגולריות}
% ===================================

\begin{defbox}
\textbf{הגדרה 11.3 (אינטגרל לא אמיתי — סינגולריות בגבול).}
תהי $f : [a, b) \to \R$ פונקציה אינטגרבילית מקומית ב־$[a, b)$.

נאמר כי \textbf{האינטגרל הלא אמיתי} $\int_a^b f(x)\dx$ \textbf{מתכנס} כאשר קיים וסופי הגבול:
\[
\int_a^b f(x)\dx = \limx{b^-} \int_a^x f(t)\dt
\]
\end{defbox}

\begin{exbox}
\textbf{דוגמה 1.} $\int_0^1 \frac{1}{\sqrt{x}}\dx = 2$ — \textbf{מתכנס}.

\textbf{פתרון:} יש סינגולריות ב־$x = 0$.
\[
\int_0^1 \frac{1}{\sqrt{x}}\dx = \lim_{\eps \to 0^+} \int_\eps^1 x^{-1/2}\dx = \lim_{\eps \to 0^+} [2\sqrt{x}]_\eps^1 = 2 - \lim_{\eps \to 0^+} 2\sqrt{\eps} = 2
\]
\end{exbox}

\begin{exbox}
\textbf{דוגמה 2.} $\int_0^1 \frac{1}{x}\dx$ — \textbf{מתבדר}.

\textbf{פתרון:}
\[
\int_0^1 \frac{1}{x}\dx = \lim_{\eps \to 0^+} \int_\eps^1 \frac{1}{x}\dx = \lim_{\eps \to 0^+} [-\ln\eps] = +\infty
\]
\end{exbox}

\begin{thmbox}
\textbf{דוגמה חשובה — אינטגרל $\int_0^1 \frac{1}{x^\alpha}\dx$.}

\textbf{סיכום:}
\[
\boxed{\int_0^1 \frac{1}{x^\alpha}\dx \text{ מתכנס } \Leftrightarrow \alpha < 1}
\]
\end{thmbox}

% ===================================
\subsection{קריטריון קושי להתכנסות}
% ===================================

\begin{thmbox}
\textbf{קריטריון קושי.}
האינטגרל $\int_a^{+\infty} f(x)\dx$ \textbf{מתכנס} אם ורק אם:

לכל $\eps > 0$ קיים $M \ge a$ כך שלכל $q > p > M$:
\[
\abs{\int_p^q f(x)\dx} < \eps
\]
\end{thmbox}

% ===================================
\subsection{מבחני השוואה}
% ===================================

\begin{thmbox}
\textbf{מבחן ההשוואה.}
תהיינה $f, g : [a, +\infty) \to \R$ אינטגרביליות מקומית עם $0 \le f(x) \le g(x)$ לכל $x \ge a$.
\begin{enumerate}
    \item אם $\int_a^{+\infty} g(x)\dx$ מתכנס — אז $\int_a^{+\infty} f(x)\dx$ מתכנס.
    \item אם $\int_a^{+\infty} f(x)\dx$ מתבדר — אז $\int_a^{+\infty} g(x)\dx$ מתבדר.
\end{enumerate}
\end{thmbox}

\begin{thmbox}
\textbf{מבחן ההשוואה הגבולי.}
תהיינה $f, g : [a, +\infty) \to \R$ אי־שליליות ואינטגרביליות מקומית.

אם $\limx{+\infty} \frac{f(x)}{g(x)} = L \in (0, +\infty)$ — אז:
\[
\int_a^{+\infty} f(x)\dx \text{ מתכנס } \Leftrightarrow \int_a^{+\infty} g(x)\dx \text{ מתכנס}
\]
\end{thmbox}

\begin{exbox}
\textbf{דוגמה.} קבעו האם $\int_1^{+\infty} \frac{1}{x^2 + x}\dx$ מתכנס.

\textbf{פתרון:} נשווה ל־$g(x) = \frac{1}{x^2}$:
\[
\limx{+\infty} \frac{f(x)}{g(x)} = \limx{+\infty} \frac{x^2}{x^2 + x} = \limx{+\infty} \frac{1}{1 + 1/x} = 1 \in (0, +\infty)
\]

כיוון ש־$\int_1^{+\infty} \frac{1}{x^2}\dx$ מתכנס ($\alpha = 2 > 1$), גם $\int_1^{+\infty} \frac{1}{x^2 + x}\dx$ \textbf{מתכנס}.
\end{exbox}

% ===================================
\subsection{התכנסות בהחלט והתכנסות בתנאי}
% ===================================

\begin{defbox}
\textbf{הגדרה 11.11 (התכנסות בהחלט).}
תהי $f : [a, +\infty) \to \R$ אינטגרבילית מקומית.

נאמר כי האינטגרל $\int_a^{+\infty} f(x)\dx$ \textbf{מתכנס בהחלט} כאשר האינטגרל $\int_a^{+\infty} |f(x)|\dx$ מתכנס.
\end{defbox}

\begin{thmbox}
\textbf{טענה 11.12.}
אם האינטגרל \textbf{מתכנס בהחלט} אז הוא \textbf{מתכנס}.
\end{thmbox}

\begin{proofbox}
\textbf{הוכחה.}
מקריטריון קושי: לכל $\eps > 0$ קיים $M$ כך שלכל $q > p > M$:
\[
\abs{\int_p^q f(x)\dx} \le \int_p^q |f(x)|\dx < \eps
\]
לכן $\int_a^{+\infty} f(x)\dx$ מתכנס. \hfill $\blacksquare$
\end{proofbox}

\begin{defbox}
\textbf{הגדרה (התכנסות בתנאי).}
אינטגרל שמתכנס אך \textbf{לא} מתכנס בהחלט נקרא \textbf{מתכנס בתנאי}.
\end{defbox}

\begin{exbox}
\textbf{דוגמה.}
$\int_1^{+\infty} \frac{\sin x}{x}\dx$ — \textbf{מתכנס בתנאי}.

(מוכיחים עם קריטריון דיריכלה; האינטגרל $\int_1^{+\infty} \frac{|\sin x|}{x}\dx$ מתבדר.)
\end{exbox}

% ===================================
\subsection{קריטריון אבל וקריטריון דיריכלה}
% ===================================

\begin{thmbox}
\textbf{טענה 11.20 (קריטריון אבל).}
תהיינה $f, g : [a, +\infty) \to \R$. נניח כי:
\begin{enumerate}
    \item $f$ רציפה ו־$\int_a^{+\infty} f(x)\dx$ מתכנס.
    \item $g$ מונוטונית וחסומה.
\end{enumerate}
אז $\int_a^{+\infty} f(x)g(x)\dx$ \textbf{מתכנס}.
\end{thmbox}

\begin{thmbox}
\textbf{טענה 11.21 (קריטריון דיריכלה).}
תהיינה $f, g : [a, +\infty) \to \R$. נניח כי:
\begin{enumerate}
    \item $f$ רציפה ו־$F(x) = \int_a^x f(t)\dt$ \textbf{חסומה}.
    \item $g$ מונוטונית ו־$\limx{+\infty} g(x) = 0$.
\end{enumerate}
אז $\int_a^{+\infty} f(x)g(x)\dx$ \textbf{מתכנס}.
\end{thmbox}

\begin{exbox}
\textbf{דוגמה — שימוש בקריטריון דיריכלה.}

הוכיחו כי $\int_1^{+\infty} \frac{\sin x}{x}\dx$ מתכנס.

\textbf{פתרון:}
נגדיר $f(x) = \sin x$, $g(x) = \frac{1}{x}$.

\begin{enumerate}
    \item $F(x) = \int_1^x \sin t\dt = -\cos x + \cos 1$ חסומה (ערכים בין $\cos 1 - 1$ ל־$\cos 1 + 1$).
    \item $g(x) = \frac{1}{x}$ יורדת ו־$\limx{+\infty} g(x) = 0$.
\end{enumerate}

לפי קריטריון דיריכלה, האינטגרל מתכנס.
\end{exbox}

% ===================================
\subsection{קשר בין טורים לאינטגרלים}
% ===================================

\begin{thmbox}
\textbf{טענה 11.19 (מבחן האינטגרל).}
תהי $f : [a, +\infty) \to \R$ אינטגרבילית מקומית, \textbf{אי־שלילית ויורדת}.

אז לכל $N \in \N_+$:
\[
\sum_{n=1}^{N} f(a+n) \le \int_a^{a+N} f(x)\dx \le \sum_{n=0}^{N-1} f(a+n)
\]

\textbf{יתר על כן:} האינטגרל $\int_a^{+\infty} f(x)\dx$ מתכנס \textbf{אם ורק אם} הטור $\sum_{n=0}^{\infty} f(a+n)$ מתכנס.
\end{thmbox}

\begin{exbox}
\textbf{יישום — הטור ההרמוני.}

הטור $\sum_{n=1}^{\infty} \frac{1}{n}$ והאינטגרל $\int_1^{+\infty} \frac{1}{x}\dx$ קשורים.

מהמשפט:
\[
\ln(N+1) \le \sum_{n=1}^N \frac{1}{n} \le 1 + \ln N
\]

לכן $\sum_{n=1}^N \frac{1}{n} \sim \ln N$ ובפרט הטור \textbf{מתבדר}.
\end{exbox}

% ===================================
\subsection{יישומים גאומטריים}
% ===================================

\begin{thmbox}
\textbf{שטח בין עקומות.}
עבור $f \ge g$ ב־$[a,b]$, השטח בין $y = f(x)$ ל־$y = g(x)$ הוא:
\[
\boxed{S = \int_a^b \big(f(x) - g(x)\big)\dx}
\]
\end{thmbox}

\begin{exbox}
\textbf{דוגמה.} שטח בין $y = x^2$ ו־$y = x$ ב־$[0, 1]$.

\textbf{פתרון:} בקטע זה $x \ge x^2$, לכן:
\[
S = \int_0^1 (x - x^2)\dx = \brackets{\frac{x^2}{2} - \frac{x^3}{3}}_0^1 = \frac{1}{2} - \frac{1}{3} = \frac{1}{6}
\]
\end{exbox}

\begin{thmbox}
\textbf{נפח גוף סיבוב (שיטת הדיסקים).}
סיבוב השטח מתחת ל־$y = f(x)$ סביב ציר $x$:
\[
\boxed{V = \pi \int_a^b [f(x)]^2\dx}
\]
\end{thmbox}

\begin{exbox}
\textbf{דוגמה.} נפח כדור ברדיוס $R$.

מסובבים את $y = \sqrt{R^2 - x^2}$ (חצי מעגל) סביב ציר $x$:
\[
V = \pi \int_{-R}^R (R^2 - x^2)\dx = \pi\brackets{R^2 x - \frac{x^3}{3}}_{-R}^R
\]
\[
= \pi\parens{R^3 - \frac{R^3}{3}} - \pi\parens{-R^3 + \frac{R^3}{3}} = \pi \cdot \frac{2R^3}{3} + \pi \cdot \frac{2R^3}{3} = \frac{4\pi R^3}{3}
\]
\end{exbox}

\begin{thmbox}
\textbf{אורך קשת.}
לעקומה $y = f(x)$ עם $f'$ רציפה:
\[
\boxed{L = \int_a^b \sqrt{1 + [f'(x)]^2}\dx}
\]
\end{thmbox}

\begin{exbox}
\textbf{דוגמה.} אורך הקשת $y = \frac{x^{3/2}}{3}$ ב־$[0, 4]$.

\textbf{פתרון:}
\[
f'(x) = \frac{1}{2}\sqrt{x}, \quad 1 + [f'(x)]^2 = 1 + \frac{x}{4} = \frac{4+x}{4}
\]
\[
L = \int_0^4 \sqrt{\frac{4+x}{4}}\dx = \int_0^4 \frac{\sqrt{4+x}}{2}\dx
\]

נציב $u = 4 + x$:
\[
= \frac{1}{2} \cdot \frac{2}{3}[(4+x)^{3/2}]_0^4 = \frac{1}{3}(8^{3/2} - 4^{3/2}) = \frac{1}{3}(16\sqrt{2} - 8) = \frac{8(2\sqrt{2} - 1)}{3}
\]
\end{exbox}

\begin{thmbox}
\textbf{שטח פנים של גוף סיבוב.}
סיבוב הקשת $y = f(x)$ סביב ציר $x$:
\[
\boxed{A = 2\pi \int_a^b f(x) \sqrt{1 + [f'(x)]^2}\dx}
\]
\end{thmbox}

% ===================================
\subsection{תרגילים}
% ===================================

\begin{exercisebox}
\textbf{תרגיל 1.}
קבעו התכנסות:
\begin{enumerate}
    \item $\int_1^{+\infty} \frac{\sin x}{x^2}\dx$
    \item $\int_1^{+\infty} \frac{1}{x \ln x}\dx$
    \item $\int_0^1 \frac{1}{\sqrt{x(1-x)}}\dx$
    \item $\int_1^{+\infty} \frac{x}{e^x}\dx$
\end{enumerate}
\end{exercisebox}

\begin{exercisebox}
\textbf{תרגיל 2.}
חשבו את נפח הגוף הנוצר מסיבוב $y = e^{-x}$ סביב ציר $x$ ב־$[0, +\infty)$.
\end{exercisebox}

\begin{exercisebox}
\textbf{תרגיל 3.}
תהי $f : [a, +\infty) \to \R$ אי־שלילית ואינטגרבילית מקומית. הוכיחו כי אם $\int_a^{+\infty} f(x)\dx$ מתכנס וגם קיים $\limx{+\infty} f(x)$, אז הגבול שווה $0$.
\end{exercisebox}

\begin{exercisebox}
\textbf{תרגיל 4.}
תהי $f : [1, +\infty) \to \R$ אי־שלילית, יורדת ואינטגרבילית מקומית. הגדירו:
\[
a_n = \sum_{k=1}^n f(k) - \int_1^n f(x)\dx
\]
הוכיחו כי $\{a_n\}$ מתכנסת.

\textbf{הערה:} זו הדרך להגדיר את \textbf{קבוע אוילר־מסקרוני} $\gamma = \lim_{n \to \infty}\parens{\sum_{k=1}^n \frac{1}{k} - \ln n} \approx 0.5772$.
\end{exercisebox}

\begin{exercisebox}
\textbf{תרגיל 5.}
הוכיחו כי:
\[
\int_0^{+\infty} e^{-x^2}\dx = \frac{\sqrt{\pi}}{2}
\]
\textbf{רמז:} זהו האינטגרל הגאוסיאני. ההוכחה המלאה משתמשת באינטגרל כפול.
\end{exercisebox}
