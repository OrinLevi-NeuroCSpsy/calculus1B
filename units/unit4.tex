% יחידה 4: שיטות אינטגרציה לאינטגרל מסוים
\section{שיטות אינטגרציה לאינטגרל מסוים}

יחידה זו עוסקת בהתאמת שיטות האינטגרציה (אינטגרציה בחלקים ושינוי משתנה) לאינטגרל המסוים, ובמשפט ערך הביניים לאינטגרלים.

% ===================================
\section{אינטגרציה בחלקים לאינטגרל מסוים}
% ===================================

\begin{thmbox}
\textbf{טענה 10.8 (אינטגרציה בחלקים).}
תהיינה $F, g : [a,b] \to \R$ פונקציות גזירות ב־$[a,b]$, ונגזרותיהן $F', g'$ אינטגרביליות רימן ב־$[a,b]$. אז:
\[
\boxed{\int_a^b F'(x) g(x)\dx = F(b)g(b) - F(a)g(a) - \int_a^b F(x) g'(x)\dx}
\]
\end{thmbox}

\begin{proofbox}
\textbf{הוכחה.}
נגדיר $H(x) = F(x) \cdot g(x)$. לפי כלל המכפלה:
\[
H'(x) = F'(x)g(x) + F(x)g'(x)
\]

לכן $F'(x)g(x) = H'(x) - F(x)g'(x)$.

נשלב ונשתמש בנוסחת ניוטון־לייבניץ:
\[
\int_a^b F'(x)g(x)\dx = \int_a^b H'(x)\dx - \int_a^b F(x)g'(x)\dx
\]
\[
= H(b) - H(a) - \int_a^b F(x)g'(x)\dx = F(b)g(b) - F(a)g(a) - \int_a^b F(x)g'(x)\dx
\]
\hfill $\blacksquare$
\end{proofbox}

\begin{notebox}
\textbf{סימון מקוצר.}
\[
\int_a^b u\,dv = [uv]_a^b - \int_a^b v\,du
\]
\end{notebox}

\begin{exbox}
\textbf{דוגמה 1.} חישוב $\int_0^1 x e^x\dx$.

נגדיר $F(x) = e^x$ (כך ש־$F'(x) = e^x$), $g(x) = x$ (כך ש־$g'(x) = 1$).

\textbf{פתרון:}
\[
\int_0^1 x e^x\dx = [x e^x]_0^1 - \int_0^1 e^x \cdot 1\dx = e - [e^x]_0^1 = e - (e - 1) = 1
\]
\end{exbox}

\begin{exbox}
\textbf{דוגמה 2.} חישוב $\int_0^{\pi} x \sin x\dx$.

נגדיר $F(x) = -\cos x$ (כך ש־$F'(x) = \sin x$), $g(x) = x$.

\textbf{פתרון:}
\[
\int_0^{\pi} x \sin x\dx = [-x\cos x]_0^{\pi} - \int_0^{\pi} (-\cos x)\dx = \pi - (-1) - 0 + [\sin x]_0^{\pi} = \pi + 0 = \pi
\]
\end{exbox}

\begin{exbox}
\textbf{דוגמה 3.} חישוב $\int_1^e (\ln x)^2\dx$.

נבצע אינטגרציה בחלקים עם $u = (\ln x)^2$, $dv = dx$.

אז $du = \frac{2\ln x}{x}dx$, $v = x$.

\[
\int_1^e (\ln x)^2\dx = [x(\ln x)^2]_1^e - \int_1^e 2\ln x\dx = e - 2\int_1^e \ln x\dx
\]

מדוגמה קודמת: $\int_1^e \ln x\dx = 1$. לכן:
\[
\int_1^e (\ln x)^2\dx = e - 2 \cdot 1 = e - 2
\]
\end{exbox}

% ===================================
\section{שינוי משתנה — גרסה ראשונה}
% ===================================

\begin{thmbox}
\textbf{טענה 10.9 (שינוי משתנה — גרסה 1).}
תהיינה $f : [a,b] \to \R$ ו־$g : [c,d] \to [a,b]$. נניח כי:
\begin{itemize}
    \item $f$ בעלת קדומה ואינטגרבילית ב־$[a,b]$
    \item $g$ גזירה ב־$[c,d]$
    \item $(f \circ g) \cdot g'$ אינטגרבילית על $[c,d]$
\end{itemize}
אז:
\[
\boxed{\int_c^d (f \circ g)(x) g'(x)\dx = \int_{g(c)}^{g(d)} f(t)\dt}
\]
\end{thmbox}

\begin{notebox}
\textbf{דרך לזכור.}
אם $t = g(x)$, אז $dt = g'(x)dx$. הגבולות משתנים: $x = c \Rightarrow t = g(c)$, $x = d \Rightarrow t = g(d)$.
\end{notebox}

% ===================================
\section{שינוי משתנה — גרסה שנייה}
% ===================================

\begin{thmbox}
\textbf{טענה 10.10 (שינוי משתנה — גרסה 2).}
תהיינה $f : [a,b] \to \R$ ו־$g : [c,d] \to [a,b]$ כך ש:
\begin{itemize}
    \item $f$ אינטגרבילית ב־$[a,b]$
    \item $g$ גזירה והפיכה
    \item $g'$ רציפה
\end{itemize}
אז $(f \circ g) \cdot g'$ אינטגרבילית ומתקיים:
\[
\boxed{\int_a^b f(x)\dx = \int_{g^{-1}(a)}^{g^{-1}(b)} (f \circ g)(t) g'(t)\dt}
\]
\end{thmbox}

\begin{notebox}
\textbf{הערה 10.11.}
מתקיים גם:
\[
\int_{g^{-1}(a)}^{g^{-1}(b)} f(g(t)) |g'(t)|\dt = \int_a^b f(x)\dx
\]
(תלוי בסדר הגבולות ובסימן של $g'$).
\end{notebox}

% ===================================
\section{דוגמאות לשינוי משתנה}
% ===================================

\begin{exbox}
\textbf{דוגמה 1.} חישוב $\int_0^1 \sqrt{1-x^2}\dx$ (שטח רבע מעגל היחידה).

נגדיר $x = \sin t$ על $[0, \frac{\pi}{2}]$. אז:
\begin{itemize}
    \item $dx = \cos t\,dt$
    \item $x = 0 \Rightarrow t = 0$
    \item $x = 1 \Rightarrow t = \frac{\pi}{2}$
\end{itemize}

\textbf{פתרון:}
\[
\int_0^1 \sqrt{1-x^2}\dx = \int_0^{\pi/2} \sqrt{1-\sin^2 t} \cdot \cos t\dt = \int_0^{\pi/2} \cos^2 t\dt
\]

משתמשים בזהות $\cos^2 t = \frac{1 + \cos(2t)}{2}$:
\[
= \int_0^{\pi/2} \frac{1 + \cos(2t)}{2}\dt = \frac{1}{2}\brackets{t + \frac{\sin(2t)}{2}}_0^{\pi/2} = \frac{1}{2}\parens{\frac{\pi}{2} + 0 - 0} = \frac{\pi}{4}
\]
\end{exbox}

\begin{notebox}
\textbf{טעות נפוצה בשינוי משתנה.}

בניסיון לחשב $\int_{-1}^1 \frac{1}{1+x^2}\dx$ עם $g(x) = \frac{1}{x}$:

\textbf{הבעיה:} הפונקציה $g(x) = \frac{1}{x}$ \textbf{אינה מוגדרת} ב־$0 \in [-1,1]$!

יתרה מכך:
\begin{itemize}
    \item $g([-1, 0)) = (-\infty, -1]$
    \item $g((0, 1]) = [1, +\infty)$
\end{itemize}
התחום והטווח אינם תואמים, ואין להפעיל את משפט שינוי המשתנה.

\textbf{הפתרון הנכון:} חישוב ישיר:
\[
\int_{-1}^1 \frac{1}{1+x^2}\dx = [\arctan x]_{-1}^1 = \frac{\pi}{4} - \parens{-\frac{\pi}{4}} = \frac{\pi}{2}
\]
\end{notebox}

\begin{exbox}
\textbf{דוגמה 2.} חישוב $\int_0^{\pi/4} \tan^2 x\dx$.

משתמשים בזהות $\tan^2 x = \sec^2 x - 1$:
\[
\int_0^{\pi/4} \tan^2 x\dx = \int_0^{\pi/4} (\sec^2 x - 1)\dx = [\tan x - x]_0^{\pi/4} = 1 - \frac{\pi}{4}
\]
\end{exbox}

% ===================================
\section{משפט ערך הביניים לאינטגרלים}
% ===================================

\begin{thmbox}
\textbf{טענה 10.12 (משפט ערך הביניים לאינטגרלים).}
תהי $f : [a,b] \to \R$ אינטגרבילית. אז \textbf{קיים $\mu \in [\inf f, \sup f]$} כך ש:
\[
\int_a^b f(x)\dx = \mu(b-a)
\]

\textbf{יתר על כן:} אם $f$ \textbf{רציפה} אז \textbf{קיים $c \in [a,b]$} כך ש:
\[
\boxed{\int_a^b f(x)\dx = f(c)(b-a)}
\]
\end{thmbox}

\begin{proofbox}
\textbf{הוכחה (למקרה הרציף).}
נסמן $m = \inf f$, $M = \sup f$.

מחד:
\[
m(b-a) \le \int_a^b f(x)\dx \le M(b-a)
\]

לכן:
\[
\mu = \frac{\int_a^b f(x)\dx}{b-a} \in [m, M]
\]

אם $f$ רציפה, לפי משפט ערך הביניים של ויירשטראס קיים $c \in [a,b]$ כך ש־$f(c) = \mu$. \hfill $\blacksquare$
\end{proofbox}

\begin{notebox}
\textbf{פרשנות גאומטרית.}
הערך $\mu = \frac{1}{b-a}\int_a^b f(x)\dx$ הוא \textbf{הממוצע} של $f$ על הקטע $[a,b]$.

המשפט אומר שקיימת נקודה $c$ שבה ערך הפונקציה שווה בדיוק לממוצע.
\end{notebox}

% ===================================
\section{משפט ערך הביניים הממושקל}
% ===================================

\begin{thmbox}
\textbf{טענה 10.13 (גרסה ממושקלת).}
תהיינה $f, g : [a,b] \to \R$ אינטגרביליות. נניח כי \textbf{$g \ge 0$}. אז \textbf{קיים $\mu \in [\inf f, \sup f]$} כך ש:
\[
\int_a^b f(x) g(x)\dx = \mu \int_a^b g(x)\dx
\]

\textbf{יתר על כן:} אם \textbf{$f$ רציפה} אז \textbf{קיים $c \in [a,b]$} כך ש:
\[
\boxed{\int_a^b f(x) g(x)\dx = f(c) \int_a^b g(x)\dx}
\]
\end{thmbox}

\begin{exbox}
\textbf{דוגמה.} הוכיחו כי $\frac{2}{3\pi} \le \int_{2\pi}^{3\pi} \frac{\sin x}{x}\dx \le \frac{1}{\pi}$.

\textbf{פתרון:}
נגדיר $f(x) = \frac{1}{x}$, $g(x) = \sin x$ על $[2\pi, 3\pi]$.

שימו לב ש־$\sin x \ge 0$ בקטע $[2\pi, 3\pi]$ (זה לא נכון! $\sin x \ge 0$ רק ב־$[2\pi, 3\pi]$ אם $x \in [2\pi, 3\pi]$... בעצם $\sin(2\pi) = 0$, $\sin(3\pi) = 0$, ובאמצע $\sin x$ עובר דרך ערכים חיוביים ושליליים).

\textbf{תיקון:} נשתמש בטענה 10.13 בקטע $[2\pi, 3\pi]$ כאשר $g(x) = \sin x \ge 0$ עבור $x \in [2\pi, 3\pi]$ (זה נכון כי הקטע $[2\pi, 3\pi]$ מכסה בדיוק חצי מחזור שלם של סינוס מ־0 עד 0 דרך 1).

לפי טענה 10.13 קיים $c \in [2\pi, 3\pi]$ כך ש:
\[
\int_{2\pi}^{3\pi} \frac{\sin x}{x}\dx = \frac{1}{c} \int_{2\pi}^{3\pi} \sin x\dx
\]

נחשב:
\[
\int_{2\pi}^{3\pi} \sin x\dx = [-\cos x]_{2\pi}^{3\pi} = -\cos(3\pi) + \cos(2\pi) = -(-1) + 1 = 2
\]

לכן:
\[
\int_{2\pi}^{3\pi} \frac{\sin x}{x}\dx = \frac{2}{c}
\]

כיוון ש־$c \in [2\pi, 3\pi]$, מתקיים $\frac{1}{3\pi} \le \frac{1}{c} \le \frac{1}{2\pi}$, ומכאן:
\[
\frac{2}{3\pi} \le \int_{2\pi}^{3\pi} \frac{\sin x}{x}\dx \le \frac{2}{2\pi} = \frac{1}{\pi}
\]
\end{exbox}

% ===================================
\section{נוסחת רדוקציה}
% ===================================

\begin{thmbox}
\textbf{נוסחת רדוקציה.}
לכל $n \ge 2$:
\[
\boxed{\int_0^{\pi/2} \sin^n x\dx = \frac{n-1}{n} \int_0^{\pi/2} \sin^{n-2} x\dx}
\]
\end{thmbox}

\begin{proofbox}
\textbf{הוכחה.}
נכתוב:
\[
\int_0^{\pi/2} \sin^n x\dx = \int_0^{\pi/2} \sin^{n-1} x \cdot \sin x\dx
\]

אינטגרציה בחלקים עם $u = \sin^{n-1} x$, $dv = \sin x\dx$:
\[
du = (n-1)\sin^{n-2} x \cos x\dx, \quad v = -\cos x
\]

\[
= \big[-\sin^{n-1} x \cos x\big]_0^{\pi/2} + (n-1)\int_0^{\pi/2} \sin^{n-2} x \cos^2 x\dx
\]

הגבולות מתאפסים. משתמשים ב־$\cos^2 x = 1 - \sin^2 x$:
\[
= (n-1)\int_0^{\pi/2} \sin^{n-2} x\dx - (n-1)\int_0^{\pi/2} \sin^n x\dx
\]

נסמן $I_n = \int_0^{\pi/2} \sin^n x\dx$:
\[
I_n = (n-1)I_{n-2} - (n-1)I_n \quad \Rightarrow \quad nI_n = (n-1)I_{n-2} \quad \Rightarrow \quad I_n = \frac{n-1}{n}I_{n-2}
\]
\hfill $\blacksquare$
\end{proofbox}

% ===================================
\section{תרגילים}
% ===================================

\begin{exercisebox}
\textbf{תרגיל 1.}
חשבו את האינטגרלים הבאים:
\begin{enumerate}
    \item $\int_0^1 x \arctan x\dx$
    \item $\int_0^{\pi/2} \sin^4 x\dx$ (השתמשו בנוסחת הרדוקציה)
    \item $\int_0^1 \frac{x^3}{\sqrt{1+x^2}}\dx$
\end{enumerate}
\end{exercisebox}

\begin{exercisebox}
\textbf{תרגיל 2.}
תהי $f : [a,b] \to \R$ רציפה. הוכיחו כי:
\[
\limx{0^+} \frac{1}{h} \int_a^{a+h} f(x)\dx = f(a)
\]
\end{exercisebox}

\begin{exercisebox}
\textbf{תרגיל 3.}
חשבו $\int_0^1 \frac{\ln(1+x)}{x}\dx$.

\textbf{רמז:} פתחו את $\ln(1+x)$ לטור טיילור ושלבו איבר־איבר.
\end{exercisebox}

\begin{exercisebox}
\textbf{תרגיל 4.}
הוכיחו כי לכל פונקציה רציפה $f : [-a, a] \to \R$:
\begin{enumerate}
    \item אם $f$ זוגית: $\int_{-a}^a f(x)\dx = 2\int_0^a f(x)\dx$
    \item אם $f$ אי־זוגית: $\int_{-a}^a f(x)\dx = 0$
\end{enumerate}
\end{exercisebox}
