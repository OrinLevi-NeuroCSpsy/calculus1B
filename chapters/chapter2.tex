% פרק 2: תכונות המספרים הממשיים
\chapter{תכונות המספרים הממשיים}

\section{ערך מוחלט}

\begin{defbox}
\textbf{הגדרה 2.1 (ערך מוחלט).}
לכל $x \in \R$ נגדיר את \textbf{הערך המוחלט} של $x$:
\[
|x| = \begin{cases}
x & x \ge 0 \\
-x & x < 0
\end{cases}
\]
\end{defbox}

\begin{thmbox}
\textbf{טענה 2.2 (תכונות ערך מוחלט).}
לכל $x, y \in \R$:
\begin{enumerate}
    \item $|x| \ge 0$, ו־$|x| = 0 \iff x = 0$
    \item $|xy| = |x| \cdot |y|$
    \item $\left|\frac{x}{y}\right| = \frac{|x|}{|y|}$ (עבור $y \neq 0$)
    \item $-|x| \le x \le |x|$
    \item \textbf{אי־שוויון המשולש:} $|x + y| \le |x| + |y|$
    \item \textbf{אי־שוויון המשולש ההפוך:} $\big||x| - |y|\big| \le |x - y|$
\end{enumerate}
\end{thmbox}

\begin{proofbox}
\textbf{הוכחת אי־שוויון המשולש.}
מתקיים $-|x| \le x \le |x|$ וגם $-|y| \le y \le |y|$.

בחיבור: $-(|x| + |y|) \le x + y \le |x| + |y|$.

לכן $|x + y| \le |x| + |y|$. \hfill $\blacksquare$
\end{proofbox}

\begin{exbox}
\textbf{דוגמה.} הוכיחו כי לכל $x, y, z \in \R$:
\[
|x - z| \le |x - y| + |y - z|
\]

\textbf{פתרון:} נציב $a = x - y$, $b = y - z$ באי־שוויון המשולש:
\[
|a + b| \le |a| + |b| \implies |(x-y) + (y-z)| \le |x-y| + |y-z| \implies |x-z| \le |x-y| + |y-z| \quad \blacksquare
\]
\end{exbox}

\section{חסמים}

\begin{defbox}
\textbf{הגדרה 2.3 (חסם).}
תהי $A \subseteq \R$ קבוצה לא ריקה.
\begin{itemize}
    \item $M \in \R$ נקרא \textbf{חסם מלעיל} של $A$ אם לכל $a \in A$ מתקיים $a \le M$.
    \item $m \in \R$ נקרא \textbf{חסם מלרע} של $A$ אם לכל $a \in A$ מתקיים $a \ge m$.
\end{itemize}
קבוצה נקראת \textbf{חסומה מלעיל/מלרע} אם יש לה חסם מלעיל/מלרע. קבוצה \textbf{חסומה} אם היא חסומה גם מלעיל וגם מלרע.
\end{defbox}

\begin{defbox}
\textbf{הגדרה 2.4 (מקסימום ומינימום).}
תהי $A \subseteq \R$ קבוצה לא ריקה.
\begin{itemize}
    \item $M \in A$ נקרא \textbf{מקסימום} של $A$ (ונסמן $M = \max A$) אם $M$ חסם מלעיל של $A$.
    \item $m \in A$ נקרא \textbf{מינימום} של $A$ (ונסמן $m = \min A$) אם $m$ חסם מלרע של $A$.
\end{itemize}
\end{defbox}

\begin{notebox}
\textbf{הערה.}
לא לכל קבוצה יש מקסימום או מינימום!

\textbf{דוגמה:} לקבוצה $(0, 1)$ אין מקסימום ואין מינימום.
\end{notebox}

\section{סופרימום ואינפימום}

\begin{defbox}
\textbf{הגדרה 2.5 (סופרימום).}
תהי $A \subseteq \R$ קבוצה לא ריקה וחסומה מלעיל.

$s \in \R$ נקרא \textbf{סופרימום (חסם עליון מינימלי)} של $A$ ונסמן $s = \sup A$ אם:
\begin{enumerate}
    \item $s$ חסם מלעיל של $A$ (לכל $a \in A$: $a \le s$)
    \item $s$ הוא החסם מלעיל \textbf{הקטן ביותר}: לכל $\eps > 0$ קיים $a \in A$ כך ש־$a > s - \eps$
\end{enumerate}
\end{defbox}

\begin{defbox}
\textbf{הגדרה 2.6 (אינפימום).}
תהי $A \subseteq \R$ קבוצה לא ריקה וחסומה מלרע.

$t \in \R$ נקרא \textbf{אינפימום (חסם תחתון מקסימלי)} של $A$ ונסמן $t = \inf A$ אם:
\begin{enumerate}
    \item $t$ חסם מלרע של $A$ (לכל $a \in A$: $a \ge t$)
    \item $t$ הוא החסם מלרע \textbf{הגדול ביותר}: לכל $\eps > 0$ קיים $a \in A$ כך ש־$a < t + \eps$
\end{enumerate}
\end{defbox}

\begin{thmbox}
\textbf{טענה 2.7 (אפיון הסופרימום).}
$s = \sup A$ אם ורק אם מתקיימים שני התנאים:
\begin{enumerate}
    \item לכל $a \in A$: $a \le s$
    \item לכל $\eps > 0$ קיים $a \in A$ כך ש־$a > s - \eps$
\end{enumerate}
\end{thmbox}

\begin{exbox}
\textbf{דוגמה 1.} מצאו $\sup A$ ו־$\inf A$ עבור $A = (0, 1]$.

\textbf{פתרון:}
\begin{itemize}
    \item $\sup A = 1$ (ו־$\max A = 1$ כי $1 \in A$)
    \item $\inf A = 0$ (אבל $\min A$ לא קיים כי $0 \notin A$)
\end{itemize}

\textbf{הוכחה ש־$\inf A = 0$:}
\begin{enumerate}
    \item $0$ חסם מלרע: לכל $x \in (0,1]$ מתקיים $x > 0$.
    \item $0$ הוא החסם הגדול ביותר: לכל $\eps > 0$, נבחר $a = \min(\frac{\eps}{2}, \frac{1}{2}) \in A$. אז $a < 0 + \eps$. $\blacksquare$
\end{enumerate}
\end{exbox}

\begin{exbox}
\textbf{דוגמה 2.} מצאו $\sup A$ עבור $A = \left\{\frac{n}{n+1} : n \in \N\right\} = \left\{0, \frac{1}{2}, \frac{2}{3}, \frac{3}{4}, \ldots\right\}$.

\textbf{פתרון:}
נטען $\sup A = 1$.

\begin{enumerate}
    \item $1$ חסם מלעיל: $\frac{n}{n+1} < 1$ לכל $n \in \N$.
    \item לכל $\eps > 0$, נבחר $n$ כך ש־$\frac{1}{n+1} < \eps$ (כלומר $n > \frac{1}{\eps} - 1$). אז:
    \[
    \frac{n}{n+1} = 1 - \frac{1}{n+1} > 1 - \eps \quad \blacksquare
    \]
\end{enumerate}
\end{exbox}

\section{אקסיומת השלמות}

\begin{thmbox}
\textbf{אקסיומה 2.8 (אקסיומת השלמות).}
לכל קבוצה $A \subseteq \R$ לא ריקה וחסומה מלעיל \textbf{קיים סופרימום}.

באופן שקול: לכל קבוצה לא ריקה וחסומה מלרע קיים אינפימום.
\end{thmbox}

\begin{notebox}
\textbf{הערה חשובה.}
אקסיומת השלמות היא מה שמבדיל את $\R$ מ־$\Q$!

\textbf{דוגמה:} הקבוצה $A = \{x \in \Q : x^2 < 2\}$ חסומה מלעיל ב־$\Q$ (למשל על ידי $2$), אבל אין לה סופרימום ב־$\Q$.

(הסופרימום היה צריך להיות $\sqrt{2}$, אבל $\sqrt{2} \notin \Q$.)
\end{notebox}

\begin{thmbox}
\textbf{משפט 2.9 (תכונת ארכימדס).}
לכל $x, y \in \R$ עם $x > 0$, קיים $n \in \N$ כך ש־$nx > y$.

\textbf{באופן שקול:} לכל $\eps > 0$ קיים $n \in \N$ כך ש־$\frac{1}{n} < \eps$.
\end{thmbox}

\begin{proofbox}
\textbf{הוכחה.}
נניח בשלילה שלכל $n \in \N$: $nx \le y$.

אז הקבוצה $A = \{nx : n \in \N\}$ חסומה מלעיל (על ידי $y$).

לפי אקסיומת השלמות, קיים $s = \sup A$.

כיוון ש־$x > 0$, מתקיים $s - x < s$, ולכן $s - x$ אינו חסם מלעיל של $A$.

לכן קיים $n \in \N$ כך ש־$nx > s - x$, כלומר $(n+1)x > s$.

אבל $(n+1)x \in A$, וזו סתירה לכך ש־$s$ חסם מלעיל של $A$. $\blacksquare$
\end{proofbox}

\begin{thmbox}
\textbf{משפט 2.10 (צפיפות הרציונליים).}
בין כל שני ממשיים שונים קיים מספר רציונלי.

כלומר: לכל $a, b \in \R$ עם $a < b$, קיים $q \in \Q$ כך ש־$a < q < b$.
\end{thmbox}

\begin{proofbox}
\textbf{הוכחה.}
לפי תכונת ארכימדס, קיים $n \in \N_+$ כך ש־$\frac{1}{n} < b - a$, כלומר $1 < n(b-a)$.

נגדיר $m = \lfloor na \rfloor + 1$ (המספר השלם הקטן ביותר הגדול מ־$na$).

אז $na < m \le na + 1$.

מצד אחד: $\frac{m}{n} > a$.

מצד שני: $\frac{m}{n} \le \frac{na + 1}{n} = a + \frac{1}{n} < a + (b - a) = b$.

לכן $a < \frac{m}{n} < b$ והמספר $q = \frac{m}{n}$ רציונלי. $\blacksquare$
\end{proofbox}

\section{תכונות נוספות}

\begin{thmbox}
\textbf{טענה 2.11 (תכונות sup ו־inf).}
יהיו $A, B \subseteq \R$ קבוצות לא ריקות וחסומות.
\begin{enumerate}
    \item $\sup(A \cup B) = \max\{\sup A, \sup B\}$
    \item $\inf(A \cup B) = \min\{\inf A, \inf B\}$
    \item אם $A \subseteq B$ אז $\sup A \le \sup B$ ו־$\inf A \ge \inf B$
    \item $\sup(A + B) = \sup A + \sup B$ כאשר $A + B = \{a + b : a \in A, b \in B\}$
    \item $\sup(-A) = -\inf A$ כאשר $-A = \{-a : a \in A\}$
\end{enumerate}
\end{thmbox}

\begin{proofbox}
\textbf{הוכחה (סעיף 4).}
נסמן $s = \sup A$, $t = \sup B$.

\textbf{שלב 1:} $s + t$ חסם מלעיל של $A + B$.

לכל $a \in A$, $b \in B$: $a \le s$ ו־$b \le t$, לכן $a + b \le s + t$.

\textbf{שלב 2:} $s + t$ הוא החסם הקטן ביותר.

יהי $\eps > 0$. קיימים $a \in A$, $b \in B$ כך ש־$a > s - \frac{\eps}{2}$ ו־$b > t - \frac{\eps}{2}$.

לכן $a + b > s + t - \eps$, והאיבר $a + b \in A + B$. $\blacksquare$
\end{proofbox}

\section{ערך שלם}

\begin{defbox}
\textbf{הגדרה 2.12 (פונקציית הערך השלם).}
לכל $x \in \R$, \textbf{הערך השלם} (או \textbf{פונקציית הרצפה}) $\lfloor x \rfloor$ הוא המספר השלם הגדול ביותר שקטן או שווה ל־$x$:
\[
\lfloor x \rfloor = \max\{n \in \Z : n \le x\}
\]
\end{defbox}

\begin{thmbox}
\textbf{טענה 2.13 (תכונות ערך שלם).}
לכל $x \in \R$:
\begin{enumerate}
    \item $\lfloor x \rfloor \le x < \lfloor x \rfloor + 1$
    \item $x - 1 < \lfloor x \rfloor \le x$
    \item $\lfloor x \rfloor = x \iff x \in \Z$
\end{enumerate}
\end{thmbox}

\section{תרגילים}

\begin{exercisebox}
\textbf{תרגיל 1.}
הוכיחו כי לכל קבוצה לא ריקה $A \subseteq \R$:
\[
\sup A = -\inf(-A)
\]

\textbf{פתרון:}
נסמן $s = \sup A$ ו־$t = \inf(-A)$.

\textbf{צ"ל:} $s = -t$.

$t$ חסם מלרע של $-A$: לכל $-a \in -A$ (כלומר $a \in A$) מתקיים $-a \ge t$, כלומר $a \le -t$.

לכן $-t$ חסם מלעיל של $A$, ומכאן $s \le -t$.

באופן דומה (או מסימטריה): $-s \le t$, כלומר $-t \le s$.

מכאן $s = -t$. $\blacksquare$
\end{exercisebox}

\begin{exercisebox}
\textbf{תרגיל 2.}
תהי $A \subseteq \R$ לא ריקה וחסומה. הוכיחו כי אם $\sup A \notin A$ אז קיימת סדרה $(a_n)$ ב־$A$ כך ש־$\limn a_n = \sup A$.

\textbf{פתרון:}
נסמן $s = \sup A$. לכל $n \in \N_+$, לפי אפיון הסופרימום, קיים $a_n \in A$ כך ש:
\[
s - \frac{1}{n} < a_n \le s
\]
(האי־שוויון $a_n \le s$ נובע מכך ש־$s$ חסם מלעיל, והאי־שוויון $a_n < s$ נובע מכך ש־$s \notin A$).

לפי כלל הסנדוויץ': $\limn a_n = s$. $\blacksquare$
\end{exercisebox}

\begin{exercisebox}
\textbf{תרגיל 3.}
מצאו $\sup A$ ו־$\inf A$ עבור:
\[
A = \left\{\frac{(-1)^n}{n} + \frac{1}{n^2} : n \in \N_+\right\}
\]

\textbf{פתרון:}
נפרק למקרים:
\begin{itemize}
    \item $n$ זוגי: $a_n = \frac{1}{n} + \frac{1}{n^2} > 0$
    \item $n$ אי־זוגי: $a_n = -\frac{1}{n} + \frac{1}{n^2} = \frac{1-n}{n^2}$
\end{itemize}

עבור $n = 1$: $a_1 = -1 + 1 = 0$.

עבור $n = 2$: $a_2 = \frac{1}{2} + \frac{1}{4} = \frac{3}{4}$.

עבור $n$ אי־זוגי $\ge 3$: $a_n = \frac{1-n}{n^2} < 0$.

\textbf{מקסימום:} $a_2 = \frac{3}{4}$ הוא האיבר הגדול ביותר (בודקים שלכל $n \ge 2$ זוגי: $a_n = \frac{1}{n} + \frac{1}{n^2} \le \frac{1}{2} + \frac{1}{4}$).

לכן $\sup A = \max A = \frac{3}{4}$.

\textbf{אינפימום:} עבור $n$ אי־זוגי גדול, $a_n = \frac{1-n}{n^2} \to 0^-$.

המינימום הוא $a_3 = \frac{1-3}{9} = -\frac{2}{9}$.

לכן $\inf A = \min A = -\frac{2}{9}$. $\blacksquare$
\end{exercisebox}
