% פרק 5: פונקציות במשתנה אחד - מושגים בסיסיים
\chapter{פונקציות במשתנה אחד}

\section{פונקציות ממשיות}

\begin{defbox}
\textbf{הגדרה 5.1.}
\textbf{פונקציה ממשית} היא פונקציה $f : D \to \R$ כאשר $D \subseteq \R$ נקרא \textbf{תחום ההגדרה}.
\end{defbox}

\begin{defbox}
\textbf{הגדרה 5.2 (סוגי פונקציות).}
פונקציה $f : D \to \R$ נקראת:
\begin{itemize}
    \item \textbf{חסומה מלעיל}: קיים $M$ כך ש־$f(x) \le M$ לכל $x \in D$
    \item \textbf{חסומה מלרע}: קיים $m$ כך ש־$f(x) \ge m$ לכל $x \in D$
    \item \textbf{חסומה}: חסומה מלעיל ומלרע
    \item \textbf{עולה}: $x_1 < x_2 \Rightarrow f(x_1) \le f(x_2)$
    \item \textbf{עולה ממש}: $x_1 < x_2 \Rightarrow f(x_1) < f(x_2)$
    \item \textbf{זוגית}: $f(-x) = f(x)$ לכל $x$ (והתחום סימטרי)
    \item \textbf{אי־זוגית}: $f(-x) = -f(x)$ לכל $x$
    \item \textbf{מחזורית}: קיים $T > 0$ כך ש־$f(x + T) = f(x)$ לכל $x$
\end{itemize}
\end{defbox}

\section{פונקציות אלמנטריות}

\begin{notebox}
\textbf{הפונקציות האלמנטריות:}
\begin{enumerate}
    \item \textbf{פולינומים:} $p(x) = a_n x^n + \cdots + a_1 x + a_0$
    \item \textbf{פונקציות רציונליות:} $\frac{p(x)}{q(x)}$ (מנת פולינומים)
    \item \textbf{פונקציות טריגונומטריות:} $\sin, \cos, \tan, \cot$
    \item \textbf{פונקציות טריגונומטריות הפוכות:} $\arcsin, \arccos, \arctan$
    \item \textbf{פונקציה מעריכית:} $e^x$ או $a^x$
    \item \textbf{פונקציה לוגריתמית:} $\ln x$ או $\log_a x$
    \item \textbf{חזקה כללית:} $x^a = e^{a \ln x}$ (עבור $x > 0$)
\end{enumerate}
\end{notebox}

\section{פונקציות מיוחדות}

\begin{defbox}
\textbf{פונקציית דיריכלה:}
\[
D(x) = \begin{cases}
1 & x \in \Q \\
0 & x \notin \Q
\end{cases}
\]
פונקציה זו אינה רציפה בשום נקודה!
\end{defbox}

\begin{defbox}
\textbf{פונקציית הסימן:}
\[
\text{sgn}(x) = \begin{cases}
1 & x > 0 \\
0 & x = 0 \\
-1 & x < 0
\end{cases}
\]
\end{defbox}

\begin{defbox}
\textbf{פונקציית הערך השלם (רצפה):}
\[
\lfloor x \rfloor = \max\{n \in \Z : n \le x\}
\]
\end{defbox}

\section{תרגילים}

\begin{exercisebox}
\textbf{תרגיל 1.}
הוכיחו כי לכל פונקציה $f : \R \to \R$ קיימות פונקציה זוגית $g$ ופונקציה אי־זוגית $h$ כך ש־$f = g + h$.

\textbf{פתרון:}
נגדיר:
\[
g(x) = \frac{f(x) + f(-x)}{2}, \quad h(x) = \frac{f(x) - f(-x)}{2}
\]

נבדוק:
\begin{itemize}
    \item $g(-x) = \frac{f(-x) + f(x)}{2} = g(x)$ — זוגית
    \item $h(-x) = \frac{f(-x) - f(x)}{2} = -h(x)$ — אי־זוגית
    \item $g(x) + h(x) = \frac{f(x) + f(-x) + f(x) - f(-x)}{2} = f(x)$ $\blacksquare$
\end{itemize}
\end{exercisebox}
