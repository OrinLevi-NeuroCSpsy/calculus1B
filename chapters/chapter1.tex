% פרק 1: השפה המתמטית ותורת הקבוצות
\chapter{השפה המתמטית ותורת הקבוצות}

\section{מושגים בסיסיים}

\begin{defbox}
\textbf{הגדרה 1.1 (קבוצה).}
\textbf{קבוצה} היא אוסף של אובייקטים הנקראים \textbf{איברים}. אם $x$ הוא איבר בקבוצה $A$, נכתוב $x \in A$. אם $x$ אינו איבר ב־$A$, נכתוב $x \notin A$.
\end{defbox}

\begin{defbox}
\textbf{הגדרה 1.2 (תת־קבוצה).}
קבוצה $A$ נקראת \textbf{תת־קבוצה} של $B$ (ונסמן $A \subseteq B$) אם כל איבר של $A$ הוא גם איבר של $B$:
\[
A \subseteq B \iff \forall x : (x \in A \Rightarrow x \in B)
\]
\end{defbox}

\begin{defbox}
\textbf{הגדרה 1.3 (שוויון קבוצות).}
שתי קבוצות $A$ ו־$B$ \textbf{שוות} (ונסמן $A = B$) אם ורק אם $A \subseteq B$ וגם $B \subseteq A$.
\end{defbox}

\begin{notebox}
\textbf{קבוצות מספרים חשובות:}
\begin{itemize}
    \item $\N = \{0, 1, 2, 3, \ldots\}$ — המספרים הטבעיים
    \item $\Z = \{\ldots, -2, -1, 0, 1, 2, \ldots\}$ — המספרים השלמים
    \item $\Q = \left\{\frac{p}{q} : p \in \Z, q \in \N \setminus \{0\}\right\}$ — המספרים הרציונליים
    \item $\R$ — המספרים הממשיים
\end{itemize}
\end{notebox}

\section{פעולות על קבוצות}

\begin{defbox}
\textbf{הגדרה 1.4 (פעולות קבוצתיות).}
יהיו $A, B$ קבוצות.
\begin{enumerate}
    \item \textbf{איחוד:} $A \cup B = \{x : x \in A \text{ או } x \in B\}$
    \item \textbf{חיתוך:} $A \cap B = \{x : x \in A \text{ וגם } x \in B\}$
    \item \textbf{הפרש:} $A \setminus B = \{x : x \in A \text{ וגם } x \notin B\}$
    \item \textbf{הפרש סימטרי:} $A \triangle B = (A \setminus B) \cup (B \setminus A)$
\end{enumerate}
\end{defbox}

\begin{thmbox}
\textbf{טענה 1.5 (חוקי דה־מורגן).}
יהיו $A, B, C$ קבוצות. אז:
\begin{enumerate}
    \item $C \setminus (A \cup B) = (C \setminus A) \cap (C \setminus B)$
    \item $C \setminus (A \cap B) = (C \setminus A) \cup (C \setminus B)$
\end{enumerate}
\end{thmbox}

\begin{proofbox}
\textbf{הוכחה (סעיף 1).}
\begin{align*}
x \in C \setminus (A \cup B) &\iff x \in C \text{ וגם } x \notin A \cup B \\
&\iff x \in C \text{ וגם } (x \notin A \text{ וגם } x \notin B) \\
&\iff (x \in C \text{ וגם } x \notin A) \text{ וגם } (x \in C \text{ וגם } x \notin B) \\
&\iff x \in (C \setminus A) \cap (C \setminus B) \quad \blacksquare
\end{align*}
\end{proofbox}

\section{פונקציות}

\begin{defbox}
\textbf{הגדרה 1.6 (פונקציה).}
\textbf{פונקציה} $f$ מקבוצה $A$ לקבוצה $B$ (ונסמן $f : A \to B$) היא התאמה שמשייכת לכל איבר $x \in A$ איבר יחיד $f(x) \in B$.
\begin{itemize}
    \item $A$ נקראת \textbf{תחום ההגדרה} (Domain) של $f$
    \item $B$ נקראת \textbf{הטווח} (Codomain) של $f$
    \item \textbf{התמונה} של $f$ היא $\im(f) = \{f(x) : x \in A\} \subseteq B$
\end{itemize}
\end{defbox}

\begin{defbox}
\textbf{הגדרה 1.7 (סוגי פונקציות).}
תהי $f : A \to B$ פונקציה.
\begin{enumerate}
    \item $f$ נקראת \textbf{חד־חד ערכית (חח"ע)} אם לכל $x_1, x_2 \in A$: $f(x_1) = f(x_2) \Rightarrow x_1 = x_2$
    \item $f$ נקראת \textbf{על} אם $\im(f) = B$, כלומר לכל $y \in B$ קיים $x \in A$ כך ש־$f(x) = y$
    \item $f$ נקראת \textbf{הפיכה (חח"ע ועל)} אם היא חח"ע וגם על
\end{enumerate}
\end{defbox}

\begin{thmbox}
\textbf{טענה 1.8.}
פונקציה $f : A \to B$ הפיכה אם ורק אם קיימת פונקציה $g : B \to A$ כך ש־$g \circ f = \text{Id}_A$ וגם $f \circ g = \text{Id}_B$.

במקרה זה $g$ יחידה ונקראת \textbf{הפונקציה ההופכית} של $f$, ומסומנת $f^{-1}$.
\end{thmbox}

\begin{exbox}
\textbf{דוגמה 1.}
הפונקציה $f : \R \to \R$ המוגדרת $f(x) = x^2$ \textbf{אינה חח"ע} כי $f(1) = f(-1) = 1$.

אבל הצמצום $f : [0, \infty) \to [0, \infty)$ הוא חח"ע ועל, ולכן הפיך. הפונקציה ההופכית היא $f^{-1}(x) = \sqrt{x}$.
\end{exbox}

\section{קבוצות סופיות ואינסופיות}

\begin{defbox}
\textbf{הגדרה 1.9.}
קבוצה $A$ נקראת \textbf{סופית} אם היא ריקה או קיים $n \in \N$ כך שקיימת התאמה חח"ע ועל בין $A$ לבין $\{1, 2, \ldots, n\}$.

קבוצה שאינה סופית נקראת \textbf{אינסופית}.
\end{defbox}

\begin{defbox}
\textbf{הגדרה 1.10 (עוצמה).}
שתי קבוצות $A$ ו־$B$ הן \textbf{שוות עוצמה} (ונסמן $|A| = |B|$) אם קיימת פונקציה חח"ע ועל $f : A \to B$.
\end{defbox}

\begin{defbox}
\textbf{הגדרה 1.11 (קבוצה בת מנייה).}
קבוצה $A$ נקראת \textbf{בת מנייה} אם היא סופית או שווה עוצמה ל־$\N$.

קבוצה אינסופית ששווה עוצמה ל־$\N$ נקראת \textbf{בת מנייה אינסופית}.
\end{defbox}

\begin{thmbox}
\textbf{טענה 1.12.}
\begin{enumerate}
    \item $\Z$ בת מנייה.
    \item $\Q$ בת מנייה.
    \item $\R$ \textbf{אינה} בת מנייה (משפט קנטור).
\end{enumerate}
\end{thmbox}

\begin{proofbox}
\textbf{הוכחה (סעיף 1).}
נגדיר $f : \N \to \Z$ על ידי:
\[
f(n) = \begin{cases}
\frac{n}{2} & n \text{ זוגי} \\
-\frac{n+1}{2} & n \text{ אי־זוגי}
\end{cases}
\]
זו התאמה חח"ע ועל: $0 \mapsto 0, 1 \mapsto -1, 2 \mapsto 1, 3 \mapsto -2, 4 \mapsto 2, \ldots$ \hfill $\blacksquare$
\end{proofbox}

\section{תרגילים}

\begin{exercisebox}
\textbf{תרגיל 1.}
הוכיחו כי לכל קבוצות $A, B, C$:
\[
A \cap (B \cup C) = (A \cap B) \cup (A \cap C)
\]

\textbf{פתרון:}
\begin{align*}
x \in A \cap (B \cup C) &\iff x \in A \text{ וגם } x \in B \cup C \\
&\iff x \in A \text{ וגם } (x \in B \text{ או } x \in C) \\
&\iff (x \in A \text{ וגם } x \in B) \text{ או } (x \in A \text{ וגם } x \in C) \\
&\iff x \in (A \cap B) \cup (A \cap C) \quad \blacksquare
\end{align*}
\end{exercisebox}

\begin{exercisebox}
\textbf{תרגיל 2.}
תהי $f : A \to B$ פונקציה. הוכיחו:
\begin{enumerate}
    \item $f$ חח"ע אם ורק אם לכל $C, D \subseteq A$: $f(C \cap D) = f(C) \cap f(D)$
    \item $f$ על אם ורק אם לכל $E \subseteq B$: $f(f^{-1}(E)) = E$
\end{enumerate}

\textbf{פתרון (סעיף 1, כיוון אחד):}
נניח $f$ חח"ע. יהיו $C, D \subseteq A$.

$(\subseteq)$ תמיד מתקיים $f(C \cap D) \subseteq f(C) \cap f(D)$.

$(\supseteq)$ יהי $y \in f(C) \cap f(D)$. אז קיימים $c \in C$, $d \in D$ כך ש־$f(c) = y = f(d)$.

כיוון ש־$f$ חח"ע, $c = d$. לכן $c \in C \cap D$, ומכאן $y = f(c) \in f(C \cap D)$. $\blacksquare$
\end{exercisebox}

\begin{exercisebox}
\textbf{תרגיל 3.}
הוכיחו כי $\Q$ בת מנייה.

\textbf{פתרון:}
נגדיר $f : \Z \times \N_+ \to \Q$ על ידי $f(p, q) = \frac{p}{q}$.

$f$ היא על (כל רציונלי הוא מנה של שלם וטבעי חיובי).

כיוון ש־$\Z$ בת מנייה ו־$\N_+$ בת מנייה, גם $\Z \times \N_+$ בת מנייה (מכפלה של בנות מנייה היא בת מנייה).

תמונה של קבוצה בת מנייה היא בת מנייה, לכן $\Q = \im(f)$ בת מנייה. $\blacksquare$
\end{exercisebox}
