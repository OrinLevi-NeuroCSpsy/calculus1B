% פרק 3: סדרות
\chapter{סדרות}

\section{מושגים בסיסיים}

\begin{defbox}
\textbf{הגדרה 3.1 (סדרה).}
\textbf{סדרה} (של מספרים ממשיים) היא פונקציה $a : \N \to \R$.

במקום $a(n)$ נכתוב $a_n$ ונסמן את הסדרה ב־$(a_n)_{n=0}^{\infty}$ או $(a_n)$.
\end{defbox}

\begin{defbox}
\textbf{הגדרה 3.2 (חסימות).}
סדרה $(a_n)$ נקראת:
\begin{itemize}
    \item \textbf{חסומה מלעיל} אם קיים $M \in \R$ כך שלכל $n$: $a_n \le M$
    \item \textbf{חסומה מלרע} אם קיים $m \in \R$ כך שלכל $n$: $a_n \ge m$
    \item \textbf{חסומה} אם היא חסומה מלעיל ומלרע (שקול: קיים $M > 0$ כך ש־$|a_n| \le M$ לכל $n$)
\end{itemize}
\end{defbox}

\begin{defbox}
\textbf{הגדרה 3.3 (מונוטוניות).}
סדרה $(a_n)$ נקראת:
\begin{itemize}
    \item \textbf{עולה (מונוטונית)}: $a_n \le a_{n+1}$ לכל $n$
    \item \textbf{עולה ממש}: $a_n < a_{n+1}$ לכל $n$
    \item \textbf{יורדת (מונוטונית)}: $a_n \ge a_{n+1}$ לכל $n$
    \item \textbf{יורדת ממש}: $a_n > a_{n+1}$ לכל $n$
\end{itemize}
\end{defbox}

\section{גבול של סדרה}

\begin{defbox}
\textbf{הגדרה 3.4 (גבול של סדרה).}
יהי $L \in \R$. נאמר כי הסדרה $(a_n)$ \textbf{מתכנסת ל־$L$} (או $L$ הוא \textbf{גבול} הסדרה) ונכתוב $\limn a_n = L$ או $a_n \to L$, אם:
\[
\boxed{\forall \eps > 0 \; \exists N \in \N \; \forall n > N : |a_n - L| < \eps}
\]
סדרה שאינה מתכנסת נקראת \textbf{מתבדרת}.
\end{defbox}

\begin{notebox}
\textbf{פירוש אינטואיטיבי.}
$\limn a_n = L$ אם לכל "סביבה" של $L$ (קטע $(L-\eps, L+\eps)$), כמעט כל איברי הסדרה נמצאים בסביבה זו (כלומר, רק מספר סופי של איברים מחוץ לה).
\end{notebox}

\begin{thmbox}
\textbf{טענה 3.5 (יחידות הגבול).}
אם $(a_n)$ מתכנסת, אז הגבול שלה יחיד.
\end{thmbox}

\begin{proofbox}
\textbf{הוכחה.}
נניח $a_n \to L$ וגם $a_n \to L'$ עם $L \neq L'$.

נבחר $\eps = \frac{|L - L'|}{2} > 0$.

קיים $N_1$ כך שלכל $n > N_1$: $|a_n - L| < \eps$.

קיים $N_2$ כך שלכל $n > N_2$: $|a_n - L'| < \eps$.

עבור $n > \max(N_1, N_2)$:
\[
|L - L'| \le |L - a_n| + |a_n - L'| < \eps + \eps = 2\eps = |L - L'|
\]
סתירה! $\blacksquare$
\end{proofbox}

\begin{thmbox}
\textbf{טענה 3.6 (סדרה מתכנסת חסומה).}
כל סדרה מתכנסת היא חסומה.
\end{thmbox}

\begin{proofbox}
\textbf{הוכחה.}
תהי $a_n \to L$. עבור $\eps = 1$, קיים $N$ כך שלכל $n > N$: $|a_n - L| < 1$, כלומר $|a_n| < |L| + 1$.

נגדיר $M = \max\{|a_0|, |a_1|, \ldots, |a_N|, |L| + 1\}$.

אז $|a_n| \le M$ לכל $n \in \N$. $\blacksquare$
\end{proofbox}

\begin{exbox}
\textbf{דוגמה 1.} הוכיחו כי $\limn \frac{1}{n} = 0$.

\textbf{פתרון:}
יהי $\eps > 0$. צ"ל: קיים $N$ כך שלכל $n > N$: $\left|\frac{1}{n} - 0\right| < \eps$.

נבחר $N = \left\lfloor \frac{1}{\eps} \right\rfloor$.

לכל $n > N$: $n > \frac{1}{\eps}$, לכן $\frac{1}{n} < \eps$, כלומר $\left|\frac{1}{n}\right| < \eps$. $\blacksquare$
\end{exbox}

\begin{exbox}
\textbf{דוגמה 2.} הוכיחו כי $\limn \frac{n+1}{n} = 1$.

\textbf{פתרון:}
\[
\left|\frac{n+1}{n} - 1\right| = \left|\frac{1}{n}\right| = \frac{1}{n}
\]
יהי $\eps > 0$. נבחר $N = \left\lfloor \frac{1}{\eps} \right\rfloor$.

לכל $n > N$: $\frac{1}{n} < \eps$. $\blacksquare$
\end{exbox}

\section{אריתמטיקה של גבולות}

\begin{thmbox}
\textbf{משפט 3.7 (אריתמטיקה של גבולות).}
יהיו $(a_n)$, $(b_n)$ סדרות מתכנסות עם $a_n \to L$ ו־$b_n \to M$. אז:
\begin{enumerate}
    \item $\limn (a_n + b_n) = L + M$
    \item $\limn (a_n - b_n) = L - M$
    \item $\limn (a_n \cdot b_n) = L \cdot M$
    \item $\limn (c \cdot a_n) = c \cdot L$ לכל $c \in \R$
    \item אם $M \neq 0$ אז $\limn \frac{a_n}{b_n} = \frac{L}{M}$
\end{enumerate}
\end{thmbox}

\begin{proofbox}
\textbf{הוכחה (מכפלה).}
\[
|a_n b_n - LM| = |a_n b_n - a_n M + a_n M - LM| \le |a_n||b_n - M| + |M||a_n - L|
\]

כיוון ש־$(a_n)$ מתכנסת, היא חסומה: קיים $K$ כך ש־$|a_n| \le K$ לכל $n$.

יהי $\eps > 0$. קיים $N_1$ כך שלכל $n > N_1$: $|a_n - L| < \frac{\eps}{2(|M|+1)}$.

קיים $N_2$ כך שלכל $n > N_2$: $|b_n - M| < \frac{\eps}{2K}$.

לכל $n > \max(N_1, N_2)$:
\[
|a_n b_n - LM| \le K \cdot \frac{\eps}{2K} + |M| \cdot \frac{\eps}{2(|M|+1)} < \frac{\eps}{2} + \frac{\eps}{2} = \eps \quad \blacksquare
\]
\end{proofbox}

\begin{thmbox}
\textbf{משפט 3.8 (כלל הסנדוויץ').}
יהיו $(a_n)$, $(b_n)$, $(c_n)$ סדרות. אם:
\begin{enumerate}
    \item $a_n \le b_n \le c_n$ לכל $n$ גדול מספיק
    \item $\limn a_n = \limn c_n = L$
\end{enumerate}
אז $\limn b_n = L$.
\end{thmbox}

\begin{proofbox}
\textbf{הוכחה.}
יהי $\eps > 0$. קיים $N_1$ כך שלכל $n > N_1$: $|a_n - L| < \eps$, כלומר $L - \eps < a_n$.

קיים $N_2$ כך שלכל $n > N_2$: $|c_n - L| < \eps$, כלומר $c_n < L + \eps$.

קיים $N_3$ כך שלכל $n > N_3$: $a_n \le b_n \le c_n$.

לכל $n > \max(N_1, N_2, N_3)$:
\[
L - \eps < a_n \le b_n \le c_n < L + \eps
\]
לכן $|b_n - L| < \eps$. $\blacksquare$
\end{proofbox}

\begin{exbox}
\textbf{דוגמה.} חשבו $\limn \frac{\sin n}{n}$.

\textbf{פתרון:}
$-1 \le \sin n \le 1$, לכן $-\frac{1}{n} \le \frac{\sin n}{n} \le \frac{1}{n}$.

כיוון ש־$\limn \frac{1}{n} = \limn \left(-\frac{1}{n}\right) = 0$, לפי כלל הסנדוויץ': $\limn \frac{\sin n}{n} = 0$. $\blacksquare$
\end{exbox}

\section{משפט המונוטוניות}

\begin{thmbox}
\textbf{משפט 3.9 (התכנסות סדרה מונוטונית חסומה).}
\begin{enumerate}
    \item סדרה \textbf{עולה וחסומה מלעיל} מתכנסת, וגבולה הוא $\sup\{a_n : n \in \N\}$.
    \item סדרה \textbf{יורדת וחסומה מלרע} מתכנסת, וגבולה הוא $\inf\{a_n : n \in \N\}$.
\end{enumerate}
\end{thmbox}

\begin{proofbox}
\textbf{הוכחה (סעיף 1).}
תהי $(a_n)$ עולה וחסומה מלעיל. נגדיר $L = \sup\{a_n : n \in \N\}$ (קיים לפי אקסיומת השלמות).

יהי $\eps > 0$. לפי אפיון הסופרימום, קיים $N \in \N$ כך ש־$a_N > L - \eps$.

כיוון שהסדרה עולה, לכל $n > N$: $a_n \ge a_N > L - \eps$.

וכיוון ש־$L$ חסם מלעיל: $a_n \le L < L + \eps$.

לכן $|a_n - L| < \eps$ לכל $n > N$. $\blacksquare$
\end{proofbox}

\section{גבולות שימושיים}

\begin{thmbox}
\textbf{טענה 3.10 (גבולות חשובים).}
\begin{enumerate}
    \item $\limn \frac{1}{n^k} = 0$ לכל $k > 0$
    \item $\limn \sqrt[n]{n} = 1$
    \item $\limn \sqrt[n]{a} = 1$ לכל $a > 0$
    \item $\limn q^n = 0$ לכל $|q| < 1$
    \item $\limn \frac{a^n}{n!} = 0$ לכל $a \in \R$
    \item $\limn \frac{n^k}{a^n} = 0$ לכל $k \in \N$ ו־$a > 1$
\end{enumerate}
\end{thmbox}

\begin{proofbox}
\textbf{הוכחה ($\limn \sqrt[n]{n} = 1$).}
נכתוב $\sqrt[n]{n} = 1 + h_n$ כאשר $h_n \ge 0$ (כי $\sqrt[n]{n} \ge 1$ לכל $n \ge 1$).

אז $n = (1 + h_n)^n \ge 1 + \binom{n}{2}h_n^2 = 1 + \frac{n(n-1)}{2}h_n^2$ (מאי־שוויון ברנולי המורחב).

לכן $h_n^2 \le \frac{2(n-1)}{n(n-1)} = \frac{2}{n}$, ומכאן $0 \le h_n \le \sqrt{\frac{2}{n}} \to 0$.

לפי כלל הסנדוויץ': $h_n \to 0$, לכן $\sqrt[n]{n} = 1 + h_n \to 1$. $\blacksquare$
\end{proofbox}

\section{המספר $e$}

\begin{thmbox}
\textbf{משפט 3.11.}
הסדרה $a_n = \left(1 + \frac{1}{n}\right)^n$ מתכנסת.

גבולה מוגדר כ־\textbf{המספר $e$}:
\[
\boxed{e = \limn \left(1 + \frac{1}{n}\right)^n \approx 2.71828...}
\]
\end{thmbox}

\begin{proofbox}
\textbf{הוכחה (רעיון).}
מראים ש־$(a_n)$ עולה וחסומה מלעיל (על ידי $3$).

\textbf{עולה:} משתמשים באי־שוויון בין ממוצע חשבוני לגיאומטרי.

\textbf{חסומה:} $\left(1 + \frac{1}{n}\right)^n < 3$ לכל $n$. $\blacksquare$
\end{proofbox}

\begin{thmbox}
\textbf{טענה 3.12.}
מתקיים גם:
\[
e = \limn \left(1 + \frac{1}{n}\right)^n = \limn \left(1 - \frac{1}{n}\right)^{-n}
\]
ובאופן כללי, לכל סדרה $a_n \to \infty$:
\[
\limn \left(1 + \frac{1}{a_n}\right)^{a_n} = e
\]
\end{thmbox}

\section{סדרות קושי}

\begin{defbox}
\textbf{הגדרה 3.13 (סדרת קושי).}
סדרה $(a_n)$ נקראת \textbf{סדרת קושי} אם:
\[
\forall \eps > 0 \; \exists N \in \N \; \forall m, n > N : |a_m - a_n| < \eps
\]
\end{defbox}

\begin{thmbox}
\textbf{משפט 3.14 (קריטריון קושי להתכנסות).}
סדרה מתכנסת אם ורק אם היא סדרת קושי.
\end{thmbox}

\begin{proofbox}
\textbf{הוכחה ($\Rightarrow$).}
תהי $a_n \to L$. יהי $\eps > 0$.

קיים $N$ כך שלכל $n > N$: $|a_n - L| < \frac{\eps}{2}$.

לכל $m, n > N$:
\[
|a_m - a_n| \le |a_m - L| + |L - a_n| < \frac{\eps}{2} + \frac{\eps}{2} = \eps \quad \blacksquare
\]
\end{proofbox}

\section{גבולות חלקיים, $\limsup$ ו־$\liminf$}

\begin{defbox}
\textbf{הגדרה 3.15 (תת־סדרה).}
\textbf{תת־סדרה} של $(a_n)$ היא סדרה מהצורה $(a_{n_k})_{k=0}^{\infty}$ כאשר $n_0 < n_1 < n_2 < \cdots$ סדרה עולה ממש של אינדקסים.
\end{defbox}

\begin{defbox}
\textbf{הגדרה 3.16 (גבול חלקי).}
$L \in \R$ נקרא \textbf{גבול חלקי} של $(a_n)$ אם קיימת תת־סדרה $(a_{n_k})$ כך ש־$a_{n_k} \to L$.
\end{defbox}

\begin{thmbox}
\textbf{משפט 3.17 (בולצאנו־ויירשטראס).}
לכל סדרה חסומה יש תת־סדרה מתכנסת.
\end{thmbox}

\begin{defbox}
\textbf{הגדרה 3.18 ($\limsup$ ו־$\liminf$).}
תהי $(a_n)$ סדרה חסומה.
\begin{itemize}
    \item $\limsup_{n \to \infty} a_n$ הוא \textbf{הגבול החלקי הגדול ביותר} של $(a_n)$
    \item $\liminf_{n \to \infty} a_n$ הוא \textbf{הגבול החלקי הקטן ביותר} של $(a_n)$
\end{itemize}
\end{defbox}

\begin{thmbox}
\textbf{טענה 3.19 (אפיון $\limsup$).}
$L = \limsup a_n$ אם ורק אם:
\begin{enumerate}
    \item לכל $\eps > 0$ קיימים אינסוף אינדקסים $n$ כך ש־$a_n > L - \eps$
    \item לכל $\eps > 0$ קיימים רק מספר סופי של אינדקסים $n$ כך ש־$a_n > L + \eps$
\end{enumerate}
\end{thmbox}

\begin{thmbox}
\textbf{טענה 3.20.}
סדרה חסומה $(a_n)$ מתכנסת אם ורק אם $\limsup a_n = \liminf a_n$.

במקרה זה $\limn a_n = \limsup a_n = \liminf a_n$.
\end{thmbox}

\section{תרגילים}

\begin{exercisebox}
\textbf{תרגיל 1.}
חשבו $\limn \frac{2n^2 + 3n - 1}{5n^2 - n + 2}$.

\textbf{פתרון:}
\[
\frac{2n^2 + 3n - 1}{5n^2 - n + 2} = \frac{n^2(2 + \frac{3}{n} - \frac{1}{n^2})}{n^2(5 - \frac{1}{n} + \frac{2}{n^2})} = \frac{2 + \frac{3}{n} - \frac{1}{n^2}}{5 - \frac{1}{n} + \frac{2}{n^2}} \to \frac{2 + 0 - 0}{5 - 0 + 0} = \frac{2}{5} \quad \blacksquare
\]
\end{exercisebox}

\begin{exercisebox}
\textbf{תרגיל 2.}
הוכיחו כי הסדרה $a_n = \frac{n!}{n^n}$ מתכנסת ומצאו את גבולה.

\textbf{פתרון:}
נבדוק מונוטוניות:
\[
\frac{a_{n+1}}{a_n} = \frac{(n+1)!}{(n+1)^{n+1}} \cdot \frac{n^n}{n!} = \frac{n+1}{(n+1)^{n+1}} \cdot n^n = \frac{n^n}{(n+1)^n} = \left(\frac{n}{n+1}\right)^n = \left(1 - \frac{1}{n+1}\right)^n
\]

עבור $n$ גדול מספיק: $\left(1 - \frac{1}{n+1}\right)^n < 1$ (כי $\left(1 - \frac{1}{n+1}\right)^{n+1} \to \frac{1}{e} < 1$).

לכן הסדרה יורדת (מ־$n$ מסוים). היא גם חסומה מלרע (על ידי $0$).

לכן מתכנסת. נסמן $L = \limn a_n$.

מהיחס: $L = L \cdot \frac{1}{e}$, כלומר $L(1 - \frac{1}{e}) = 0$, לכן $L = 0$. $\blacksquare$
\end{exercisebox}

\begin{exercisebox}
\textbf{תרגיל 3.}
חשבו $\limn \left(1 + \frac{2}{n}\right)^n$.

\textbf{פתרון:}
\[
\left(1 + \frac{2}{n}\right)^n = \left[\left(1 + \frac{1}{n/2}\right)^{n/2}\right]^2 \to e^2 \quad \blacksquare
\]
\end{exercisebox}

\begin{exercisebox}
\textbf{תרגיל 4.}
תהי $(a_n)$ סדרה המקיימת $a_{n+1} = \frac{1}{2}(a_n + \frac{2}{a_n})$ עם $a_1 = 2$.

הוכיחו כי הסדרה מתכנסת ומצאו את גבולה.

\textbf{פתרון:}
\textbf{שלב 1:} $a_n > 0$ לכל $n$ (באינדוקציה).

\textbf{שלב 2:} $a_n \ge \sqrt{2}$ לכל $n \ge 2$.

$a_{n+1} = \frac{a_n^2 + 2}{2a_n} \ge \sqrt{2}$ (מאי־שוויון AM-GM: $\frac{a_n^2 + 2}{2} \ge \sqrt{a_n^2 \cdot 2} = a_n\sqrt{2}$).

\textbf{שלב 3:} הסדרה יורדת (מ־$n = 2$):
\[
a_{n+1} - a_n = \frac{a_n^2 + 2 - 2a_n^2}{2a_n} = \frac{2 - a_n^2}{2a_n} \le 0 \text{ (כי } a_n \ge \sqrt{2}\text{)}
\]

\textbf{שלב 4:} הסדרה יורדת וחסומה מלרע, לכן מתכנסת.

נסמן $L = \limn a_n$. מהנוסחה: $L = \frac{1}{2}(L + \frac{2}{L})$, לכן $2L = L + \frac{2}{L}$, $L^2 = 2$, $L = \sqrt{2}$. $\blacksquare$
\end{exercisebox}
