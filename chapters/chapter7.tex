% פרק 7: רציפות
\chapter{רציפות}

\section{הגדרת רציפות}

\begin{defbox}
\textbf{הגדרה 7.1 (רציפות בנקודה).}
$f$ \textbf{רציפה בנקודה $x_0$} אם $\limx{x_0} f(x) = f(x_0)$.

כלומר:
\[
\boxed{\forall \eps > 0 \; \exists \delta > 0 : |x - x_0| < \delta \Rightarrow |f(x) - f(x_0)| < \eps}
\]
\end{defbox}

\begin{defbox}
\textbf{הגדרה 7.2 (רציפות בקטע).}
$f$ \textbf{רציפה בקטע $I$} אם היא רציפה בכל נקודה של $I$.

(בקצוות סגורים — רציפות חד־צדדית.)
\end{defbox}

\begin{thmbox}
\textbf{טענה 7.3 (אפיון היינה לרציפות).}
$f$ רציפה ב־$x_0$ אם ורק אם לכל סדרה $x_n \to x_0$ מתקיים $f(x_n) \to f(x_0)$.
\end{thmbox}

\begin{exbox}
\textbf{דוגמה 1.}
פונקציית דיריכלה $D(x)$ אינה רציפה בשום נקודה.

\textbf{הוכחה:}
לכל $x_0 \in \R$, קיימות סדרות $r_n \to x_0$ (רציונליות) ו־$i_n \to x_0$ (אי־רציונליות).

$D(r_n) = 1$ לכל $n$, אבל $D(i_n) = 0$ לכל $n$.

לכן הגבול $\limx{x_0} D(x)$ לא קיים. $\blacksquare$
\end{exbox}

\section{אריתמטיקה של פונקציות רציפות}

\begin{thmbox}
\textbf{משפט 7.4.}
אם $f, g$ רציפות ב־$x_0$ אז גם:
\begin{enumerate}
    \item $f + g$ רציפה ב־$x_0$
    \item $f \cdot g$ רציפה ב־$x_0$
    \item $\frac{f}{g}$ רציפה ב־$x_0$ (אם $g(x_0) \neq 0$)
    \item $f \circ g$ רציפה ב־$x_0$ (אם $g$ רציפה ב־$x_0$ ו־$f$ רציפה ב־$g(x_0)$)
\end{enumerate}
\end{thmbox}

\begin{thmbox}
\textbf{טענה 7.5.}
כל פונקציה אלמנטרית רציפה בתחום הגדרתה.
\end{thmbox}

\section{משפט ערך הביניים}

\begin{thmbox}
\textbf{משפט 7.6 (משפט ערך הביניים — IVT).}
תהי $f : [a, b] \to \R$ רציפה. אם $f(a) < c < f(b)$ (או $f(b) < c < f(a)$), אז קיים $x_0 \in (a, b)$ כך ש־$f(x_0) = c$.
\end{thmbox}

\begin{proofbox}
\textbf{הוכחה (רעיון).}
נגדיר $A = \{x \in [a, b] : f(x) < c\}$.

$A$ לא ריקה (כי $a \in A$) וחסומה מלעיל (על ידי $b$).

נגדיר $x_0 = \sup A$.

מראים ש־$f(x_0) = c$ (אם $f(x_0) < c$ או $f(x_0) > c$ מגיעים לסתירה). $\blacksquare$
\end{proofbox}

\begin{exbox}
\textbf{דוגמה (קיום שורש).}
לכל פולינום ממעלה אי־זוגית יש שורש ממשי.

\textbf{הוכחה:}
יהי $p(x) = x^n + a_{n-1}x^{n-1} + \cdots + a_0$ עם $n$ אי־זוגי.

$\limx{\infty} p(x) = \infty$ ו־$\limx{-\infty} p(x) = -\infty$.

לכן קיימים $a < 0 < b$ כך ש־$p(a) < 0 < p(b)$.

לפי IVT, קיים $x_0 \in (a, b)$ כך ש־$p(x_0) = 0$. $\blacksquare$
\end{exbox}

\section{משפט ויירשטראס}

\begin{defbox}
\textbf{הגדרה 7.7 (קטע קומפקטי).}
קטע $[a, b]$ סגור וחסום נקרא \textbf{קומפקטי}.
\end{defbox}

\begin{thmbox}
\textbf{משפט 7.8 (ויירשטראס — קיום מקסימום ומינימום).}
תהי $f : [a, b] \to \R$ רציפה בקטע סגור וחסום. אז:
\begin{enumerate}
    \item $f$ חסומה.
    \item $f$ מקבלת מקסימום ומינימום: קיימים $x_1, x_2 \in [a, b]$ כך ש־$f(x_1) = \min f$ ו־$f(x_2) = \max f$.
\end{enumerate}
\end{thmbox}

\begin{proofbox}
\textbf{הוכחה (סעיף 2, קיום מקסימום).}
$f$ חסומה, לכן קיים $M = \sup\{f(x) : x \in [a,b]\}$.

לכל $n$, קיים $x_n \in [a, b]$ כך ש־$f(x_n) > M - \frac{1}{n}$.

$(x_n)$ סדרה חסומה ב־$[a, b]$, לכן (לפי בולצאנו־ויירשטראס) קיימת תת־סדרה $x_{n_k} \to x_0 \in [a, b]$.

מרציפות $f$: $f(x_{n_k}) \to f(x_0)$.

גם $f(x_{n_k}) \to M$ (כי $M - \frac{1}{n_k} < f(x_{n_k}) \le M$).

מיחידות הגבול: $f(x_0) = M$. $\blacksquare$
\end{proofbox}

\section{רציפות במידה שווה}

\begin{defbox}
\textbf{הגדרה 7.9 (רציפות במידה שווה).}
$f : D \to \R$ \textbf{רציפה במידה שווה} אם:
\[
\forall \eps > 0 \; \exists \delta > 0 \; \forall x, y \in D : |x - y| < \delta \Rightarrow |f(x) - f(y)| < \eps
\]
\end{defbox}

\begin{notebox}
\textbf{ההבדל:}
\begin{itemize}
    \item \textbf{רציפות:} $\delta$ תלוי ב־$\eps$ \textbf{וב־$x_0$}
    \item \textbf{רציפות במידה שווה:} $\delta$ תלוי \textbf{רק ב־$\eps$}, עובד לכל הנקודות
\end{itemize}
\end{notebox}

\begin{thmbox}
\textbf{משפט 7.10 (היינה־קנטור).}
אם $f : [a, b] \to \R$ רציפה בקטע סגור וחסום, אז $f$ רציפה במידה שווה.
\end{thmbox}

\begin{exbox}
\textbf{דוגמה (לא רציפה במידה שווה).}
$f(x) = \frac{1}{x}$ על $(0, 1]$ \textbf{אינה} רציפה במידה שווה.

\textbf{הוכחה:}
נבחר $\eps = 1$. לכל $\delta > 0$, נבחר $x = \delta$, $y = \frac{\delta}{2}$.

אז $|x - y| = \frac{\delta}{2} < \delta$, אבל:
\[
|f(x) - f(y)| = \left|\frac{1}{\delta} - \frac{2}{\delta}\right| = \frac{1}{\delta}
\]
עבור $\delta$ קטן מספיק, $\frac{1}{\delta} > 1 = \eps$. $\blacksquare$
\end{exbox}

\section{פונקציות הפיכות}

\begin{thmbox}
\textbf{משפט 7.11.}
תהי $f : [a, b] \to \R$ רציפה ומונוטונית ממש. אז:
\begin{enumerate}
    \item $f$ הפיכה
    \item $f^{-1} : f([a, b]) \to [a, b]$ רציפה
\end{enumerate}
\end{thmbox}

\section{תרגילים}

\begin{exercisebox}
\textbf{תרגיל 1.}
הוכיחו כי המשוואה $x^5 + x = 1$ יש לה פתרון יחיד בקטע $[0, 1]$.

\textbf{פתרון:}
נגדיר $f(x) = x^5 + x - 1$.

$f(0) = -1 < 0$, $f(1) = 1 > 0$.

לפי IVT, קיים $x_0 \in (0, 1)$ כך ש־$f(x_0) = 0$.

\textbf{יחידות:} $f'(x) = 5x^4 + 1 > 0$ לכל $x$, לכן $f$ עולה ממש, ולכן יש לכל היותר פתרון אחד. $\blacksquare$
\end{exercisebox}

\begin{exercisebox}
\textbf{תרגיל 2.}
תהי $f : [0, 1] \to [0, 1]$ רציפה. הוכיחו שקיימת נקודת שבת, כלומר $x_0 \in [0, 1]$ כך ש־$f(x_0) = x_0$.

\textbf{פתרון:}
נגדיר $g(x) = f(x) - x$.

$g(0) = f(0) - 0 = f(0) \ge 0$ (כי $f(0) \in [0, 1]$).

$g(1) = f(1) - 1 \le 0$ (כי $f(1) \in [0, 1]$).

אם $g(0) = 0$ אז $x_0 = 0$ נקודת שבת.

אם $g(1) = 0$ אז $x_0 = 1$ נקודת שבת.

אחרת, $g(0) > 0 > g(1)$, ולפי IVT קיים $x_0 \in (0, 1)$ כך ש־$g(x_0) = 0$, כלומר $f(x_0) = x_0$. $\blacksquare$
\end{exercisebox}

\begin{exercisebox}
\textbf{תרגיל 3.}
הוכיחו כי $f(x) = \sqrt{x}$ רציפה במידה שווה על $[0, \infty)$.

\textbf{פתרון:}
יהי $\eps > 0$.

\textbf{מקרה 1:} $x, y \ge \eps^2$. אז:
\[
|\sqrt{x} - \sqrt{y}| = \frac{|x - y|}{\sqrt{x} + \sqrt{y}} \le \frac{|x - y|}{2\eps}
\]
נבחר $\delta_1 = 2\eps^2$, אז $|x - y| < \delta_1 \Rightarrow |\sqrt{x} - \sqrt{y}| < \eps$.

\textbf{מקרה 2:} $x$ או $y$ קטנים מ־$\eps^2$. אז $|x - y| < \eps^2 \Rightarrow |\sqrt{x} - \sqrt{y}| < \eps$ (כי $\sqrt{a} < \eps$ אם $a < \eps^2$).

נבחר $\delta = \min(\delta_1, \eps^2) = \eps^2$. $\blacksquare$
\end{exercisebox}
