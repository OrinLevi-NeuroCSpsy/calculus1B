
% ===============================
% Language (Hebrew + English)
% ===============================
\usepackage{fontspec}
\usepackage{polyglossia}
\setmainlanguage{hebrew}
\setotherlanguage{english}
\newfontfamily\hebrewfont{David CLM}

% ===============================
% Mathematics
% ===============================
\usepackage{amsmath, amssymb, amsthm}
\usepackage{mathtools}

% ===============================
% Page Layout
% ===============================
\usepackage[a4paper,margin=2.5cm]{geometry}

% ===============================
% Lists
% ===============================
\usepackage{enumitem}
\setlist[itemize]{itemsep=0.3em, label=\textbullet}
\renewcommand{\labelitemi}{\ensuremath{\circ}}
\renewcommand{\labelitemii}{\ensuremath{-}}

% ===============================
% Tables and Colors
% ===============================
\usepackage[table,xcdraw]{xcolor}
\usepackage{longtable}
\usepackage{booktabs}
\usepackage{colortbl}
\usepackage{array}

% סגנון לטבלאות יפות
\newcommand{\truthmark}[1]{\textbf{#1}}
\newcolumntype{C}{>{\centering\arraybackslash}m{1.5cm}}

% צבעים לטבלאות - טורקיז פסטלי (מתאים לחדו"א)
\definecolor{tableheader}{RGB}{0,128,128}
\definecolor{tablerow1}{RGB}{224,255,255}
\definecolor{tablerow2}{RGB}{240,255,255}
\definecolor{tableborder}{RGB}{0,100,100}

% הגדרות מסגרת לטבלאות
\setlength{\arrayrulewidth}{1.5pt}
\arrayrulecolor{tableborder}

% ===============================
% Graphics (TikZ + pgfplots)
% ===============================
\usepackage{float}
\usepackage{caption}
\usepackage{pifont}
\usepackage{pgfplots}
\pgfplotsset{compat=1.18}

\usepackage{tikz}
\usetikzlibrary{shapes.geometric, arrows.meta, positioning, calc, decorations.pathreplacing}

% ===============================
% Colored Boxes (mdframed - works with Hebrew)
% ===============================
\usepackage[framemethod=tikz]{mdframed}

% הגדרה - כחול
\newmdenv[
  linecolor=blue!75!black,
  backgroundcolor=blue!5,
  linewidth=2pt,
  roundcorner=5pt,
  innertopmargin=10pt,
  innerbottommargin=10pt,
  innerrightmargin=10pt,
  innerleftmargin=10pt,
  skipabove=12pt,
  skipbelow=12pt,
  nobreak=true
]{defbox}

% משפט - ירוק
\newmdenv[
  linecolor=green!60!black,
  backgroundcolor=green!5,
  linewidth=2pt,
  roundcorner=5pt,
  innertopmargin=10pt,
  innerbottommargin=10pt,
  innerrightmargin=10pt,
  innerleftmargin=10pt,
  skipabove=12pt,
  skipbelow=12pt,
  nobreak=true
]{thmbox}

% דוגמה - כתום
\newmdenv[
  linecolor=orange!75!black,
  backgroundcolor=orange!5,
  linewidth=2pt,
  roundcorner=5pt,
  innertopmargin=10pt,
  innerbottommargin=10pt,
  innerrightmargin=10pt,
  innerleftmargin=10pt,
  skipabove=12pt,
  skipbelow=12pt,
  nobreak=true
]{exbox}

% הערה - צהוב
\newmdenv[
  linecolor=yellow!75!black,
  backgroundcolor=yellow!10,
  linewidth=2pt,
  roundcorner=5pt,
  innertopmargin=10pt,
  innerbottommargin=10pt,
  innerrightmargin=10pt,
  innerleftmargin=10pt,
  skipabove=12pt,
  skipbelow=12pt,
  nobreak=true
]{notebox}

% הוכחה - אפור
\newmdenv[
  linecolor=gray!75!black,
  backgroundcolor=gray!5,
  linewidth=2pt,
  roundcorner=5pt,
  innertopmargin=10pt,
  innerbottommargin=10pt,
  innerrightmargin=10pt,
  innerleftmargin=10pt,
  skipabove=12pt,
  skipbelow=12pt,
  nobreak=true
]{proofbox}

% תרגיל - סגול
\newmdenv[
  linecolor=purple!75!black,
  backgroundcolor=purple!5,
  linewidth=2pt,
  roundcorner=5pt,
  innertopmargin=10pt,
  innerbottommargin=10pt,
  innerrightmargin=10pt,
  innerleftmargin=10pt,
  skipabove=12pt,
  skipbelow=12pt,
  nobreak=true
]{exercisebox}

% Theorem Environments are defined in macros.tex

% ===============================
% Custom Commands - Calculus (non-duplicate)
% ===============================
\newcommand{\parens}[1]{\left(#1\right)}
\newcommand{\brackets}[1]{\left[#1\right]}
\newcommand{\dd}{\,\mathrm{d}}
\newcommand{\dx}{\,\mathrm{d}x}
\newcommand{\dt}{\,\mathrm{d}t}
\newcommand{\du}{\,\mathrm{d}u}

% Limits and integrals
\newcommand{\limn}{\lim_{n \to \infty}}
\newcommand{\limx}[1]{\lim_{x \to #1}}
\newcommand{\intab}{\int_a^b}
\newcommand{\intinf}{\int_a^{+\infty}}

% Functions
\newcommand{\dom}{\text{Dom}}
\newcommand{\range}{\text{Range}}
\newcommand{\im}{\text{Im}}

% Check/X marks
\newcommand{\xmark}{\ding{55}}
\newcommand{\cmark}{\ding{51}}

% ===============================
% Hyperref (must be last)
% ===============================
\usepackage{hyperref}
\hypersetup{
  colorlinks=true,
  linkcolor=blue,
  citecolor=green,
  filecolor=magenta,
  urlcolor=cyan
}
