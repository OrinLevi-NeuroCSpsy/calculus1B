% !TeX program = xelatex
\documentclass[12pt, a4paper]{book}


% ===============================
% Language (Hebrew + English)
% ===============================
\usepackage{fontspec}
\usepackage{polyglossia}
\setmainlanguage{hebrew}
\setotherlanguage{english}
\newfontfamily\hebrewfont{David CLM}

% ===============================
% Mathematics
% ===============================
\usepackage{amsmath, amssymb, amsthm}
\usepackage{mathtools}

% ===============================
% Page Layout
% ===============================
\usepackage[a4paper,margin=2.5cm]{geometry}

% ===============================
% Lists
% ===============================
\usepackage{enumitem}
\setlist[itemize]{itemsep=0.3em, label=\textbullet}
\renewcommand{\labelitemi}{\ensuremath{\circ}}
\renewcommand{\labelitemii}{\ensuremath{-}}

% ===============================
% Tables and Colors
% ===============================
\usepackage[table,xcdraw]{xcolor}
\usepackage{longtable}
\usepackage{booktabs}
\usepackage{colortbl}
\usepackage{array}

% סגנון לטבלאות יפות
\newcommand{\truthmark}[1]{\textbf{#1}}
\newcolumntype{C}{>{\centering\arraybackslash}m{1.5cm}}

% צבעים לטבלאות - טורקיז פסטלי (מתאים לחדו"א)
\definecolor{tableheader}{RGB}{0,128,128}
\definecolor{tablerow1}{RGB}{224,255,255}
\definecolor{tablerow2}{RGB}{240,255,255}
\definecolor{tableborder}{RGB}{0,100,100}

% הגדרות מסגרת לטבלאות
\setlength{\arrayrulewidth}{1.5pt}
\arrayrulecolor{tableborder}

% ===============================
% Graphics (TikZ + pgfplots)
% ===============================
\usepackage{float}
\usepackage{caption}
\usepackage{pifont}
\usepackage{pgfplots}
\pgfplotsset{compat=1.18}

\usepackage{tikz}
\usetikzlibrary{shapes.geometric, arrows.meta, positioning, calc, decorations.pathreplacing}

% ===============================
% Colored Boxes (mdframed - works with Hebrew)
% ===============================
\usepackage[framemethod=tikz]{mdframed}

% הגדרה - כחול
\newmdenv[
  linecolor=blue!75!black,
  backgroundcolor=blue!5,
  linewidth=2pt,
  roundcorner=5pt,
  innertopmargin=10pt,
  innerbottommargin=10pt,
  innerrightmargin=10pt,
  innerleftmargin=10pt,
  skipabove=12pt,
  skipbelow=12pt,
  nobreak=true
]{defbox}

% משפט - ירוק
\newmdenv[
  linecolor=green!60!black,
  backgroundcolor=green!5,
  linewidth=2pt,
  roundcorner=5pt,
  innertopmargin=10pt,
  innerbottommargin=10pt,
  innerrightmargin=10pt,
  innerleftmargin=10pt,
  skipabove=12pt,
  skipbelow=12pt,
  nobreak=true
]{thmbox}

% דוגמה - כתום
\newmdenv[
  linecolor=orange!75!black,
  backgroundcolor=orange!5,
  linewidth=2pt,
  roundcorner=5pt,
  innertopmargin=10pt,
  innerbottommargin=10pt,
  innerrightmargin=10pt,
  innerleftmargin=10pt,
  skipabove=12pt,
  skipbelow=12pt,
  nobreak=true
]{exbox}

% הערה - צהוב
\newmdenv[
  linecolor=yellow!75!black,
  backgroundcolor=yellow!10,
  linewidth=2pt,
  roundcorner=5pt,
  innertopmargin=10pt,
  innerbottommargin=10pt,
  innerrightmargin=10pt,
  innerleftmargin=10pt,
  skipabove=12pt,
  skipbelow=12pt,
  nobreak=true
]{notebox}

% הוכחה - אפור
\newmdenv[
  linecolor=gray!75!black,
  backgroundcolor=gray!5,
  linewidth=2pt,
  roundcorner=5pt,
  innertopmargin=10pt,
  innerbottommargin=10pt,
  innerrightmargin=10pt,
  innerleftmargin=10pt,
  skipabove=12pt,
  skipbelow=12pt,
  nobreak=true
]{proofbox}

% תרגיל - סגול
\newmdenv[
  linecolor=purple!75!black,
  backgroundcolor=purple!5,
  linewidth=2pt,
  roundcorner=5pt,
  innertopmargin=10pt,
  innerbottommargin=10pt,
  innerrightmargin=10pt,
  innerleftmargin=10pt,
  skipabove=12pt,
  skipbelow=12pt,
  nobreak=true
]{exercisebox}

% Theorem Environments are defined in macros.tex

% ===============================
% Custom Commands - Calculus (non-duplicate)
% ===============================
\newcommand{\parens}[1]{\left(#1\right)}
\newcommand{\brackets}[1]{\left[#1\right]}
\newcommand{\dd}{\,\mathrm{d}}
\newcommand{\dx}{\,\mathrm{d}x}
\newcommand{\dt}{\,\mathrm{d}t}
\newcommand{\du}{\,\mathrm{d}u}

% Limits and integrals
\newcommand{\limn}{\lim_{n \to \infty}}
\newcommand{\limx}[1]{\lim_{x \to #1}}
\newcommand{\intab}{\int_a^b}
\newcommand{\intinf}{\int_a^{+\infty}}

% Functions
\newcommand{\dom}{\text{Dom}}
\newcommand{\range}{\text{Range}}
\newcommand{\im}{\text{Im}}

% Check/X marks
\newcommand{\xmark}{\ding{55}}
\newcommand{\cmark}{\ding{51}}

% ===============================
% Hyperref (must be last)
% ===============================
\usepackage{hyperref}
\hypersetup{
  colorlinks=true,
  linkcolor=blue,
  citecolor=green,
  filecolor=magenta,
  urlcolor=cyan
}

% ===============================
% Macros — חדו״א 1ב
% ===============================

% כותרות דו-לשוניות (עברית + English) עם PDF bookmarks תקינים
\newcommand{\hebeng}[2]{#1 \texorpdfstring{\LRE{(#2)}}{(#2)}}

% קבוצות מספרים
\newcommand{\N}{\mathbb{N}}
\newcommand{\Z}{\mathbb{Z}}
\newcommand{\Q}{\mathbb{Q}}
\newcommand{\R}{\mathbb{R}}
\newcommand{\C}{\mathbb{C}}

% סימונים בסיסיים
\newcommand{\eps}{\varepsilon}
\newcommand{\abs}[1]{\left\lvert #1 \right\rvert}
\newcommand{\set}[1]{\left\{ #1 \right\}}
\newcommand{\paren}[1]{\left( #1 \right)}

% סביבות (חדו״א – הגדרות ומשפטים)
\theoremstyle{definition}
\newtheorem{definition}{הגדרה}[section]
\newtheorem{example}{דוגמה}[section]

\theoremstyle{plain}
\newtheorem{theorem}{משפט}[section]
\newtheorem{lemma}{למה}[section]

\begin{document}

% עמוד שער
\begin{titlepage}
\begin{center}
\vspace*{2cm}

\begin{english}
{\Huge \textbf{Calculus 1B}}\\[0.5cm]
{\Large Complete Course Summary}\\[1cm]
{\large Orin Levi}\\[0.5cm]
\end{english}

\vspace{1cm}

{\Huge \textbf{חשבון אינפיניטסימלי 1ב}}\\[0.5cm]
{\Large סיכום מקיף לקורס}\\[1cm]
{\large אוניברסיטת תל אביב}\\[0.3cm]
{\large מדעי המחשב}\\[2cm]

\vfill

{\large \today}

\end{center}
\end{titlepage}

\setcounter{tocdepth}{2}
\tableofcontents
\clearpage

% ===================================
% חלק I: תורת הקבוצות והמספרים הממשיים
% ===================================
\part{תורת הקבוצות והמספרים הממשיים}

% פרק 1: השפה המתמטית ותורת הקבוצות
\chapter{השפה המתמטית ותורת הקבוצות}

\section{מושגים בסיסיים}

\begin{defbox}
\textbf{הגדרה 1.1 (קבוצה).}
\textbf{קבוצה} היא אוסף של אובייקטים הנקראים \textbf{איברים}. אם $x$ הוא איבר בקבוצה $A$, נכתוב $x \in A$. אם $x$ אינו איבר ב־$A$, נכתוב $x \notin A$.
\end{defbox}

\begin{defbox}
\textbf{הגדרה 1.2 (תת־קבוצה).}
קבוצה $A$ נקראת \textbf{תת־קבוצה} של $B$ (ונסמן $A \subseteq B$) אם כל איבר של $A$ הוא גם איבר של $B$:
\[
A \subseteq B \iff \forall x : (x \in A \Rightarrow x \in B)
\]
\end{defbox}

\begin{defbox}
\textbf{הגדרה 1.3 (שוויון קבוצות).}
שתי קבוצות $A$ ו־$B$ \textbf{שוות} (ונסמן $A = B$) אם ורק אם $A \subseteq B$ וגם $B \subseteq A$.
\end{defbox}

\begin{notebox}
\textbf{קבוצות מספרים חשובות:}
\begin{itemize}
    \item $\N = \{0, 1, 2, 3, \ldots\}$ — המספרים הטבעיים
    \item $\Z = \{\ldots, -2, -1, 0, 1, 2, \ldots\}$ — המספרים השלמים
    \item $\Q = \left\{\frac{p}{q} : p \in \Z, q \in \N \setminus \{0\}\right\}$ — המספרים הרציונליים
    \item $\R$ — המספרים הממשיים
\end{itemize}
\end{notebox}

\section{פעולות על קבוצות}

\begin{defbox}
\textbf{הגדרה 1.4 (פעולות קבוצתיות).}
יהיו $A, B$ קבוצות.
\begin{enumerate}
    \item \textbf{איחוד:} $A \cup B = \{x : x \in A \text{ או } x \in B\}$
    \item \textbf{חיתוך:} $A \cap B = \{x : x \in A \text{ וגם } x \in B\}$
    \item \textbf{הפרש:} $A \setminus B = \{x : x \in A \text{ וגם } x \notin B\}$
    \item \textbf{הפרש סימטרי:} $A \triangle B = (A \setminus B) \cup (B \setminus A)$
\end{enumerate}
\end{defbox}

\begin{thmbox}
\textbf{טענה 1.5 (חוקי דה־מורגן).}
יהיו $A, B, C$ קבוצות. אז:
\begin{enumerate}
    \item $C \setminus (A \cup B) = (C \setminus A) \cap (C \setminus B)$
    \item $C \setminus (A \cap B) = (C \setminus A) \cup (C \setminus B)$
\end{enumerate}
\end{thmbox}

\begin{proofbox}
\textbf{הוכחה (סעיף 1).}
\begin{align*}
x \in C \setminus (A \cup B) &\iff x \in C \text{ וגם } x \notin A \cup B \\
&\iff x \in C \text{ וגם } (x \notin A \text{ וגם } x \notin B) \\
&\iff (x \in C \text{ וגם } x \notin A) \text{ וגם } (x \in C \text{ וגם } x \notin B) \\
&\iff x \in (C \setminus A) \cap (C \setminus B) \quad \blacksquare
\end{align*}
\end{proofbox}

\section{פונקציות}

\begin{defbox}
\textbf{הגדרה 1.6 (פונקציה).}
\textbf{פונקציה} $f$ מקבוצה $A$ לקבוצה $B$ (ונסמן $f : A \to B$) היא התאמה שמשייכת לכל איבר $x \in A$ איבר יחיד $f(x) \in B$.
\begin{itemize}
    \item $A$ נקראת \textbf{תחום ההגדרה} (Domain) של $f$
    \item $B$ נקראת \textbf{הטווח} (Codomain) של $f$
    \item \textbf{התמונה} של $f$ היא $\im(f) = \{f(x) : x \in A\} \subseteq B$
\end{itemize}
\end{defbox}

\begin{defbox}
\textbf{הגדרה 1.7 (סוגי פונקציות).}
תהי $f : A \to B$ פונקציה.
\begin{enumerate}
    \item $f$ נקראת \textbf{חד־חד ערכית (חח"ע)} אם לכל $x_1, x_2 \in A$: $f(x_1) = f(x_2) \Rightarrow x_1 = x_2$
    \item $f$ נקראת \textbf{על} אם $\im(f) = B$, כלומר לכל $y \in B$ קיים $x \in A$ כך ש־$f(x) = y$
    \item $f$ נקראת \textbf{הפיכה (חח"ע ועל)} אם היא חח"ע וגם על
\end{enumerate}
\end{defbox}

\begin{thmbox}
\textbf{טענה 1.8.}
פונקציה $f : A \to B$ הפיכה אם ורק אם קיימת פונקציה $g : B \to A$ כך ש־$g \circ f = \text{Id}_A$ וגם $f \circ g = \text{Id}_B$.

במקרה זה $g$ יחידה ונקראת \textbf{הפונקציה ההופכית} של $f$, ומסומנת $f^{-1}$.
\end{thmbox}

\begin{exbox}
\textbf{דוגמה 1.}
הפונקציה $f : \R \to \R$ המוגדרת $f(x) = x^2$ \textbf{אינה חח"ע} כי $f(1) = f(-1) = 1$.

אבל הצמצום $f : [0, \infty) \to [0, \infty)$ הוא חח"ע ועל, ולכן הפיך. הפונקציה ההופכית היא $f^{-1}(x) = \sqrt{x}$.
\end{exbox}

\section{קבוצות סופיות ואינסופיות}

\begin{defbox}
\textbf{הגדרה 1.9.}
קבוצה $A$ נקראת \textbf{סופית} אם היא ריקה או קיים $n \in \N$ כך שקיימת התאמה חח"ע ועל בין $A$ לבין $\{1, 2, \ldots, n\}$.

קבוצה שאינה סופית נקראת \textbf{אינסופית}.
\end{defbox}

\begin{defbox}
\textbf{הגדרה 1.10 (עוצמה).}
שתי קבוצות $A$ ו־$B$ הן \textbf{שוות עוצמה} (ונסמן $|A| = |B|$) אם קיימת פונקציה חח"ע ועל $f : A \to B$.
\end{defbox}

\begin{defbox}
\textbf{הגדרה 1.11 (קבוצה בת מנייה).}
קבוצה $A$ נקראת \textbf{בת מנייה} אם היא סופית או שווה עוצמה ל־$\N$.

קבוצה אינסופית ששווה עוצמה ל־$\N$ נקראת \textbf{בת מנייה אינסופית}.
\end{defbox}

\begin{thmbox}
\textbf{טענה 1.12.}
\begin{enumerate}
    \item $\Z$ בת מנייה.
    \item $\Q$ בת מנייה.
    \item $\R$ \textbf{אינה} בת מנייה (משפט קנטור).
\end{enumerate}
\end{thmbox}

\begin{proofbox}
\textbf{הוכחה (סעיף 1).}
נגדיר $f : \N \to \Z$ על ידי:
\[
f(n) = \begin{cases}
\frac{n}{2} & n \text{ זוגי} \\
-\frac{n+1}{2} & n \text{ אי־זוגי}
\end{cases}
\]
זו התאמה חח"ע ועל: $0 \mapsto 0, 1 \mapsto -1, 2 \mapsto 1, 3 \mapsto -2, 4 \mapsto 2, \ldots$ \hfill $\blacksquare$
\end{proofbox}

\section{תרגילים}

\begin{exercisebox}
\textbf{תרגיל 1.}
הוכיחו כי לכל קבוצות $A, B, C$:
\[
A \cap (B \cup C) = (A \cap B) \cup (A \cap C)
\]

\textbf{פתרון:}
\begin{align*}
x \in A \cap (B \cup C) &\iff x \in A \text{ וגם } x \in B \cup C \\
&\iff x \in A \text{ וגם } (x \in B \text{ או } x \in C) \\
&\iff (x \in A \text{ וגם } x \in B) \text{ או } (x \in A \text{ וגם } x \in C) \\
&\iff x \in (A \cap B) \cup (A \cap C) \quad \blacksquare
\end{align*}
\end{exercisebox}

\begin{exercisebox}
\textbf{תרגיל 2.}
תהי $f : A \to B$ פונקציה. הוכיחו:
\begin{enumerate}
    \item $f$ חח"ע אם ורק אם לכל $C, D \subseteq A$: $f(C \cap D) = f(C) \cap f(D)$
    \item $f$ על אם ורק אם לכל $E \subseteq B$: $f(f^{-1}(E)) = E$
\end{enumerate}

\textbf{פתרון (סעיף 1, כיוון אחד):}
נניח $f$ חח"ע. יהיו $C, D \subseteq A$.

$(\subseteq)$ תמיד מתקיים $f(C \cap D) \subseteq f(C) \cap f(D)$.

$(\supseteq)$ יהי $y \in f(C) \cap f(D)$. אז קיימים $c \in C$, $d \in D$ כך ש־$f(c) = y = f(d)$.

כיוון ש־$f$ חח"ע, $c = d$. לכן $c \in C \cap D$, ומכאן $y = f(c) \in f(C \cap D)$. $\blacksquare$
\end{exercisebox}

\begin{exercisebox}
\textbf{תרגיל 3.}
הוכיחו כי $\Q$ בת מנייה.

\textbf{פתרון:}
נגדיר $f : \Z \times \N_+ \to \Q$ על ידי $f(p, q) = \frac{p}{q}$.

$f$ היא על (כל רציונלי הוא מנה של שלם וטבעי חיובי).

כיוון ש־$\Z$ בת מנייה ו־$\N_+$ בת מנייה, גם $\Z \times \N_+$ בת מנייה (מכפלה של בנות מנייה היא בת מנייה).

תמונה של קבוצה בת מנייה היא בת מנייה, לכן $\Q = \im(f)$ בת מנייה. $\blacksquare$
\end{exercisebox}
  % תורת הקבוצות
% פרק 2: תכונות המספרים הממשיים
\chapter{תכונות המספרים הממשיים}

\section{ערך מוחלט}

\begin{defbox}
\textbf{הגדרה 2.1 (ערך מוחלט).}
לכל $x \in \R$ נגדיר את \textbf{הערך המוחלט} של $x$:
\[
|x| = \begin{cases}
x & x \ge 0 \\
-x & x < 0
\end{cases}
\]
\end{defbox}

\begin{thmbox}
\textbf{טענה 2.2 (תכונות ערך מוחלט).}
לכל $x, y \in \R$:
\begin{enumerate}
    \item $|x| \ge 0$, ו־$|x| = 0 \iff x = 0$
    \item $|xy| = |x| \cdot |y|$
    \item $\left|\frac{x}{y}\right| = \frac{|x|}{|y|}$ (עבור $y \neq 0$)
    \item $-|x| \le x \le |x|$
    \item \textbf{אי־שוויון המשולש:} $|x + y| \le |x| + |y|$
    \item \textbf{אי־שוויון המשולש ההפוך:} $\big||x| - |y|\big| \le |x - y|$
\end{enumerate}
\end{thmbox}

\begin{proofbox}
\textbf{הוכחת אי־שוויון המשולש.}
מתקיים $-|x| \le x \le |x|$ וגם $-|y| \le y \le |y|$.

בחיבור: $-(|x| + |y|) \le x + y \le |x| + |y|$.

לכן $|x + y| \le |x| + |y|$. \hfill $\blacksquare$
\end{proofbox}

\begin{exbox}
\textbf{דוגמה.} הוכיחו כי לכל $x, y, z \in \R$:
\[
|x - z| \le |x - y| + |y - z|
\]

\textbf{פתרון:} נציב $a = x - y$, $b = y - z$ באי־שוויון המשולש:
\[
|a + b| \le |a| + |b| \implies |(x-y) + (y-z)| \le |x-y| + |y-z| \implies |x-z| \le |x-y| + |y-z| \quad \blacksquare
\]
\end{exbox}

\section{חסמים}

\begin{defbox}
\textbf{הגדרה 2.3 (חסם).}
תהי $A \subseteq \R$ קבוצה לא ריקה.
\begin{itemize}
    \item $M \in \R$ נקרא \textbf{חסם מלעיל} של $A$ אם לכל $a \in A$ מתקיים $a \le M$.
    \item $m \in \R$ נקרא \textbf{חסם מלרע} של $A$ אם לכל $a \in A$ מתקיים $a \ge m$.
\end{itemize}
קבוצה נקראת \textbf{חסומה מלעיל/מלרע} אם יש לה חסם מלעיל/מלרע. קבוצה \textbf{חסומה} אם היא חסומה גם מלעיל וגם מלרע.
\end{defbox}

\begin{defbox}
\textbf{הגדרה 2.4 (מקסימום ומינימום).}
תהי $A \subseteq \R$ קבוצה לא ריקה.
\begin{itemize}
    \item $M \in A$ נקרא \textbf{מקסימום} של $A$ (ונסמן $M = \max A$) אם $M$ חסם מלעיל של $A$.
    \item $m \in A$ נקרא \textbf{מינימום} של $A$ (ונסמן $m = \min A$) אם $m$ חסם מלרע של $A$.
\end{itemize}
\end{defbox}

\begin{notebox}
\textbf{הערה.}
לא לכל קבוצה יש מקסימום או מינימום!

\textbf{דוגמה:} לקבוצה $(0, 1)$ אין מקסימום ואין מינימום.
\end{notebox}

\section{סופרימום ואינפימום}

\begin{defbox}
\textbf{הגדרה 2.5 (סופרימום).}
תהי $A \subseteq \R$ קבוצה לא ריקה וחסומה מלעיל.

$s \in \R$ נקרא \textbf{סופרימום (חסם עליון מינימלי)} של $A$ ונסמן $s = \sup A$ אם:
\begin{enumerate}
    \item $s$ חסם מלעיל של $A$ (לכל $a \in A$: $a \le s$)
    \item $s$ הוא החסם מלעיל \textbf{הקטן ביותר}: לכל $\eps > 0$ קיים $a \in A$ כך ש־$a > s - \eps$
\end{enumerate}
\end{defbox}

\begin{defbox}
\textbf{הגדרה 2.6 (אינפימום).}
תהי $A \subseteq \R$ קבוצה לא ריקה וחסומה מלרע.

$t \in \R$ נקרא \textbf{אינפימום (חסם תחתון מקסימלי)} של $A$ ונסמן $t = \inf A$ אם:
\begin{enumerate}
    \item $t$ חסם מלרע של $A$ (לכל $a \in A$: $a \ge t$)
    \item $t$ הוא החסם מלרע \textbf{הגדול ביותר}: לכל $\eps > 0$ קיים $a \in A$ כך ש־$a < t + \eps$
\end{enumerate}
\end{defbox}

\begin{thmbox}
\textbf{טענה 2.7 (אפיון הסופרימום).}
$s = \sup A$ אם ורק אם מתקיימים שני התנאים:
\begin{enumerate}
    \item לכל $a \in A$: $a \le s$
    \item לכל $\eps > 0$ קיים $a \in A$ כך ש־$a > s - \eps$
\end{enumerate}
\end{thmbox}

\begin{exbox}
\textbf{דוגמה 1.} מצאו $\sup A$ ו־$\inf A$ עבור $A = (0, 1]$.

\textbf{פתרון:}
\begin{itemize}
    \item $\sup A = 1$ (ו־$\max A = 1$ כי $1 \in A$)
    \item $\inf A = 0$ (אבל $\min A$ לא קיים כי $0 \notin A$)
\end{itemize}

\textbf{הוכחה ש־$\inf A = 0$:}
\begin{enumerate}
    \item $0$ חסם מלרע: לכל $x \in (0,1]$ מתקיים $x > 0$.
    \item $0$ הוא החסם הגדול ביותר: לכל $\eps > 0$, נבחר $a = \min(\frac{\eps}{2}, \frac{1}{2}) \in A$. אז $a < 0 + \eps$. $\blacksquare$
\end{enumerate}
\end{exbox}

\begin{exbox}
\textbf{דוגמה 2.} מצאו $\sup A$ עבור $A = \left\{\frac{n}{n+1} : n \in \N\right\} = \left\{0, \frac{1}{2}, \frac{2}{3}, \frac{3}{4}, \ldots\right\}$.

\textbf{פתרון:}
נטען $\sup A = 1$.

\begin{enumerate}
    \item $1$ חסם מלעיל: $\frac{n}{n+1} < 1$ לכל $n \in \N$.
    \item לכל $\eps > 0$, נבחר $n$ כך ש־$\frac{1}{n+1} < \eps$ (כלומר $n > \frac{1}{\eps} - 1$). אז:
    \[
    \frac{n}{n+1} = 1 - \frac{1}{n+1} > 1 - \eps \quad \blacksquare
    \]
\end{enumerate}
\end{exbox}

\section{אקסיומת השלמות}

\begin{thmbox}
\textbf{אקסיומה 2.8 (אקסיומת השלמות).}
לכל קבוצה $A \subseteq \R$ לא ריקה וחסומה מלעיל \textbf{קיים סופרימום}.

באופן שקול: לכל קבוצה לא ריקה וחסומה מלרע קיים אינפימום.
\end{thmbox}

\begin{notebox}
\textbf{הערה חשובה.}
אקסיומת השלמות היא מה שמבדיל את $\R$ מ־$\Q$!

\textbf{דוגמה:} הקבוצה $A = \{x \in \Q : x^2 < 2\}$ חסומה מלעיל ב־$\Q$ (למשל על ידי $2$), אבל אין לה סופרימום ב־$\Q$.

(הסופרימום היה צריך להיות $\sqrt{2}$, אבל $\sqrt{2} \notin \Q$.)
\end{notebox}

\begin{thmbox}
\textbf{משפט 2.9 (תכונת ארכימדס).}
לכל $x, y \in \R$ עם $x > 0$, קיים $n \in \N$ כך ש־$nx > y$.

\textbf{באופן שקול:} לכל $\eps > 0$ קיים $n \in \N$ כך ש־$\frac{1}{n} < \eps$.
\end{thmbox}

\begin{proofbox}
\textbf{הוכחה.}
נניח בשלילה שלכל $n \in \N$: $nx \le y$.

אז הקבוצה $A = \{nx : n \in \N\}$ חסומה מלעיל (על ידי $y$).

לפי אקסיומת השלמות, קיים $s = \sup A$.

כיוון ש־$x > 0$, מתקיים $s - x < s$, ולכן $s - x$ אינו חסם מלעיל של $A$.

לכן קיים $n \in \N$ כך ש־$nx > s - x$, כלומר $(n+1)x > s$.

אבל $(n+1)x \in A$, וזו סתירה לכך ש־$s$ חסם מלעיל של $A$. $\blacksquare$
\end{proofbox}

\begin{thmbox}
\textbf{משפט 2.10 (צפיפות הרציונליים).}
בין כל שני ממשיים שונים קיים מספר רציונלי.

כלומר: לכל $a, b \in \R$ עם $a < b$, קיים $q \in \Q$ כך ש־$a < q < b$.
\end{thmbox}

\begin{proofbox}
\textbf{הוכחה.}
לפי תכונת ארכימדס, קיים $n \in \N_+$ כך ש־$\frac{1}{n} < b - a$, כלומר $1 < n(b-a)$.

נגדיר $m = \lfloor na \rfloor + 1$ (המספר השלם הקטן ביותר הגדול מ־$na$).

אז $na < m \le na + 1$.

מצד אחד: $\frac{m}{n} > a$.

מצד שני: $\frac{m}{n} \le \frac{na + 1}{n} = a + \frac{1}{n} < a + (b - a) = b$.

לכן $a < \frac{m}{n} < b$ והמספר $q = \frac{m}{n}$ רציונלי. $\blacksquare$
\end{proofbox}

\section{תכונות נוספות}

\begin{thmbox}
\textbf{טענה 2.11 (תכונות sup ו־inf).}
יהיו $A, B \subseteq \R$ קבוצות לא ריקות וחסומות.
\begin{enumerate}
    \item $\sup(A \cup B) = \max\{\sup A, \sup B\}$
    \item $\inf(A \cup B) = \min\{\inf A, \inf B\}$
    \item אם $A \subseteq B$ אז $\sup A \le \sup B$ ו־$\inf A \ge \inf B$
    \item $\sup(A + B) = \sup A + \sup B$ כאשר $A + B = \{a + b : a \in A, b \in B\}$
    \item $\sup(-A) = -\inf A$ כאשר $-A = \{-a : a \in A\}$
\end{enumerate}
\end{thmbox}

\begin{proofbox}
\textbf{הוכחה (סעיף 4).}
נסמן $s = \sup A$, $t = \sup B$.

\textbf{שלב 1:} $s + t$ חסם מלעיל של $A + B$.

לכל $a \in A$, $b \in B$: $a \le s$ ו־$b \le t$, לכן $a + b \le s + t$.

\textbf{שלב 2:} $s + t$ הוא החסם הקטן ביותר.

יהי $\eps > 0$. קיימים $a \in A$, $b \in B$ כך ש־$a > s - \frac{\eps}{2}$ ו־$b > t - \frac{\eps}{2}$.

לכן $a + b > s + t - \eps$, והאיבר $a + b \in A + B$. $\blacksquare$
\end{proofbox}

\section{ערך שלם}

\begin{defbox}
\textbf{הגדרה 2.12 (פונקציית הערך השלם).}
לכל $x \in \R$, \textbf{הערך השלם} (או \textbf{פונקציית הרצפה}) $\lfloor x \rfloor$ הוא המספר השלם הגדול ביותר שקטן או שווה ל־$x$:
\[
\lfloor x \rfloor = \max\{n \in \Z : n \le x\}
\]
\end{defbox}

\begin{thmbox}
\textbf{טענה 2.13 (תכונות ערך שלם).}
לכל $x \in \R$:
\begin{enumerate}
    \item $\lfloor x \rfloor \le x < \lfloor x \rfloor + 1$
    \item $x - 1 < \lfloor x \rfloor \le x$
    \item $\lfloor x \rfloor = x \iff x \in \Z$
\end{enumerate}
\end{thmbox}

\section{תרגילים}

\begin{exercisebox}
\textbf{תרגיל 1.}
הוכיחו כי לכל קבוצה לא ריקה $A \subseteq \R$:
\[
\sup A = -\inf(-A)
\]

\textbf{פתרון:}
נסמן $s = \sup A$ ו־$t = \inf(-A)$.

\textbf{צ"ל:} $s = -t$.

$t$ חסם מלרע של $-A$: לכל $-a \in -A$ (כלומר $a \in A$) מתקיים $-a \ge t$, כלומר $a \le -t$.

לכן $-t$ חסם מלעיל של $A$, ומכאן $s \le -t$.

באופן דומה (או מסימטריה): $-s \le t$, כלומר $-t \le s$.

מכאן $s = -t$. $\blacksquare$
\end{exercisebox}

\begin{exercisebox}
\textbf{תרגיל 2.}
תהי $A \subseteq \R$ לא ריקה וחסומה. הוכיחו כי אם $\sup A \notin A$ אז קיימת סדרה $(a_n)$ ב־$A$ כך ש־$\limn a_n = \sup A$.

\textbf{פתרון:}
נסמן $s = \sup A$. לכל $n \in \N_+$, לפי אפיון הסופרימום, קיים $a_n \in A$ כך ש:
\[
s - \frac{1}{n} < a_n \le s
\]
(האי־שוויון $a_n \le s$ נובע מכך ש־$s$ חסם מלעיל, והאי־שוויון $a_n < s$ נובע מכך ש־$s \notin A$).

לפי כלל הסנדוויץ': $\limn a_n = s$. $\blacksquare$
\end{exercisebox}

\begin{exercisebox}
\textbf{תרגיל 3.}
מצאו $\sup A$ ו־$\inf A$ עבור:
\[
A = \left\{\frac{(-1)^n}{n} + \frac{1}{n^2} : n \in \N_+\right\}
\]

\textbf{פתרון:}
נפרק למקרים:
\begin{itemize}
    \item $n$ זוגי: $a_n = \frac{1}{n} + \frac{1}{n^2} > 0$
    \item $n$ אי־זוגי: $a_n = -\frac{1}{n} + \frac{1}{n^2} = \frac{1-n}{n^2}$
\end{itemize}

עבור $n = 1$: $a_1 = -1 + 1 = 0$.

עבור $n = 2$: $a_2 = \frac{1}{2} + \frac{1}{4} = \frac{3}{4}$.

עבור $n$ אי־זוגי $\ge 3$: $a_n = \frac{1-n}{n^2} < 0$.

\textbf{מקסימום:} $a_2 = \frac{3}{4}$ הוא האיבר הגדול ביותר (בודקים שלכל $n \ge 2$ זוגי: $a_n = \frac{1}{n} + \frac{1}{n^2} \le \frac{1}{2} + \frac{1}{4}$).

לכן $\sup A = \max A = \frac{3}{4}$.

\textbf{אינפימום:} עבור $n$ אי־זוגי גדול, $a_n = \frac{1-n}{n^2} \to 0^-$.

המינימום הוא $a_3 = \frac{1-3}{9} = -\frac{2}{9}$.

לכן $\inf A = \min A = -\frac{2}{9}$. $\blacksquare$
\end{exercisebox}
  % תכונות הממשיים, sup, inf

% ===================================
% חלק II: סדרות וטורים
% ===================================
\part{סדרות וטורים}

% פרק 3: סדרות
\chapter{סדרות}

\section{מושגים בסיסיים}

\begin{defbox}
\textbf{הגדרה 3.1 (סדרה).}
\textbf{סדרה} (של מספרים ממשיים) היא פונקציה $a : \N \to \R$.

במקום $a(n)$ נכתוב $a_n$ ונסמן את הסדרה ב־$(a_n)_{n=0}^{\infty}$ או $(a_n)$.
\end{defbox}

\begin{defbox}
\textbf{הגדרה 3.2 (חסימות).}
סדרה $(a_n)$ נקראת:
\begin{itemize}
    \item \textbf{חסומה מלעיל} אם קיים $M \in \R$ כך שלכל $n$: $a_n \le M$
    \item \textbf{חסומה מלרע} אם קיים $m \in \R$ כך שלכל $n$: $a_n \ge m$
    \item \textbf{חסומה} אם היא חסומה מלעיל ומלרע (שקול: קיים $M > 0$ כך ש־$|a_n| \le M$ לכל $n$)
\end{itemize}
\end{defbox}

\begin{defbox}
\textbf{הגדרה 3.3 (מונוטוניות).}
סדרה $(a_n)$ נקראת:
\begin{itemize}
    \item \textbf{עולה (מונוטונית)}: $a_n \le a_{n+1}$ לכל $n$
    \item \textbf{עולה ממש}: $a_n < a_{n+1}$ לכל $n$
    \item \textbf{יורדת (מונוטונית)}: $a_n \ge a_{n+1}$ לכל $n$
    \item \textbf{יורדת ממש}: $a_n > a_{n+1}$ לכל $n$
\end{itemize}
\end{defbox}

\section{גבול של סדרה}

\begin{defbox}
\textbf{הגדרה 3.4 (גבול של סדרה).}
יהי $L \in \R$. נאמר כי הסדרה $(a_n)$ \textbf{מתכנסת ל־$L$} (או $L$ הוא \textbf{גבול} הסדרה) ונכתוב $\limn a_n = L$ או $a_n \to L$, אם:
\[
\boxed{\forall \eps > 0 \; \exists N \in \N \; \forall n > N : |a_n - L| < \eps}
\]
סדרה שאינה מתכנסת נקראת \textbf{מתבדרת}.
\end{defbox}

\begin{notebox}
\textbf{פירוש אינטואיטיבי.}
$\limn a_n = L$ אם לכל "סביבה" של $L$ (קטע $(L-\eps, L+\eps)$), כמעט כל איברי הסדרה נמצאים בסביבה זו (כלומר, רק מספר סופי של איברים מחוץ לה).
\end{notebox}

\begin{thmbox}
\textbf{טענה 3.5 (יחידות הגבול).}
אם $(a_n)$ מתכנסת, אז הגבול שלה יחיד.
\end{thmbox}

\begin{proofbox}
\textbf{הוכחה.}
נניח $a_n \to L$ וגם $a_n \to L'$ עם $L \neq L'$.

נבחר $\eps = \frac{|L - L'|}{2} > 0$.

קיים $N_1$ כך שלכל $n > N_1$: $|a_n - L| < \eps$.

קיים $N_2$ כך שלכל $n > N_2$: $|a_n - L'| < \eps$.

עבור $n > \max(N_1, N_2)$:
\[
|L - L'| \le |L - a_n| + |a_n - L'| < \eps + \eps = 2\eps = |L - L'|
\]
סתירה! $\blacksquare$
\end{proofbox}

\begin{thmbox}
\textbf{טענה 3.6 (סדרה מתכנסת חסומה).}
כל סדרה מתכנסת היא חסומה.
\end{thmbox}

\begin{proofbox}
\textbf{הוכחה.}
תהי $a_n \to L$. עבור $\eps = 1$, קיים $N$ כך שלכל $n > N$: $|a_n - L| < 1$, כלומר $|a_n| < |L| + 1$.

נגדיר $M = \max\{|a_0|, |a_1|, \ldots, |a_N|, |L| + 1\}$.

אז $|a_n| \le M$ לכל $n \in \N$. $\blacksquare$
\end{proofbox}

\begin{exbox}
\textbf{דוגמה 1.} הוכיחו כי $\limn \frac{1}{n} = 0$.

\textbf{פתרון:}
יהי $\eps > 0$. צ"ל: קיים $N$ כך שלכל $n > N$: $\left|\frac{1}{n} - 0\right| < \eps$.

נבחר $N = \left\lfloor \frac{1}{\eps} \right\rfloor$.

לכל $n > N$: $n > \frac{1}{\eps}$, לכן $\frac{1}{n} < \eps$, כלומר $\left|\frac{1}{n}\right| < \eps$. $\blacksquare$
\end{exbox}

\begin{exbox}
\textbf{דוגמה 2.} הוכיחו כי $\limn \frac{n+1}{n} = 1$.

\textbf{פתרון:}
\[
\left|\frac{n+1}{n} - 1\right| = \left|\frac{1}{n}\right| = \frac{1}{n}
\]
יהי $\eps > 0$. נבחר $N = \left\lfloor \frac{1}{\eps} \right\rfloor$.

לכל $n > N$: $\frac{1}{n} < \eps$. $\blacksquare$
\end{exbox}

\section{אריתמטיקה של גבולות}

\begin{thmbox}
\textbf{משפט 3.7 (אריתמטיקה של גבולות).}
יהיו $(a_n)$, $(b_n)$ סדרות מתכנסות עם $a_n \to L$ ו־$b_n \to M$. אז:
\begin{enumerate}
    \item $\limn (a_n + b_n) = L + M$
    \item $\limn (a_n - b_n) = L - M$
    \item $\limn (a_n \cdot b_n) = L \cdot M$
    \item $\limn (c \cdot a_n) = c \cdot L$ לכל $c \in \R$
    \item אם $M \neq 0$ אז $\limn \frac{a_n}{b_n} = \frac{L}{M}$
\end{enumerate}
\end{thmbox}

\begin{proofbox}
\textbf{הוכחה (מכפלה).}
\[
|a_n b_n - LM| = |a_n b_n - a_n M + a_n M - LM| \le |a_n||b_n - M| + |M||a_n - L|
\]

כיוון ש־$(a_n)$ מתכנסת, היא חסומה: קיים $K$ כך ש־$|a_n| \le K$ לכל $n$.

יהי $\eps > 0$. קיים $N_1$ כך שלכל $n > N_1$: $|a_n - L| < \frac{\eps}{2(|M|+1)}$.

קיים $N_2$ כך שלכל $n > N_2$: $|b_n - M| < \frac{\eps}{2K}$.

לכל $n > \max(N_1, N_2)$:
\[
|a_n b_n - LM| \le K \cdot \frac{\eps}{2K} + |M| \cdot \frac{\eps}{2(|M|+1)} < \frac{\eps}{2} + \frac{\eps}{2} = \eps \quad \blacksquare
\]
\end{proofbox}

\begin{thmbox}
\textbf{משפט 3.8 (כלל הסנדוויץ').}
יהיו $(a_n)$, $(b_n)$, $(c_n)$ סדרות. אם:
\begin{enumerate}
    \item $a_n \le b_n \le c_n$ לכל $n$ גדול מספיק
    \item $\limn a_n = \limn c_n = L$
\end{enumerate}
אז $\limn b_n = L$.
\end{thmbox}

\begin{proofbox}
\textbf{הוכחה.}
יהי $\eps > 0$. קיים $N_1$ כך שלכל $n > N_1$: $|a_n - L| < \eps$, כלומר $L - \eps < a_n$.

קיים $N_2$ כך שלכל $n > N_2$: $|c_n - L| < \eps$, כלומר $c_n < L + \eps$.

קיים $N_3$ כך שלכל $n > N_3$: $a_n \le b_n \le c_n$.

לכל $n > \max(N_1, N_2, N_3)$:
\[
L - \eps < a_n \le b_n \le c_n < L + \eps
\]
לכן $|b_n - L| < \eps$. $\blacksquare$
\end{proofbox}

\begin{exbox}
\textbf{דוגמה.} חשבו $\limn \frac{\sin n}{n}$.

\textbf{פתרון:}
$-1 \le \sin n \le 1$, לכן $-\frac{1}{n} \le \frac{\sin n}{n} \le \frac{1}{n}$.

כיוון ש־$\limn \frac{1}{n} = \limn \left(-\frac{1}{n}\right) = 0$, לפי כלל הסנדוויץ': $\limn \frac{\sin n}{n} = 0$. $\blacksquare$
\end{exbox}

\section{משפט המונוטוניות}

\begin{thmbox}
\textbf{משפט 3.9 (התכנסות סדרה מונוטונית חסומה).}
\begin{enumerate}
    \item סדרה \textbf{עולה וחסומה מלעיל} מתכנסת, וגבולה הוא $\sup\{a_n : n \in \N\}$.
    \item סדרה \textbf{יורדת וחסומה מלרע} מתכנסת, וגבולה הוא $\inf\{a_n : n \in \N\}$.
\end{enumerate}
\end{thmbox}

\begin{proofbox}
\textbf{הוכחה (סעיף 1).}
תהי $(a_n)$ עולה וחסומה מלעיל. נגדיר $L = \sup\{a_n : n \in \N\}$ (קיים לפי אקסיומת השלמות).

יהי $\eps > 0$. לפי אפיון הסופרימום, קיים $N \in \N$ כך ש־$a_N > L - \eps$.

כיוון שהסדרה עולה, לכל $n > N$: $a_n \ge a_N > L - \eps$.

וכיוון ש־$L$ חסם מלעיל: $a_n \le L < L + \eps$.

לכן $|a_n - L| < \eps$ לכל $n > N$. $\blacksquare$
\end{proofbox}

\section{גבולות שימושיים}

\begin{thmbox}
\textbf{טענה 3.10 (גבולות חשובים).}
\begin{enumerate}
    \item $\limn \frac{1}{n^k} = 0$ לכל $k > 0$
    \item $\limn \sqrt[n]{n} = 1$
    \item $\limn \sqrt[n]{a} = 1$ לכל $a > 0$
    \item $\limn q^n = 0$ לכל $|q| < 1$
    \item $\limn \frac{a^n}{n!} = 0$ לכל $a \in \R$
    \item $\limn \frac{n^k}{a^n} = 0$ לכל $k \in \N$ ו־$a > 1$
\end{enumerate}
\end{thmbox}

\begin{proofbox}
\textbf{הוכחה ($\limn \sqrt[n]{n} = 1$).}
נכתוב $\sqrt[n]{n} = 1 + h_n$ כאשר $h_n \ge 0$ (כי $\sqrt[n]{n} \ge 1$ לכל $n \ge 1$).

אז $n = (1 + h_n)^n \ge 1 + \binom{n}{2}h_n^2 = 1 + \frac{n(n-1)}{2}h_n^2$ (מאי־שוויון ברנולי המורחב).

לכן $h_n^2 \le \frac{2(n-1)}{n(n-1)} = \frac{2}{n}$, ומכאן $0 \le h_n \le \sqrt{\frac{2}{n}} \to 0$.

לפי כלל הסנדוויץ': $h_n \to 0$, לכן $\sqrt[n]{n} = 1 + h_n \to 1$. $\blacksquare$
\end{proofbox}

\section{המספר $e$}

\begin{thmbox}
\textbf{משפט 3.11.}
הסדרה $a_n = \left(1 + \frac{1}{n}\right)^n$ מתכנסת.

גבולה מוגדר כ־\textbf{המספר $e$}:
\[
\boxed{e = \limn \left(1 + \frac{1}{n}\right)^n \approx 2.71828...}
\]
\end{thmbox}

\begin{proofbox}
\textbf{הוכחה (רעיון).}
מראים ש־$(a_n)$ עולה וחסומה מלעיל (על ידי $3$).

\textbf{עולה:} משתמשים באי־שוויון בין ממוצע חשבוני לגיאומטרי.

\textbf{חסומה:} $\left(1 + \frac{1}{n}\right)^n < 3$ לכל $n$. $\blacksquare$
\end{proofbox}

\begin{thmbox}
\textbf{טענה 3.12.}
מתקיים גם:
\[
e = \limn \left(1 + \frac{1}{n}\right)^n = \limn \left(1 - \frac{1}{n}\right)^{-n}
\]
ובאופן כללי, לכל סדרה $a_n \to \infty$:
\[
\limn \left(1 + \frac{1}{a_n}\right)^{a_n} = e
\]
\end{thmbox}

\section{סדרות קושי}

\begin{defbox}
\textbf{הגדרה 3.13 (סדרת קושי).}
סדרה $(a_n)$ נקראת \textbf{סדרת קושי} אם:
\[
\forall \eps > 0 \; \exists N \in \N \; \forall m, n > N : |a_m - a_n| < \eps
\]
\end{defbox}

\begin{thmbox}
\textbf{משפט 3.14 (קריטריון קושי להתכנסות).}
סדרה מתכנסת אם ורק אם היא סדרת קושי.
\end{thmbox}

\begin{proofbox}
\textbf{הוכחה ($\Rightarrow$).}
תהי $a_n \to L$. יהי $\eps > 0$.

קיים $N$ כך שלכל $n > N$: $|a_n - L| < \frac{\eps}{2}$.

לכל $m, n > N$:
\[
|a_m - a_n| \le |a_m - L| + |L - a_n| < \frac{\eps}{2} + \frac{\eps}{2} = \eps \quad \blacksquare
\]
\end{proofbox}

\section{גבולות חלקיים, $\limsup$ ו־$\liminf$}

\begin{defbox}
\textbf{הגדרה 3.15 (תת־סדרה).}
\textbf{תת־סדרה} של $(a_n)$ היא סדרה מהצורה $(a_{n_k})_{k=0}^{\infty}$ כאשר $n_0 < n_1 < n_2 < \cdots$ סדרה עולה ממש של אינדקסים.
\end{defbox}

\begin{defbox}
\textbf{הגדרה 3.16 (גבול חלקי).}
$L \in \R$ נקרא \textbf{גבול חלקי} של $(a_n)$ אם קיימת תת־סדרה $(a_{n_k})$ כך ש־$a_{n_k} \to L$.
\end{defbox}

\begin{thmbox}
\textbf{משפט 3.17 (בולצאנו־ויירשטראס).}
לכל סדרה חסומה יש תת־סדרה מתכנסת.
\end{thmbox}

\begin{defbox}
\textbf{הגדרה 3.18 ($\limsup$ ו־$\liminf$).}
תהי $(a_n)$ סדרה חסומה.
\begin{itemize}
    \item $\limsup_{n \to \infty} a_n$ הוא \textbf{הגבול החלקי הגדול ביותר} של $(a_n)$
    \item $\liminf_{n \to \infty} a_n$ הוא \textbf{הגבול החלקי הקטן ביותר} של $(a_n)$
\end{itemize}
\end{defbox}

\begin{thmbox}
\textbf{טענה 3.19 (אפיון $\limsup$).}
$L = \limsup a_n$ אם ורק אם:
\begin{enumerate}
    \item לכל $\eps > 0$ קיימים אינסוף אינדקסים $n$ כך ש־$a_n > L - \eps$
    \item לכל $\eps > 0$ קיימים רק מספר סופי של אינדקסים $n$ כך ש־$a_n > L + \eps$
\end{enumerate}
\end{thmbox}

\begin{thmbox}
\textbf{טענה 3.20.}
סדרה חסומה $(a_n)$ מתכנסת אם ורק אם $\limsup a_n = \liminf a_n$.

במקרה זה $\limn a_n = \limsup a_n = \liminf a_n$.
\end{thmbox}

\section{תרגילים}

\begin{exercisebox}
\textbf{תרגיל 1.}
חשבו $\limn \frac{2n^2 + 3n - 1}{5n^2 - n + 2}$.

\textbf{פתרון:}
\[
\frac{2n^2 + 3n - 1}{5n^2 - n + 2} = \frac{n^2(2 + \frac{3}{n} - \frac{1}{n^2})}{n^2(5 - \frac{1}{n} + \frac{2}{n^2})} = \frac{2 + \frac{3}{n} - \frac{1}{n^2}}{5 - \frac{1}{n} + \frac{2}{n^2}} \to \frac{2 + 0 - 0}{5 - 0 + 0} = \frac{2}{5} \quad \blacksquare
\]
\end{exercisebox}

\begin{exercisebox}
\textbf{תרגיל 2.}
הוכיחו כי הסדרה $a_n = \frac{n!}{n^n}$ מתכנסת ומצאו את גבולה.

\textbf{פתרון:}
נבדוק מונוטוניות:
\[
\frac{a_{n+1}}{a_n} = \frac{(n+1)!}{(n+1)^{n+1}} \cdot \frac{n^n}{n!} = \frac{n+1}{(n+1)^{n+1}} \cdot n^n = \frac{n^n}{(n+1)^n} = \left(\frac{n}{n+1}\right)^n = \left(1 - \frac{1}{n+1}\right)^n
\]

עבור $n$ גדול מספיק: $\left(1 - \frac{1}{n+1}\right)^n < 1$ (כי $\left(1 - \frac{1}{n+1}\right)^{n+1} \to \frac{1}{e} < 1$).

לכן הסדרה יורדת (מ־$n$ מסוים). היא גם חסומה מלרע (על ידי $0$).

לכן מתכנסת. נסמן $L = \limn a_n$.

מהיחס: $L = L \cdot \frac{1}{e}$, כלומר $L(1 - \frac{1}{e}) = 0$, לכן $L = 0$. $\blacksquare$
\end{exercisebox}

\begin{exercisebox}
\textbf{תרגיל 3.}
חשבו $\limn \left(1 + \frac{2}{n}\right)^n$.

\textbf{פתרון:}
\[
\left(1 + \frac{2}{n}\right)^n = \left[\left(1 + \frac{1}{n/2}\right)^{n/2}\right]^2 \to e^2 \quad \blacksquare
\]
\end{exercisebox}

\begin{exercisebox}
\textbf{תרגיל 4.}
תהי $(a_n)$ סדרה המקיימת $a_{n+1} = \frac{1}{2}(a_n + \frac{2}{a_n})$ עם $a_1 = 2$.

הוכיחו כי הסדרה מתכנסת ומצאו את גבולה.

\textbf{פתרון:}
\textbf{שלב 1:} $a_n > 0$ לכל $n$ (באינדוקציה).

\textbf{שלב 2:} $a_n \ge \sqrt{2}$ לכל $n \ge 2$.

$a_{n+1} = \frac{a_n^2 + 2}{2a_n} \ge \sqrt{2}$ (מאי־שוויון AM-GM: $\frac{a_n^2 + 2}{2} \ge \sqrt{a_n^2 \cdot 2} = a_n\sqrt{2}$).

\textbf{שלב 3:} הסדרה יורדת (מ־$n = 2$):
\[
a_{n+1} - a_n = \frac{a_n^2 + 2 - 2a_n^2}{2a_n} = \frac{2 - a_n^2}{2a_n} \le 0 \text{ (כי } a_n \ge \sqrt{2}\text{)}
\]

\textbf{שלב 4:} הסדרה יורדת וחסומה מלרע, לכן מתכנסת.

נסמן $L = \limn a_n$. מהנוסחה: $L = \frac{1}{2}(L + \frac{2}{L})$, לכן $2L = L + \frac{2}{L}$, $L^2 = 2$, $L = \sqrt{2}$. $\blacksquare$
\end{exercisebox}
  % סדרות
% פרק 4: טורים
\chapter{טורים}

\section{הגדרות ותכונות בסיסיות}

\begin{defbox}
\textbf{הגדרה 4.1 (טור).}
יהי $(a_n)_{n=1}^{\infty}$ סדרה. \textbf{הטור} $\sum_{n=1}^{\infty} a_n$ הוא הביטוי הפורמלי $a_1 + a_2 + a_3 + \cdots$

\textbf{הסכום החלקי ה־$N$־י} הוא $S_N = \sum_{n=1}^{N} a_n = a_1 + a_2 + \cdots + a_N$.
\end{defbox}

\begin{defbox}
\textbf{הגדרה 4.2 (התכנסות טור).}
הטור $\sum_{n=1}^{\infty} a_n$ \textbf{מתכנס} אם סדרת הסכומים החלקיים $(S_N)$ מתכנסת.

במקרה זה, \textbf{סכום הטור} הוא $\sum_{n=1}^{\infty} a_n = \lim_{N \to \infty} S_N$.

טור שאינו מתכנס נקרא \textbf{מתבדר}.
\end{defbox}

\begin{exbox}
\textbf{דוגמה 1 (הטור הגיאומטרי).}
$\sum_{n=0}^{\infty} q^n = 1 + q + q^2 + \cdots$

\textbf{פתרון:}
$S_N = \frac{1 - q^{N+1}}{1 - q}$ (עבור $q \neq 1$).

הטור מתכנס אם ורק אם $|q| < 1$, ואז:
\[
\boxed{\sum_{n=0}^{\infty} q^n = \frac{1}{1-q} \quad (|q| < 1)}
\]
\end{exbox}

\begin{exbox}
\textbf{דוגמה 2 (הטור ההרמוני).}
$\sum_{n=1}^{\infty} \frac{1}{n} = 1 + \frac{1}{2} + \frac{1}{3} + \cdots$ \textbf{מתבדר}.

\textbf{הוכחה:}
\[
S_{2^k} = 1 + \frac{1}{2} + \left(\frac{1}{3} + \frac{1}{4}\right) + \left(\frac{1}{5} + \cdots + \frac{1}{8}\right) + \cdots > 1 + \frac{1}{2} + \frac{2}{4} + \frac{4}{8} + \cdots = 1 + \frac{k}{2} \to \infty
\]
\end{exbox}

\begin{thmbox}
\textbf{טענה 4.3 (תנאי הכרחי להתכנסות).}
אם הטור $\sum a_n$ מתכנס אז $\limn a_n = 0$.
\end{thmbox}

\begin{proofbox}
\textbf{הוכחה.}
אם $S = \sum a_n$ מתכנס, אז $S_n \to S$ ו־$S_{n-1} \to S$.

לכן $a_n = S_n - S_{n-1} \to S - S = 0$. $\blacksquare$
\end{proofbox}

\begin{notebox}
\textbf{אזהרה!}
התנאי $a_n \to 0$ הוא \textbf{הכרחי אך לא מספיק}.

דוגמה נגדית: $a_n = \frac{1}{n} \to 0$ אבל $\sum \frac{1}{n}$ מתבדר!
\end{notebox}

\begin{thmbox}
\textbf{טענה 4.4 (לינאריות).}
אם $\sum a_n$ ו־$\sum b_n$ מתכנסים, אז:
\begin{enumerate}
    \item $\sum (a_n + b_n) = \sum a_n + \sum b_n$
    \item $\sum (c \cdot a_n) = c \cdot \sum a_n$ לכל $c \in \R$
\end{enumerate}
\end{thmbox}

\section{מבחני התכנסות לטורים אי־שליליים}

\begin{thmbox}
\textbf{טענה 4.5 (התכנסות טור אי־שלילי).}
טור $\sum a_n$ עם $a_n \ge 0$ מתכנס אם ורק אם סדרת הסכומים החלקיים חסומה.
\end{thmbox}

\begin{proofbox}
\textbf{הוכחה.}
$(S_N)$ עולה (כי $S_{N+1} = S_N + a_{N+1} \ge S_N$).

סדרה עולה מתכנסת אם ורק אם היא חסומה מלעיל. $\blacksquare$
\end{proofbox}

\subsection{מבחן ההשוואה}

\begin{thmbox}
\textbf{משפט 4.6 (מבחן ההשוואה).}
יהיו $(a_n)$, $(b_n)$ סדרות עם $0 \le a_n \le b_n$ לכל $n$.
\begin{enumerate}
    \item אם $\sum b_n$ מתכנס אז $\sum a_n$ מתכנס.
    \item אם $\sum a_n$ מתבדר אז $\sum b_n$ מתבדר.
\end{enumerate}
\end{thmbox}

\begin{proofbox}
\textbf{הוכחה (סעיף 1).}
$S_N^{(a)} = \sum_{n=1}^N a_n \le \sum_{n=1}^N b_n = S_N^{(b)} \le \sum_{n=1}^{\infty} b_n$

לכן $(S_N^{(a)})$ חסומה מלעיל, ולכן $\sum a_n$ מתכנס. $\blacksquare$
\end{proofbox}

\begin{exbox}
\textbf{דוגמה.} הוכיחו כי $\sum_{n=1}^{\infty} \frac{1}{n^2}$ מתכנס.

\textbf{פתרון:}
לכל $n \ge 2$: $\frac{1}{n^2} < \frac{1}{n(n-1)} = \frac{1}{n-1} - \frac{1}{n}$ (טור טלסקופי).

$\sum_{n=2}^N \frac{1}{n(n-1)} = 1 - \frac{1}{N} \to 1$.

לפי מבחן ההשוואה, $\sum \frac{1}{n^2}$ מתכנס. $\blacksquare$
\end{exbox}

\subsection{מבחן ההשוואה הגבולי}

\begin{thmbox}
\textbf{משפט 4.7 (מבחן ההשוואה הגבולי).}
יהיו $(a_n)$, $(b_n)$ סדרות חיוביות. אם $\limn \frac{a_n}{b_n} = L$ כאשר $0 < L < \infty$, אז:
\[
\sum a_n \text{ מתכנס } \iff \sum b_n \text{ מתכנס}
\]
\end{thmbox}

\begin{proofbox}
\textbf{הוכחה.}
קיים $N$ כך שלכל $n > N$: $\frac{L}{2} < \frac{a_n}{b_n} < \frac{3L}{2}$.

לכן $\frac{L}{2} b_n < a_n < \frac{3L}{2} b_n$.

לפי מבחן ההשוואה (ולינאריות), $\sum a_n$ ו־$\sum b_n$ מתכנסים או מתבדרים יחד. $\blacksquare$
\end{proofbox}

\begin{exbox}
\textbf{דוגמה.} בדקו התכנסות $\sum_{n=1}^{\infty} \frac{n+1}{n^3 + 2n}$.

\textbf{פתרון:}
נשווה ל־$b_n = \frac{1}{n^2}$ (הטור $\sum \frac{1}{n^2}$ מתכנס).

\[
\frac{a_n}{b_n} = \frac{(n+1) \cdot n^2}{n^3 + 2n} = \frac{n^3 + n^2}{n^3 + 2n} = \frac{1 + \frac{1}{n}}{1 + \frac{2}{n^2}} \to 1 \in (0, \infty)
\]

לפי מבחן ההשוואה הגבולי, הטור \textbf{מתכנס}. $\blacksquare$
\end{exbox}

\subsection{מבחן המנה (דלמבר)}

\begin{thmbox}
\textbf{משפט 4.8 (מבחן המנה / דלמבר).}
תהי $(a_n)$ סדרה חיובית. נסמן $L = \limn \frac{a_{n+1}}{a_n}$ (אם קיים).
\begin{enumerate}
    \item אם $L < 1$ אז $\sum a_n$ \textbf{מתכנס}.
    \item אם $L > 1$ אז $\sum a_n$ \textbf{מתבדר}.
    \item אם $L = 1$ אז \textbf{המבחן לא מכריע}.
\end{enumerate}
\end{thmbox}

\begin{proofbox}
\textbf{הוכחה (מקרה $L < 1$).}
נבחר $q$ כך ש־$L < q < 1$.

קיים $N$ כך שלכל $n > N$: $\frac{a_{n+1}}{a_n} < q$.

לכן $a_{N+k} < q^k \cdot a_N$ לכל $k \ge 1$.

הטור $\sum_{k=1}^{\infty} q^k \cdot a_N$ מתכנס (טור גיאומטרי עם $|q| < 1$).

לפי מבחן ההשוואה, $\sum a_n$ מתכנס. $\blacksquare$
\end{proofbox}

\begin{exbox}
\textbf{דוגמה 1.} בדקו התכנסות $\sum_{n=1}^{\infty} \frac{n!}{n^n}$.

\textbf{פתרון:}
\[
\frac{a_{n+1}}{a_n} = \frac{(n+1)!}{(n+1)^{n+1}} \cdot \frac{n^n}{n!} = \frac{n^n}{(n+1)^n} = \left(\frac{n}{n+1}\right)^n = \left(1 - \frac{1}{n+1}\right)^n \to \frac{1}{e} < 1
\]

לפי מבחן המנה, הטור \textbf{מתכנס}. $\blacksquare$
\end{exbox}

\begin{exbox}
\textbf{דוגמה 2.} בדקו התכנסות $\sum_{n=1}^{\infty} \frac{2^n}{n!}$.

\textbf{פתרון:}
\[
\frac{a_{n+1}}{a_n} = \frac{2^{n+1}}{(n+1)!} \cdot \frac{n!}{2^n} = \frac{2}{n+1} \to 0 < 1
\]

לפי מבחן המנה, הטור \textbf{מתכנס}. $\blacksquare$
\end{exbox}

\subsection{מבחן השורש (קושי)}

\begin{thmbox}
\textbf{משפט 4.9 (מבחן השורש / קושי).}
תהי $(a_n)$ סדרה אי־שלילית. נסמן $L = \limsup_{n \to \infty} \sqrt[n]{a_n}$.
\begin{enumerate}
    \item אם $L < 1$ אז $\sum a_n$ \textbf{מתכנס}.
    \item אם $L > 1$ אז $\sum a_n$ \textbf{מתבדר}.
    \item אם $L = 1$ אז \textbf{המבחן לא מכריע}.
\end{enumerate}
\end{thmbox}

\begin{exbox}
\textbf{דוגמה.} בדקו התכנסות $\sum_{n=1}^{\infty} \frac{1}{2^n + 3^n}$.

\textbf{פתרון:}
\[
\sqrt[n]{a_n} = \frac{1}{\sqrt[n]{2^n + 3^n}} = \frac{1}{3 \cdot \sqrt[n]{(\frac{2}{3})^n + 1}} \to \frac{1}{3} < 1
\]

לפי מבחן השורש, הטור \textbf{מתכנס}. $\blacksquare$
\end{exbox}

\subsection{מבחן האינטגרל}

\begin{thmbox}
\textbf{משפט 4.10 (מבחן האינטגרל).}
תהי $f : [1, \infty) \to [0, \infty)$ פונקציה יורדת ורציפה.

אז הטור $\sum_{n=1}^{\infty} f(n)$ מתכנס אם ורק אם האינטגרל $\int_1^{\infty} f(x) dx$ מתכנס.
\end{thmbox}

\begin{exbox}
\textbf{דוגמה (טור $p$).}
\[
\sum_{n=1}^{\infty} \frac{1}{n^p} \text{ מתכנס } \iff p > 1
\]

\textbf{הוכחה:}
$f(x) = \frac{1}{x^p}$ יורדת ורציפה.

$\int_1^{\infty} \frac{1}{x^p} dx$ מתכנס אם ורק אם $p > 1$ (ראינו בפרק 11).

לפי מבחן האינטגרל, גם הטור מתכנס אם ורק אם $p > 1$. $\blacksquare$
\end{exbox}

\section{מבחני התכנסות לטורים כלליים}

\subsection{התכנסות בהחלט והתכנסות בתנאי}

\begin{defbox}
\textbf{הגדרה 4.11 (התכנסות בהחלט).}
טור $\sum a_n$ \textbf{מתכנס בהחלט} אם הטור $\sum |a_n|$ מתכנס.
\end{defbox}

\begin{thmbox}
\textbf{משפט 4.12.}
אם טור מתכנס בהחלט אז הוא מתכנס.
\end{thmbox}

\begin{proofbox}
\textbf{הוכחה.}
נשתמש בקריטריון קושי. יהי $\eps > 0$.

כיוון ש־$\sum |a_n|$ מתכנס, קיים $N$ כך שלכל $m > n > N$:
\[
\sum_{k=n+1}^{m} |a_k| < \eps
\]

לכן:
\[
\left|\sum_{k=n+1}^{m} a_k\right| \le \sum_{k=n+1}^{m} |a_k| < \eps
\]

לפי קריטריון קושי, $\sum a_n$ מתכנס. $\blacksquare$
\end{proofbox}

\begin{defbox}
\textbf{הגדרה 4.13 (התכנסות בתנאי).}
טור \textbf{מתכנס בתנאי} אם הוא מתכנס אך לא מתכנס בהחלט.
\end{defbox}

\subsection{מבחן לייבניץ}

\begin{thmbox}
\textbf{משפט 4.14 (מבחן לייבניץ לטורים מתחלפים).}
יהי $(a_n)$ סדרה עם:
\begin{enumerate}
    \item $a_n \ge 0$ לכל $n$
    \item $(a_n)$ יורדת מונוטונית
    \item $\limn a_n = 0$
\end{enumerate}
אז הטור המתחלף $\sum_{n=1}^{\infty} (-1)^{n+1} a_n = a_1 - a_2 + a_3 - a_4 + \cdots$ \textbf{מתכנס}.
\end{thmbox}

\begin{proofbox}
\textbf{הוכחה.}
נבחן את הסכומים החלקיים:
\[
S_{2n} = (a_1 - a_2) + (a_3 - a_4) + \cdots + (a_{2n-1} - a_{2n})
\]
כל סוגריים חיובי (כי $(a_n)$ יורדת), לכן $(S_{2n})$ עולה.

גם:
\[
S_{2n} = a_1 - (a_2 - a_3) - (a_4 - a_5) - \cdots - (a_{2n-2} - a_{2n-1}) - a_{2n} \le a_1
\]
לכן $(S_{2n})$ חסומה מלעיל.

סדרה עולה וחסומה מתכנסת. נסמן $S = \lim S_{2n}$.

גם $S_{2n+1} = S_{2n} + a_{2n+1} \to S + 0 = S$.

לכן $S_n \to S$. $\blacksquare$
\end{proofbox}

\begin{exbox}
\textbf{דוגמה (הטור ההרמוני המתחלף).}
\[
\sum_{n=1}^{\infty} \frac{(-1)^{n+1}}{n} = 1 - \frac{1}{2} + \frac{1}{3} - \frac{1}{4} + \cdots = \ln 2
\]

\textbf{הוכחה שמתכנס:}
$a_n = \frac{1}{n}$ מקיימת: $a_n > 0$, יורדת, $a_n \to 0$.

לפי לייבניץ, הטור מתכנס.

\textbf{הערה:} זהו טור שמתכנס \textbf{בתנאי} (לא בהחלט, כי $\sum \frac{1}{n}$ מתבדר). $\blacksquare$
\end{exbox}

\subsection{מבחן דיריכלה ומבחן אבל}

\begin{thmbox}
\textbf{משפט 4.15 (מבחן דיריכלה).}
יהיו $(a_n)$, $(b_n)$ סדרות כך ש:
\begin{enumerate}
    \item סדרת הסכומים החלקיים $S_n = \sum_{k=1}^n a_k$ \textbf{חסומה}
    \item $(b_n)$ \textbf{מונוטונית ומתכנסת ל־$0$}
\end{enumerate}
אז הטור $\sum a_n b_n$ מתכנס.
\end{thmbox}

\begin{thmbox}
\textbf{משפט 4.16 (מבחן אבל).}
יהיו $(a_n)$, $(b_n)$ סדרות כך ש:
\begin{enumerate}
    \item הטור $\sum a_n$ \textbf{מתכנס}
    \item $(b_n)$ \textbf{מונוטונית וחסומה}
\end{enumerate}
אז הטור $\sum a_n b_n$ מתכנס.
\end{thmbox}

\section{תרגילים}

\begin{exercisebox}
\textbf{תרגיל 1.}
בדקו התכנסות הטורים הבאים:
\begin{enumerate}
    \item $\sum_{n=1}^{\infty} \frac{n^2}{2^n}$
    \item $\sum_{n=2}^{\infty} \frac{1}{n \ln n}$
    \item $\sum_{n=1}^{\infty} \frac{(-1)^n}{\sqrt{n}}$
    \item $\sum_{n=1}^{\infty} \frac{n!}{n^n}$
\end{enumerate}

\textbf{פתרונות:}

\textbf{(א)} מבחן המנה:
\[
\frac{a_{n+1}}{a_n} = \frac{(n+1)^2}{2^{n+1}} \cdot \frac{2^n}{n^2} = \frac{1}{2} \cdot \frac{(n+1)^2}{n^2} = \frac{1}{2} \cdot \left(1 + \frac{1}{n}\right)^2 \to \frac{1}{2} < 1
\]
\textbf{מתכנס}. $\blacksquare$

\textbf{(ב)} מבחן האינטגרל עם $f(x) = \frac{1}{x \ln x}$:
\[
\int_2^{\infty} \frac{dx}{x \ln x} = [\ln(\ln x)]_2^{\infty} = \infty
\]
\textbf{מתבדר}. $\blacksquare$

\textbf{(ג)} מבחן לייבניץ: $a_n = \frac{1}{\sqrt{n}}$ יורדת ומתכנסת ל־$0$.

הטור $\sum \frac{(-1)^n}{\sqrt{n}}$ \textbf{מתכנס} (בתנאי, לא בהחלט). $\blacksquare$

\textbf{(ד)} מבחן המנה (ראינו קודם):
\[
\frac{a_{n+1}}{a_n} \to \frac{1}{e} < 1
\]
\textbf{מתכנס}. $\blacksquare$
\end{exercisebox}

\begin{exercisebox}
\textbf{תרגיל 2.}
הוכיחו כי $\sum_{n=1}^{\infty} \frac{1}{n^2} = \frac{\pi^2}{6}$.

\textbf{פתרון (רמז):}
זוהי תוצאה מפורסמת (בעיית באזל). ההוכחה המלאה משתמשת בפיתוח טיילור של $\frac{\sin x}{x}$ או בניתוח פורייה.

נוכיח רק את ההתכנסות: לכל $n \ge 2$:
\[
\frac{1}{n^2} < \frac{1}{n(n-1)} = \frac{1}{n-1} - \frac{1}{n}
\]

לכן:
\[
\sum_{n=2}^{N} \frac{1}{n^2} < \sum_{n=2}^{N} \left(\frac{1}{n-1} - \frac{1}{n}\right) = 1 - \frac{1}{N} < 1
\]

לכן $\sum_{n=1}^{\infty} \frac{1}{n^2} < 1 + 1 = 2$. הטור מתכנס. $\blacksquare$
\end{exercisebox}

\begin{exercisebox}
\textbf{תרגיל 3.}
קבעו אם הטורים הבאים מתכנסים בהחלט, בתנאי, או מתבדרים:
\begin{enumerate}
    \item $\sum_{n=1}^{\infty} \frac{(-1)^n}{n^2}$
    \item $\sum_{n=1}^{\infty} \frac{\sin n}{n^2}$
    \item $\sum_{n=1}^{\infty} \frac{(-1)^n \cdot n}{n+1}$
\end{enumerate}

\textbf{פתרונות:}

\textbf{(א)} $\sum \frac{1}{n^2}$ מתכנס, לכן $\sum \frac{(-1)^n}{n^2}$ \textbf{מתכנס בהחלט}. $\blacksquare$

\textbf{(ב)} $\left|\frac{\sin n}{n^2}\right| \le \frac{1}{n^2}$, ו־$\sum \frac{1}{n^2}$ מתכנס.

לפי מבחן ההשוואה, $\sum \frac{|\sin n|}{n^2}$ מתכנס, לכן הטור \textbf{מתכנס בהחלט}. $\blacksquare$

\textbf{(ג)} $a_n = \frac{(-1)^n \cdot n}{n+1} \to (-1)^n \cdot 1$, לא מתכנס ל־$0$.

לפי התנאי ההכרחי, הטור \textbf{מתבדר}. $\blacksquare$
\end{exercisebox}
  % טורים

% ===================================
% חלק III: גבולות ורציפות
% ===================================
\part{גבולות ורציפות}

% פרק 5: פונקציות במשתנה אחד - מושגים בסיסיים
\chapter{פונקציות במשתנה אחד}

\section{פונקציות ממשיות}

\begin{defbox}
\textbf{הגדרה 5.1.}
\textbf{פונקציה ממשית} היא פונקציה $f : D \to \R$ כאשר $D \subseteq \R$ נקרא \textbf{תחום ההגדרה}.
\end{defbox}

\begin{defbox}
\textbf{הגדרה 5.2 (סוגי פונקציות).}
פונקציה $f : D \to \R$ נקראת:
\begin{itemize}
    \item \textbf{חסומה מלעיל}: קיים $M$ כך ש־$f(x) \le M$ לכל $x \in D$
    \item \textbf{חסומה מלרע}: קיים $m$ כך ש־$f(x) \ge m$ לכל $x \in D$
    \item \textbf{חסומה}: חסומה מלעיל ומלרע
    \item \textbf{עולה}: $x_1 < x_2 \Rightarrow f(x_1) \le f(x_2)$
    \item \textbf{עולה ממש}: $x_1 < x_2 \Rightarrow f(x_1) < f(x_2)$
    \item \textbf{זוגית}: $f(-x) = f(x)$ לכל $x$ (והתחום סימטרי)
    \item \textbf{אי־זוגית}: $f(-x) = -f(x)$ לכל $x$
    \item \textbf{מחזורית}: קיים $T > 0$ כך ש־$f(x + T) = f(x)$ לכל $x$
\end{itemize}
\end{defbox}

\section{פונקציות אלמנטריות}

\begin{notebox}
\textbf{הפונקציות האלמנטריות:}
\begin{enumerate}
    \item \textbf{פולינומים:} $p(x) = a_n x^n + \cdots + a_1 x + a_0$
    \item \textbf{פונקציות רציונליות:} $\frac{p(x)}{q(x)}$ (מנת פולינומים)
    \item \textbf{פונקציות טריגונומטריות:} $\sin, \cos, \tan, \cot$
    \item \textbf{פונקציות טריגונומטריות הפוכות:} $\arcsin, \arccos, \arctan$
    \item \textbf{פונקציה מעריכית:} $e^x$ או $a^x$
    \item \textbf{פונקציה לוגריתמית:} $\ln x$ או $\log_a x$
    \item \textbf{חזקה כללית:} $x^a = e^{a \ln x}$ (עבור $x > 0$)
\end{enumerate}
\end{notebox}

\section{פונקציות מיוחדות}

\begin{defbox}
\textbf{פונקציית דיריכלה:}
\[
D(x) = \begin{cases}
1 & x \in \Q \\
0 & x \notin \Q
\end{cases}
\]
פונקציה זו אינה רציפה בשום נקודה!
\end{defbox}

\begin{defbox}
\textbf{פונקציית הסימן:}
\[
\text{sgn}(x) = \begin{cases}
1 & x > 0 \\
0 & x = 0 \\
-1 & x < 0
\end{cases}
\]
\end{defbox}

\begin{defbox}
\textbf{פונקציית הערך השלם (רצפה):}
\[
\lfloor x \rfloor = \max\{n \in \Z : n \le x\}
\]
\end{defbox}

\section{תרגילים}

\begin{exercisebox}
\textbf{תרגיל 1.}
הוכיחו כי לכל פונקציה $f : \R \to \R$ קיימות פונקציה זוגית $g$ ופונקציה אי־זוגית $h$ כך ש־$f = g + h$.

\textbf{פתרון:}
נגדיר:
\[
g(x) = \frac{f(x) + f(-x)}{2}, \quad h(x) = \frac{f(x) - f(-x)}{2}
\]

נבדוק:
\begin{itemize}
    \item $g(-x) = \frac{f(-x) + f(x)}{2} = g(x)$ — זוגית
    \item $h(-x) = \frac{f(-x) - f(x)}{2} = -h(x)$ — אי־זוגית
    \item $g(x) + h(x) = \frac{f(x) + f(-x) + f(x) - f(-x)}{2} = f(x)$ $\blacksquare$
\end{itemize}
\end{exercisebox}
  % פונקציות במשתנה אחד
% פרק 6: גבולות של פונקציות
\chapter{גבולות של פונקציות}

\section{הגדרת גבול}

\begin{defbox}
\textbf{הגדרה 6.1 (גבול של פונקציה — הגדרת $\eps$-$\delta$).}
יהי $x_0$ נקודת צבירה של $D$. נאמר ש־$\limx{x_0} f(x) = L$ אם:
\[
\boxed{\forall \eps > 0 \; \exists \delta > 0 \; \forall x \in D : 0 < |x - x_0| < \delta \Rightarrow |f(x) - L| < \eps}
\]
\end{defbox}

\begin{defbox}
\textbf{הגדרה 6.2 (גבול — הגדרת היינה).}
$\limx{x_0} f(x) = L$ אם לכל סדרה $(x_n)$ בתחום עם $x_n \neq x_0$ ו־$x_n \to x_0$ מתקיים $f(x_n) \to L$.
\end{defbox}

\begin{thmbox}
\textbf{משפט 6.3 (שקילות ההגדרות).}
הגדרת $\eps$-$\delta$ והגדרת היינה שקולות.
\end{thmbox}

\begin{exbox}
\textbf{דוגמה 1.} הוכיחו כי $\limx{2} (3x - 1) = 5$.

\textbf{פתרון:}
יהי $\eps > 0$. צ"ל: קיים $\delta > 0$ כך שאם $0 < |x - 2| < \delta$ אז $|(3x-1) - 5| < \eps$.

$|(3x-1) - 5| = |3x - 6| = 3|x - 2|$.

נבחר $\delta = \frac{\eps}{3}$.

אם $|x - 2| < \delta = \frac{\eps}{3}$ אז $|3x - 6| = 3|x-2| < 3 \cdot \frac{\eps}{3} = \eps$. $\blacksquare$
\end{exbox}

\begin{exbox}
\textbf{דוגמה 2.} הוכיחו כי $\limx{0} \frac{\sin x}{x} = 1$.

\textbf{פתרון (רעיון):}
מאי־שוויון גיאומטרי: לכל $0 < |x| < \frac{\pi}{2}$:
\[
\cos x < \frac{\sin x}{x} < 1
\]

כיוון ש־$\cos x \to 1$ כש־$x \to 0$, לפי כלל הסנדוויץ': $\frac{\sin x}{x} \to 1$. $\blacksquare$
\end{exbox}

\section{אריתמטיקה של גבולות}

\begin{thmbox}
\textbf{משפט 6.4 (אריתמטיקה של גבולות).}
אם $\limx{x_0} f(x) = L$ ו־$\limx{x_0} g(x) = M$ אז:
\begin{enumerate}
    \item $\limx{x_0} (f(x) + g(x)) = L + M$
    \item $\limx{x_0} (f(x) \cdot g(x)) = L \cdot M$
    \item $\limx{x_0} \frac{f(x)}{g(x)} = \frac{L}{M}$ (אם $M \neq 0$)
\end{enumerate}
\end{thmbox}

\begin{thmbox}
\textbf{משפט 6.5 (כלל הסנדוויץ').}
אם $f(x) \le g(x) \le h(x)$ בסביבה מנוקבת של $x_0$, ו־$\limx{x_0} f(x) = \limx{x_0} h(x) = L$, אז $\limx{x_0} g(x) = L$.
\end{thmbox}

\section{גבולות חד־צדדיים}

\begin{defbox}
\textbf{הגדרה 6.6 (גבולות חד־צדדיים).}
\begin{itemize}
    \item \textbf{גבול מימין:} $\limx{x_0^+} f(x) = L$ אם $\forall \eps > 0 \; \exists \delta > 0 : x_0 < x < x_0 + \delta \Rightarrow |f(x) - L| < \eps$
    \item \textbf{גבול משמאל:} $\limx{x_0^-} f(x) = L$ אם $\forall \eps > 0 \; \exists \delta > 0 : x_0 - \delta < x < x_0 \Rightarrow |f(x) - L| < \eps$
\end{itemize}
\end{defbox}

\begin{thmbox}
\textbf{טענה 6.7.}
$\limx{x_0} f(x) = L$ אם ורק אם $\limx{x_0^+} f(x) = \limx{x_0^-} f(x) = L$.
\end{thmbox}

\section{גבולות באינסוף}

\begin{defbox}
\textbf{הגדרה 6.8.}
$\limx{\infty} f(x) = L$ אם $\forall \eps > 0 \; \exists M > 0 : x > M \Rightarrow |f(x) - L| < \eps$.
\end{defbox}

\begin{defbox}
\textbf{הגדרה 6.9.}
$\limx{x_0} f(x) = \infty$ אם $\forall K > 0 \; \exists \delta > 0 : 0 < |x - x_0| < \delta \Rightarrow f(x) > K$.
\end{defbox}

\section{גבולות חשובים}

\begin{thmbox}
\textbf{טענה 6.10 (גבולות חשובים).}
\begin{enumerate}
    \item $\limx{0} \frac{\sin x}{x} = 1$
    \item $\limx{0} \frac{1 - \cos x}{x^2} = \frac{1}{2}$
    \item $\limx{0} \frac{e^x - 1}{x} = 1$
    \item $\limx{0} \frac{\ln(1+x)}{x} = 1$
    \item $\limx{0} \frac{(1+x)^a - 1}{x} = a$
    \item $\limx{\infty} \left(1 + \frac{1}{x}\right)^x = e$
    \item $\limx{0} (1 + x)^{1/x} = e$
\end{enumerate}
\end{thmbox}

\section{תרגילים}

\begin{exercisebox}
\textbf{תרגיל 1.}
חשבו $\limx{0} \frac{\tan x - \sin x}{x^3}$.

\textbf{פתרון:}
\[
\frac{\tan x - \sin x}{x^3} = \frac{\sin x(\frac{1}{\cos x} - 1)}{x^3} = \frac{\sin x}{x} \cdot \frac{1 - \cos x}{x^2 \cos x}
\]

כש־$x \to 0$:
\[
\frac{\sin x}{x} \to 1, \quad \frac{1 - \cos x}{x^2} \to \frac{1}{2}, \quad \cos x \to 1
\]

לכן הגבול הוא $1 \cdot \frac{1/2}{1} = \frac{1}{2}$. $\blacksquare$
\end{exercisebox}

\begin{exercisebox}
\textbf{תרגיל 2.}
חשבו $\limx{\infty} \left(\frac{x+1}{x-1}\right)^x$.

\textbf{פתרון:}
\[
\left(\frac{x+1}{x-1}\right)^x = \left(1 + \frac{2}{x-1}\right)^x = \left[\left(1 + \frac{2}{x-1}\right)^{(x-1)/2}\right]^{2x/(x-1)}
\]

כש־$x \to \infty$: $(1 + \frac{2}{x-1})^{(x-1)/2} \to e$ ו־$\frac{2x}{x-1} \to 2$.

לכן הגבול הוא $e^2$. $\blacksquare$
\end{exercisebox}
  % גבולות של פונקציות
% פרק 7: רציפות
\chapter{רציפות}

\section{הגדרת רציפות}

\begin{defbox}
\textbf{הגדרה 7.1 (רציפות בנקודה).}
$f$ \textbf{רציפה בנקודה $x_0$} אם $\limx{x_0} f(x) = f(x_0)$.

כלומר:
\[
\boxed{\forall \eps > 0 \; \exists \delta > 0 : |x - x_0| < \delta \Rightarrow |f(x) - f(x_0)| < \eps}
\]
\end{defbox}

\begin{defbox}
\textbf{הגדרה 7.2 (רציפות בקטע).}
$f$ \textbf{רציפה בקטע $I$} אם היא רציפה בכל נקודה של $I$.

(בקצוות סגורים — רציפות חד־צדדית.)
\end{defbox}

\begin{thmbox}
\textbf{טענה 7.3 (אפיון היינה לרציפות).}
$f$ רציפה ב־$x_0$ אם ורק אם לכל סדרה $x_n \to x_0$ מתקיים $f(x_n) \to f(x_0)$.
\end{thmbox}

\begin{exbox}
\textbf{דוגמה 1.}
פונקציית דיריכלה $D(x)$ אינה רציפה בשום נקודה.

\textbf{הוכחה:}
לכל $x_0 \in \R$, קיימות סדרות $r_n \to x_0$ (רציונליות) ו־$i_n \to x_0$ (אי־רציונליות).

$D(r_n) = 1$ לכל $n$, אבל $D(i_n) = 0$ לכל $n$.

לכן הגבול $\limx{x_0} D(x)$ לא קיים. $\blacksquare$
\end{exbox}

\section{אריתמטיקה של פונקציות רציפות}

\begin{thmbox}
\textbf{משפט 7.4.}
אם $f, g$ רציפות ב־$x_0$ אז גם:
\begin{enumerate}
    \item $f + g$ רציפה ב־$x_0$
    \item $f \cdot g$ רציפה ב־$x_0$
    \item $\frac{f}{g}$ רציפה ב־$x_0$ (אם $g(x_0) \neq 0$)
    \item $f \circ g$ רציפה ב־$x_0$ (אם $g$ רציפה ב־$x_0$ ו־$f$ רציפה ב־$g(x_0)$)
\end{enumerate}
\end{thmbox}

\begin{thmbox}
\textbf{טענה 7.5.}
כל פונקציה אלמנטרית רציפה בתחום הגדרתה.
\end{thmbox}

\section{משפט ערך הביניים}

\begin{thmbox}
\textbf{משפט 7.6 (משפט ערך הביניים — IVT).}
תהי $f : [a, b] \to \R$ רציפה. אם $f(a) < c < f(b)$ (או $f(b) < c < f(a)$), אז קיים $x_0 \in (a, b)$ כך ש־$f(x_0) = c$.
\end{thmbox}

\begin{proofbox}
\textbf{הוכחה (רעיון).}
נגדיר $A = \{x \in [a, b] : f(x) < c\}$.

$A$ לא ריקה (כי $a \in A$) וחסומה מלעיל (על ידי $b$).

נגדיר $x_0 = \sup A$.

מראים ש־$f(x_0) = c$ (אם $f(x_0) < c$ או $f(x_0) > c$ מגיעים לסתירה). $\blacksquare$
\end{proofbox}

\begin{exbox}
\textbf{דוגמה (קיום שורש).}
לכל פולינום ממעלה אי־זוגית יש שורש ממשי.

\textbf{הוכחה:}
יהי $p(x) = x^n + a_{n-1}x^{n-1} + \cdots + a_0$ עם $n$ אי־זוגי.

$\limx{\infty} p(x) = \infty$ ו־$\limx{-\infty} p(x) = -\infty$.

לכן קיימים $a < 0 < b$ כך ש־$p(a) < 0 < p(b)$.

לפי IVT, קיים $x_0 \in (a, b)$ כך ש־$p(x_0) = 0$. $\blacksquare$
\end{exbox}

\section{משפט ויירשטראס}

\begin{defbox}
\textbf{הגדרה 7.7 (קטע קומפקטי).}
קטע $[a, b]$ סגור וחסום נקרא \textbf{קומפקטי}.
\end{defbox}

\begin{thmbox}
\textbf{משפט 7.8 (ויירשטראס — קיום מקסימום ומינימום).}
תהי $f : [a, b] \to \R$ רציפה בקטע סגור וחסום. אז:
\begin{enumerate}
    \item $f$ חסומה.
    \item $f$ מקבלת מקסימום ומינימום: קיימים $x_1, x_2 \in [a, b]$ כך ש־$f(x_1) = \min f$ ו־$f(x_2) = \max f$.
\end{enumerate}
\end{thmbox}

\begin{proofbox}
\textbf{הוכחה (סעיף 2, קיום מקסימום).}
$f$ חסומה, לכן קיים $M = \sup\{f(x) : x \in [a,b]\}$.

לכל $n$, קיים $x_n \in [a, b]$ כך ש־$f(x_n) > M - \frac{1}{n}$.

$(x_n)$ סדרה חסומה ב־$[a, b]$, לכן (לפי בולצאנו־ויירשטראס) קיימת תת־סדרה $x_{n_k} \to x_0 \in [a, b]$.

מרציפות $f$: $f(x_{n_k}) \to f(x_0)$.

גם $f(x_{n_k}) \to M$ (כי $M - \frac{1}{n_k} < f(x_{n_k}) \le M$).

מיחידות הגבול: $f(x_0) = M$. $\blacksquare$
\end{proofbox}

\section{רציפות במידה שווה}

\begin{defbox}
\textbf{הגדרה 7.9 (רציפות במידה שווה).}
$f : D \to \R$ \textbf{רציפה במידה שווה} אם:
\[
\forall \eps > 0 \; \exists \delta > 0 \; \forall x, y \in D : |x - y| < \delta \Rightarrow |f(x) - f(y)| < \eps
\]
\end{defbox}

\begin{notebox}
\textbf{ההבדל:}
\begin{itemize}
    \item \textbf{רציפות:} $\delta$ תלוי ב־$\eps$ \textbf{וב־$x_0$}
    \item \textbf{רציפות במידה שווה:} $\delta$ תלוי \textbf{רק ב־$\eps$}, עובד לכל הנקודות
\end{itemize}
\end{notebox}

\begin{thmbox}
\textbf{משפט 7.10 (היינה־קנטור).}
אם $f : [a, b] \to \R$ רציפה בקטע סגור וחסום, אז $f$ רציפה במידה שווה.
\end{thmbox}

\begin{exbox}
\textbf{דוגמה (לא רציפה במידה שווה).}
$f(x) = \frac{1}{x}$ על $(0, 1]$ \textbf{אינה} רציפה במידה שווה.

\textbf{הוכחה:}
נבחר $\eps = 1$. לכל $\delta > 0$, נבחר $x = \delta$, $y = \frac{\delta}{2}$.

אז $|x - y| = \frac{\delta}{2} < \delta$, אבל:
\[
|f(x) - f(y)| = \left|\frac{1}{\delta} - \frac{2}{\delta}\right| = \frac{1}{\delta}
\]
עבור $\delta$ קטן מספיק, $\frac{1}{\delta} > 1 = \eps$. $\blacksquare$
\end{exbox}

\section{פונקציות הפיכות}

\begin{thmbox}
\textbf{משפט 7.11.}
תהי $f : [a, b] \to \R$ רציפה ומונוטונית ממש. אז:
\begin{enumerate}
    \item $f$ הפיכה
    \item $f^{-1} : f([a, b]) \to [a, b]$ רציפה
\end{enumerate}
\end{thmbox}

\section{תרגילים}

\begin{exercisebox}
\textbf{תרגיל 1.}
הוכיחו כי המשוואה $x^5 + x = 1$ יש לה פתרון יחיד בקטע $[0, 1]$.

\textbf{פתרון:}
נגדיר $f(x) = x^5 + x - 1$.

$f(0) = -1 < 0$, $f(1) = 1 > 0$.

לפי IVT, קיים $x_0 \in (0, 1)$ כך ש־$f(x_0) = 0$.

\textbf{יחידות:} $f'(x) = 5x^4 + 1 > 0$ לכל $x$, לכן $f$ עולה ממש, ולכן יש לכל היותר פתרון אחד. $\blacksquare$
\end{exercisebox}

\begin{exercisebox}
\textbf{תרגיל 2.}
תהי $f : [0, 1] \to [0, 1]$ רציפה. הוכיחו שקיימת נקודת שבת, כלומר $x_0 \in [0, 1]$ כך ש־$f(x_0) = x_0$.

\textbf{פתרון:}
נגדיר $g(x) = f(x) - x$.

$g(0) = f(0) - 0 = f(0) \ge 0$ (כי $f(0) \in [0, 1]$).

$g(1) = f(1) - 1 \le 0$ (כי $f(1) \in [0, 1]$).

אם $g(0) = 0$ אז $x_0 = 0$ נקודת שבת.

אם $g(1) = 0$ אז $x_0 = 1$ נקודת שבת.

אחרת, $g(0) > 0 > g(1)$, ולפי IVT קיים $x_0 \in (0, 1)$ כך ש־$g(x_0) = 0$, כלומר $f(x_0) = x_0$. $\blacksquare$
\end{exercisebox}

\begin{exercisebox}
\textbf{תרגיל 3.}
הוכיחו כי $f(x) = \sqrt{x}$ רציפה במידה שווה על $[0, \infty)$.

\textbf{פתרון:}
יהי $\eps > 0$.

\textbf{מקרה 1:} $x, y \ge \eps^2$. אז:
\[
|\sqrt{x} - \sqrt{y}| = \frac{|x - y|}{\sqrt{x} + \sqrt{y}} \le \frac{|x - y|}{2\eps}
\]
נבחר $\delta_1 = 2\eps^2$, אז $|x - y| < \delta_1 \Rightarrow |\sqrt{x} - \sqrt{y}| < \eps$.

\textbf{מקרה 2:} $x$ או $y$ קטנים מ־$\eps^2$. אז $|x - y| < \eps^2 \Rightarrow |\sqrt{x} - \sqrt{y}| < \eps$ (כי $\sqrt{a} < \eps$ אם $a < \eps^2$).

נבחר $\delta = \min(\delta_1, \eps^2) = \eps^2$. $\blacksquare$
\end{exercisebox}
  % רציפות

% ===================================
% חלק IV: גזירות
% ===================================
\part{גזירות}

% פרק 8: גזירות
\chapter{גזירות}

\section{הגדרת הנגזרת}

\begin{defbox}
\textbf{הגדרה 8.1 (נגזרת).}
תהי $f$ מוגדרת בסביבה של $x_0$. \textbf{הנגזרת} של $f$ בנקודה $x_0$ היא:
\[
\boxed{f'(x_0) = \limx{x_0} \frac{f(x) - f(x_0)}{x - x_0} = \lim_{h \to 0} \frac{f(x_0 + h) - f(x_0)}{h}}
\]
אם הגבול קיים וסופי, נאמר ש־$f$ \textbf{גזירה} ב־$x_0$.
\end{defbox}

\begin{thmbox}
\textbf{טענה 8.2.}
אם $f$ גזירה ב־$x_0$ אז $f$ רציפה ב־$x_0$.
\end{thmbox}

\begin{proofbox}
\textbf{הוכחה.}
\[
\limx{x_0} (f(x) - f(x_0)) = \limx{x_0} \frac{f(x) - f(x_0)}{x - x_0} \cdot (x - x_0) = f'(x_0) \cdot 0 = 0
\]
לכן $\limx{x_0} f(x) = f(x_0)$. $\blacksquare$
\end{proofbox}

\begin{notebox}
\textbf{אזהרה!}
ההפך לא נכון: $f(x) = |x|$ רציפה ב־$0$ אבל לא גזירה שם.
\end{notebox}

\section{כללי גזירה}

\begin{thmbox}
\textbf{משפט 8.3 (אריתמטיקה של נגזרות).}
אם $f, g$ גזירות ב־$x_0$ אז:
\begin{enumerate}
    \item $(f + g)'(x_0) = f'(x_0) + g'(x_0)$
    \item $(cf)'(x_0) = c \cdot f'(x_0)$
    \item $(fg)'(x_0) = f'(x_0)g(x_0) + f(x_0)g'(x_0)$ \textbf{(כלל המכפלה)}
    \item $\left(\frac{f}{g}\right)'(x_0) = \frac{f'(x_0)g(x_0) - f(x_0)g'(x_0)}{[g(x_0)]^2}$ (אם $g(x_0) \neq 0$) \textbf{(כלל המנה)}
\end{enumerate}
\end{thmbox}

\begin{thmbox}
\textbf{משפט 8.4 (כלל השרשרת).}
אם $g$ גזירה ב־$x_0$ ו־$f$ גזירה ב־$g(x_0)$, אז:
\[
\boxed{(f \circ g)'(x_0) = f'(g(x_0)) \cdot g'(x_0)}
\]
\end{thmbox}

\begin{thmbox}
\textbf{משפט 8.5 (נגזרת פונקציה הפוכה).}
אם $f$ גזירה והפיכה עם $f'(x_0) \neq 0$, אז:
\[
(f^{-1})'(f(x_0)) = \frac{1}{f'(x_0)}
\]
או בסימון אחר: $(f^{-1})'(y) = \frac{1}{f'(f^{-1}(y))}$.
\end{thmbox}

\section{טבלת נגזרות}

\begin{notebox}
\textbf{נגזרות חשובות:}
\begin{center}
\begin{tabular}{|c|c|}
\hline
$f(x)$ & $f'(x)$ \\
\hline
$x^n$ & $nx^{n-1}$ \\
$e^x$ & $e^x$ \\
$a^x$ & $a^x \ln a$ \\
$\ln x$ & $\frac{1}{x}$ \\
$\log_a x$ & $\frac{1}{x \ln a}$ \\
$\sin x$ & $\cos x$ \\
$\cos x$ & $-\sin x$ \\
$\tan x$ & $\frac{1}{\cos^2 x} = 1 + \tan^2 x$ \\
$\arcsin x$ & $\frac{1}{\sqrt{1-x^2}}$ \\
$\arccos x$ & $-\frac{1}{\sqrt{1-x^2}}$ \\
$\arctan x$ & $\frac{1}{1+x^2}$ \\
\hline
\end{tabular}
\end{center}
\end{notebox}

\section{משפטי ערך הביניים}

\begin{thmbox}
\textbf{משפט 8.6 (פרמה).}
אם $f$ מקבלת מקסימום או מינימום מקומי ב־$x_0$ פנימית, ו־$f$ גזירה ב־$x_0$, אז $f'(x_0) = 0$.
\end{thmbox}

\begin{thmbox}
\textbf{משפט 8.7 (רול).}
אם $f : [a, b] \to \R$ רציפה ב־$[a, b]$, גזירה ב־$(a, b)$, ו־$f(a) = f(b)$, אז קיים $c \in (a, b)$ כך ש־$f'(c) = 0$.
\end{thmbox}

\begin{thmbox}
\textbf{משפט 8.8 (לגרנז' / ערך הביניים לנגזרות).}
אם $f : [a, b] \to \R$ רציפה ב־$[a, b]$ וגזירה ב־$(a, b)$, אז קיים $c \in (a, b)$ כך ש:
\[
\boxed{f'(c) = \frac{f(b) - f(a)}{b - a}}
\]
\end{thmbox}

\begin{proofbox}
\textbf{הוכחה.}
נגדיר $g(x) = f(x) - \frac{f(b) - f(a)}{b - a}(x - a)$.

אז $g(a) = f(a)$ ו־$g(b) = f(a)$.

לפי רול, קיים $c \in (a, b)$ כך ש־$g'(c) = 0$.

$g'(c) = f'(c) - \frac{f(b) - f(a)}{b - a} = 0$. $\blacksquare$
\end{proofbox}

\begin{thmbox}
\textbf{משפט 8.9 (קושי).}
אם $f, g : [a, b] \to \R$ רציפות ב־$[a, b]$ וגזירות ב־$(a, b)$ עם $g'(x) \neq 0$ לכל $x \in (a, b)$, אז קיים $c \in (a, b)$ כך ש:
\[
\frac{f'(c)}{g'(c)} = \frac{f(b) - f(a)}{g(b) - g(a)}
\]
\end{thmbox}

\section{כלל לופיטל}

\begin{thmbox}
\textbf{משפט 8.10 (כלל לופיטל — צורה $\frac{0}{0}$).}
אם $\limx{a} f(x) = \limx{a} g(x) = 0$, ו־$g'(x) \neq 0$ בסביבה מנוקבת של $a$, וקיים $\limx{a} \frac{f'(x)}{g'(x)} = L$ (כולל $\pm\infty$), אז:
\[
\limx{a} \frac{f(x)}{g(x)} = L
\]
\end{thmbox}

\begin{thmbox}
\textbf{משפט 8.11 (כלל לופיטל — צורה $\frac{\infty}{\infty}$).}
אם $\limx{a} |f(x)| = \limx{a} |g(x)| = \infty$, וקיים $\limx{a} \frac{f'(x)}{g'(x)} = L$, אז:
\[
\limx{a} \frac{f(x)}{g(x)} = L
\]
\end{thmbox}

\begin{exbox}
\textbf{דוגמה 1.} חשבו $\limx{0} \frac{e^x - 1 - x}{x^2}$.

\textbf{פתרון:} צורה $\frac{0}{0}$. לופיטל:
\[
\limx{0} \frac{e^x - 1 - x}{x^2} = \limx{0} \frac{e^x - 1}{2x} = \limx{0} \frac{e^x}{2} = \frac{1}{2} \quad \blacksquare
\]
\end{exbox}

\begin{exbox}
\textbf{דוגמה 2.} חשבו $\limx{0^+} x \ln x$.

\textbf{פתרון:} צורה $0 \cdot (-\infty)$. נכתוב $x \ln x = \frac{\ln x}{1/x}$ (צורה $\frac{-\infty}{\infty}$).

לופיטל:
\[
\limx{0^+} \frac{\ln x}{1/x} = \limx{0^+} \frac{1/x}{-1/x^2} = \limx{0^+} (-x) = 0 \quad \blacksquare
\]
\end{exbox}

\begin{exbox}
\textbf{דוגמה 3.} חשבו $\limx{0^+} x^x$.

\textbf{פתרון:} צורה $0^0$. נכתוב $x^x = e^{x \ln x}$.

מדוגמה 2: $x \ln x \to 0$.

לכן $x^x = e^{x \ln x} \to e^0 = 1$. $\blacksquare$
\end{exbox}

\section{פולינום טיילור}

\begin{defbox}
\textbf{הגדרה 8.12 (פולינום טיילור).}
אם $f$ גזירה $n$ פעמים ב־$x_0$, \textbf{פולינום טיילור מדרגה $n$} סביב $x_0$ הוא:
\[
\boxed{P_n(x) = \sum_{k=0}^{n} \frac{f^{(k)}(x_0)}{k!}(x - x_0)^k = f(x_0) + f'(x_0)(x-x_0) + \frac{f''(x_0)}{2!}(x-x_0)^2 + \cdots}
\]
\end{defbox}

\begin{thmbox}
\textbf{משפט 8.13 (טיילור עם שארית לגרנז').}
אם $f$ גזירה $n+1$ פעמים ב־$[a, b]$, אז לכל $x \in [a, b]$ קיים $c$ בין $x_0$ ל־$x$ כך ש:
\[
f(x) = P_n(x) + R_n(x)
\]
כאשר השארית היא:
\[
R_n(x) = \frac{f^{(n+1)}(c)}{(n+1)!}(x - x_0)^{n+1}
\]
\end{thmbox}

\begin{notebox}
\textbf{פיתוחי מקלורן חשובים (סביב $x_0 = 0$):}
\begin{align*}
e^x &= \sum_{n=0}^{\infty} \frac{x^n}{n!} = 1 + x + \frac{x^2}{2!} + \frac{x^3}{3!} + \cdots \\
\sin x &= \sum_{n=0}^{\infty} \frac{(-1)^n x^{2n+1}}{(2n+1)!} = x - \frac{x^3}{3!} + \frac{x^5}{5!} - \cdots \\
\cos x &= \sum_{n=0}^{\infty} \frac{(-1)^n x^{2n}}{(2n)!} = 1 - \frac{x^2}{2!} + \frac{x^4}{4!} - \cdots \\
\ln(1+x) &= \sum_{n=1}^{\infty} \frac{(-1)^{n+1} x^n}{n} = x - \frac{x^2}{2} + \frac{x^3}{3} - \cdots \quad (|x| \le 1, x \neq -1) \\
\frac{1}{1-x} &= \sum_{n=0}^{\infty} x^n = 1 + x + x^2 + x^3 + \cdots \quad (|x| < 1) \\
(1+x)^\alpha &= \sum_{n=0}^{\infty} \binom{\alpha}{n} x^n = 1 + \alpha x + \frac{\alpha(\alpha-1)}{2!}x^2 + \cdots
\end{align*}
\end{notebox}

\section{חקירת פונקציות}

\begin{thmbox}
\textbf{טענה 8.14 (תנאי לעליה/ירידה).}
תהי $f$ גזירה בקטע $I$.
\begin{itemize}
    \item $f' > 0$ ב־$I$ $\Rightarrow$ $f$ עולה ממש ב־$I$
    \item $f' < 0$ ב־$I$ $\Rightarrow$ $f$ יורדת ממש ב־$I$
    \item $f' \ge 0$ ב־$I$ $\Rightarrow$ $f$ עולה (לא בהכרח ממש) ב־$I$
\end{itemize}
\end{thmbox}

\begin{thmbox}
\textbf{טענה 8.15 (מבחן הנגזרת השנייה לקיצון).}
אם $f'(x_0) = 0$ ו־$f''(x_0) \neq 0$:
\begin{itemize}
    \item $f''(x_0) > 0$ $\Rightarrow$ $x_0$ נקודת מינימום מקומי
    \item $f''(x_0) < 0$ $\Rightarrow$ $x_0$ נקודת מקסימום מקומי
\end{itemize}
\end{thmbox}

\begin{defbox}
\textbf{הגדרה 8.16 (קעירות).}
$f$ \textbf{קעורה כלפי מעלה} (קמורה) בקטע אם $f'' > 0$ בקטע.

$f$ \textbf{קעורה כלפי מטה} (קעורה) בקטע אם $f'' < 0$ בקטע.

\textbf{נקודת פיתול}: נקודה שבה $f$ משנה קעירות.
\end{defbox}

\section{תרגילים}

\begin{exercisebox}
\textbf{תרגיל 1.}
חשבו $\limx{0} \frac{\sin x - x + \frac{x^3}{6}}{x^5}$.

\textbf{פתרון:}
נשתמש בפיתוח טיילור: $\sin x = x - \frac{x^3}{6} + \frac{x^5}{120} - O(x^7)$.

\[
\sin x - x + \frac{x^3}{6} = \frac{x^5}{120} - O(x^7)
\]

\[
\frac{\sin x - x + \frac{x^3}{6}}{x^5} = \frac{1}{120} - O(x^2) \to \frac{1}{120} \quad \blacksquare
\]
\end{exercisebox}

\begin{exercisebox}
\textbf{תרגיל 2.}
חקרו את הפונקציה $f(x) = x e^{-x}$ (מקסימום, מינימום, קעירות, אסימפטוטות).

\textbf{פתרון:}
\textbf{תחום:} $\R$.

\textbf{נגזרות:}
\[
f'(x) = e^{-x} - xe^{-x} = e^{-x}(1 - x)
\]
\[
f''(x) = -e^{-x}(1-x) - e^{-x} = e^{-x}(x - 2)
\]

\textbf{נקודות קריטיות:} $f'(x) = 0 \Rightarrow x = 1$.

$f''(1) = e^{-1}(1-2) = -e^{-1} < 0$ $\Rightarrow$ $x = 1$ מקסימום מקומי.

$f(1) = e^{-1} = \frac{1}{e}$.

\textbf{קעירות:} $f''(x) = 0 \Rightarrow x = 2$.

$x < 2$: $f'' < 0$ (קעורה כלפי מטה).

$x > 2$: $f'' > 0$ (קעורה כלפי מעלה).

$x = 2$ נקודת פיתול, $f(2) = 2e^{-2}$.

\textbf{אסימפטוטות:}
\[
\limx{-\infty} xe^{-x} = -\infty, \quad \limx{\infty} xe^{-x} = 0 \text{ (לופיטל)}
\]
אסימפטוטה אופקית $y = 0$ ב־$x \to \infty$. $\blacksquare$
\end{exercisebox}

\begin{exercisebox}
\textbf{תרגיל 3.}
הוכיחו כי לכל $x > 0$: $\ln(1 + x) < x$.

\textbf{פתרון:}
נגדיר $f(x) = x - \ln(1+x)$ לכל $x > 0$.

$f(0) = 0$.

$f'(x) = 1 - \frac{1}{1+x} = \frac{x}{1+x} > 0$ לכל $x > 0$.

לכן $f$ עולה ממש ב־$(0, \infty)$.

מכאן $f(x) > f(0) = 0$ לכל $x > 0$, כלומר $x > \ln(1+x)$. $\blacksquare$
\end{exercisebox}
  % גזירות, לופיטל, טיילור

% ===================================
% חלק V: אינטגרל רימן
% ===================================
\part{אינטגרל רימן}

\input{units/unit1}  % פונקציות קדומות
% יחידה 2 - אינטגרל מסוים וסכומי רימן
%====================================
\section{יחידה 2: אינטגרל מסוים וסכומי רימן}

\subsection{מבוא}
%====================================

ביחידה זו נגדיר את \textbf{האינטגרל המסוים} באמצעות סכומי דרבו. נלמד מתי פונקציה היא אינטגרבילית ונכיר את תכונות האינטגרל.

%====================================
\subsection{חלוקות ועידונים}
%====================================

\begin{defbox}
\textbf{הגדרה 9.1: חלוקה}

יהי $[a, b]$ קטע. \textbf{חלוקה} של הקטע $[a, b]$ היא קבוצה $\Pi = \{x_i\}_{i=0}^n$ של נקודות המקיימות:
\[
a = x_0 < x_1 < \cdots < x_n = b
\]

\textbf{הפרמטר של החלוקה} מסומן $\lambda(\Pi)$ ומוגדר:
\[
\lambda(\Pi) = \max_{0 \leq i \leq n-1} (x_{i+1} - x_i)
\]
\end{defbox}

\begin{defbox}
\textbf{הגדרה 9.2: עידון}

יהי $[a, b]$ קטע. תהיינה $\Pi_1, \Pi_2$ חלוקות של $[a, b]$. נאמר כי $\Pi_2$ היא \textbf{עידון} של $\Pi_1$ כאשר $\Pi_1 \subseteq \Pi_2$.
\end{defbox}

\begin{notebox}
\textbf{הערה:}
\begin{itemize}
    \item אם $\Pi_2$ היא עידון של $\Pi_1$ אז $\lambda(\Pi_2) \leq \lambda(\Pi_1)$.
    \item לכל שתי חלוקות קיימת חלוקה שהיא עידון של שתיהן (האיחוד שלהן).
\end{itemize}
\end{notebox}

%====================================
\subsection{סכומי דרבו}
%====================================

\begin{defbox}
\textbf{הגדרה 9.3: סכומי דרבו}

תהי $f : [a, b] \to \R$ פונקציה \textbf{חסומה} ותהי $\Pi = \{x_i\}_{i=0}^n$ חלוקה של $[a, b]$.

\textbf{סכום דרבו העליון} של $f$ ביחס לחלוקה $\Pi$:
\[
\overline{S}(f, \Pi) = \sum_{i=0}^{n-1} \left( \sup_{x \in [x_i, x_{i+1}]} f(x) \right) \cdot (x_{i+1} - x_i)
\]

\textbf{סכום דרבו התחתון} של $f$ ביחס לחלוקה $\Pi$:
\[
\underline{S}(f, \Pi) = \sum_{i=0}^{n-1} \left( \inf_{x \in [x_i, x_{i+1}]} f(x) \right) \cdot (x_{i+1} - x_i)
\]
\end{defbox}

\begin{thmbox}
\textbf{טענה 9.4: תכונות סכומי דרבו}

תהי $f : [a, b] \to \R$ פונקציה חסומה. אז:
\begin{enumerate}
    \item לכל חלוקה $\Pi$ של $[a, b]$ מתקיים:
    \[
    (b-a) \cdot \inf f \leq \underline{S}(f, \Pi) \leq \overline{S}(f, \Pi) \leq (b-a) \cdot \sup f
    \]

    \item לכל שתי חלוקות $\Pi_1, \Pi_2$ של $[a, b]$, אם $\Pi_2$ היא עידון של $\Pi_1$ אז:
    \[
    \underline{S}(f, \Pi_1) \leq \underline{S}(f, \Pi_2) \leq \overline{S}(f, \Pi_2) \leq \overline{S}(f, \Pi_1)
    \]

    \item לכל שתי חלוקות $\Pi_1, \Pi_2$ של $[a, b]$ מתקיים:
    \[
    \underline{S}(f, \Pi_1) \leq \overline{S}(f, \Pi_2)
    \]
\end{enumerate}
\end{thmbox}

\begin{proofbox}
\textbf{הוכחה (רעיון):}

\textbf{(1)} לכל $i$ מתקיים $\inf f \leq \inf_{[x_i, x_{i+1}]} f \leq \sup_{[x_i, x_{i+1}]} f \leq \sup f$. כפל ב־$(x_{i+1} - x_i) > 0$ וסכימה נותנים את אי-השוויון.

\textbf{(2)} מספיק להוכיח עבור עידון בנקודה אחת. אם $\Pi_2 = \Pi_1 \cup \{y\}$ כאשר $y \in (x_k, x_{k+1})$, אז:
\[
\sup_{[x_k, x_{k+1}]} f \cdot (x_{k+1} - x_k) \geq \sup_{[x_k, y]} f \cdot (y - x_k) + \sup_{[y, x_{k+1}]} f \cdot (x_{k+1} - y)
\]

\textbf{(3)} קיים עידון משותף $\Pi_3$ של $\Pi_1$ ושל $\Pi_2$. לפי (2):
\[
\underline{S}(f, \Pi_1) \leq \underline{S}(f, \Pi_3) \leq \overline{S}(f, \Pi_3) \leq \overline{S}(f, \Pi_2)
\]
\end{proofbox}

%====================================
\subsection{האינטגרל העליון והתחתון}
%====================================

\begin{defbox}
\textbf{הגדרה 9.5: אינטגרל עליון ותחתון}

תהי $f : [a, b] \to \R$ פונקציה חסומה.

\textbf{האינטגרל העליון} של $f$ ב־$[a, b]$:
\[
\overline{\int_a^b} f(x) \dx = \inf \left\{ \overline{S}(f, \Pi) : \Pi \text{ חלוקה של } [a, b] \right\}
\]

\textbf{האינטגרל התחתון} של $f$ ב־$[a, b]$:
\[
\underline{\int_a^b} f(x) \dx = \sup \left\{ \underline{S}(f, \Pi) : \Pi \text{ חלוקה של } [a, b] \right\}
\]
\end{defbox}

\begin{notebox}
\textbf{הערה 9.6:}

לכל פונקציה חסומה $f : [a, b] \to \R$ מתקיים:
\[
\underline{\int_a^b} f(x) \dx \leq \overline{\int_a^b} f(x) \dx
\]
\end{notebox}

%====================================
\subsection{אינטגרביליות לפי דרבו}
%====================================

\begin{defbox}
\textbf{הגדרה 9.7: אינטגרביליות}

תהי $f : [a, b] \to \R$ פונקציה חסומה. נאמר כי $f$ \textbf{אינטגרבילית (לפי דרבו)} ב־$[a, b]$ כאשר:
\[
\underline{\int_a^b} f(x) \dx = \overline{\int_a^b} f(x) \dx
\]
במקרה זה נסמן:
\[
\int_a^b f(x) \dx = \underline{\int_a^b} f(x) \dx = \overline{\int_a^b} f(x) \dx
\]
\end{defbox}

\begin{thmbox}
\textbf{טענה 9.8: קריטריון אינטגרביליות}

תהי $f : [a, b] \to \R$ פונקציה חסומה. $f$ אינטגרבילית ב־$[a, b]$ \textbf{אם ורק אם} לכל $\eps > 0$ קיימת חלוקה $\Pi$ של $[a, b]$ כך ש:
\[
\overline{S}(f, \Pi) - \underline{S}(f, \Pi) < \eps
\]
\end{thmbox}

%====================================
\subsection{דוגמאות לאינטגרביליות}
%====================================

\begin{exbox}
\textbf{דוגמה: פונקציה קבועה}

תהי $f(x) = \alpha$ לכל $x \in [a, b]$.

לכל חלוקה $\Pi$: $\overline{S}(f, \Pi) = \underline{S}(f, \Pi) = \alpha(b-a)$.

לכן $f$ אינטגרבילית ומתקיים $\int_a^b \alpha \dx = \alpha(b-a)$.
\end{exbox}

\begin{exbox}
\textbf{דוגמה: פונקציית דיריכלה (לא אינטגרבילית)}

הפונקציה $D : [0, 1] \to \R$ המוגדרת:
\[
D(x) = \begin{cases} 1, & x \in \Q \\ 0, & x \notin \Q \end{cases}
\]

לכל חלוקה $\Pi$ ולכל תת-קטע $[x_i, x_{i+1}]$:
\begin{itemize}
    \item $\sup_{[x_i, x_{i+1}]} D = 1$ (כי יש רציונליים בכל קטע)
    \item $\inf_{[x_i, x_{i+1}]} D = 0$ (כי יש אי-רציונליים בכל קטע)
\end{itemize}

לכן $\overline{S}(D, \Pi) = 1$ ו־$\underline{S}(D, \Pi) = 0$ לכל חלוקה.

מכאן: $\overline{\int_0^1} D = 1$, $\underline{\int_0^1} D = 0$, ולכן $D$ \textbf{לא אינטגרבילית}.
\end{exbox}

\begin{exbox}
\textbf{דוגמה: $f(x) = x^2$ ב־$[0, 1]$}

לחלוקה שווה $\Pi_n$ עם $x_i = \frac{i}{n}$:
\[
\underline{S}(f, \Pi_n) = \frac{1}{n^3} \sum_{i=0}^{n-1} i^2 = \frac{(n-1)n(2n-1)}{6n^3} \xrightarrow{n \to \infty} \frac{1}{3}
\]
\[
\overline{S}(f, \Pi_n) = \frac{1}{n^3} \sum_{i=1}^{n} i^2 = \frac{n(n+1)(2n+1)}{6n^3} \xrightarrow{n \to \infty} \frac{1}{3}
\]

לכן $\int_0^1 x^2 \dx = \frac{1}{3}$.
\end{exbox}

%====================================
\subsection{משפטי אינטגרביליות}
%====================================

\begin{thmbox}
\textbf{טענה 9.17: פונקציה רציפה היא אינטגרבילית}

תהי $f : [a, b] \to \R$ פונקציה \textbf{רציפה}. אז $f$ אינטגרבילית ב־$[a, b]$.
\end{thmbox}

\begin{proofbox}
\textbf{הוכחה (רעיון):}

$f$ רציפה בקטע סגור וחסום, לכן רציפה במידה שווה. לכל $\eps > 0$ קיים $\delta > 0$ כך שאם $|x - y| < \delta$ אז $|f(x) - f(y)| < \frac{\eps}{b-a}$.

ניקח חלוקה $\Pi$ עם $\lambda(\Pi) < \delta$. בכל תת-קטע $[x_i, x_{i+1}]$:
\[
\sup_{[x_i, x_{i+1}]} f - \inf_{[x_i, x_{i+1}]} f < \frac{\eps}{b-a}
\]

לכן:
\[
\overline{S}(f, \Pi) - \underline{S}(f, \Pi) < \frac{\eps}{b-a} \cdot (b-a) = \eps
\]
\end{proofbox}

\begin{thmbox}
\textbf{טענה 9.20: פונקציה מונוטונית היא אינטגרבילית}

תהי $f : [a, b] \to \R$ פונקציה \textbf{מונוטונית}. אז $f$ אינטגרבילית ב־$[a, b]$.
\end{thmbox}

\begin{thmbox}
\textbf{טענה 9.21: פונקציה חסומה עם מספר סופי של נקודות אי-רציפות}

תהי $f : [a, b] \to \R$ פונקציה חסומה שרציפה בכל הנקודות מלבד \textbf{מספר סופי} של נקודות. אז $f$ אינטגרבילית ב־$[a, b]$.
\end{thmbox}

%====================================
\subsection{תכונות האינטגרל המסוים}
%====================================

\begin{thmbox}
\textbf{תכונות האינטגרל:}

תהיינה $f, g : [a, b] \to \R$ אינטגרביליות. אז:

\textbf{1. לינאריות:}
\begin{itemize}
    \item $\int_a^b (f + g) \dx = \int_a^b f \dx + \int_a^b g \dx$
    \item $\int_a^b \alpha f \dx = \alpha \int_a^b f \dx$ לכל $\alpha \in \R$
\end{itemize}

\textbf{2. מונוטוניות:} אם $f(x) \leq g(x)$ לכל $x \in [a, b]$ אז $\int_a^b f \dx \leq \int_a^b g \dx$.

\textbf{3. אדיטיביות בתחום:} לכל $c \in (a, b)$:
\[
\int_a^b f \dx = \int_a^c f \dx + \int_c^b f \dx
\]

\textbf{4. אי-שוויון המשולש:}
\[
\left| \int_a^b f \dx \right| \leq \int_a^b |f| \dx
\]
\end{thmbox}

\begin{thmbox}
\textbf{טענה 9.25: מכפלת פונקציות אינטגרביליות}

תהיינה $f, g : [a, b] \to \R$ אינטגרביליות. אז $f \cdot g$ אינטגרבילית.
\end{thmbox}

\begin{proofbox}
\textbf{הוכחה (רעיון):}

ראשית מוכיחים שאם $f$ אינטגרבילית אז $f^2$ אינטגרבילית. לאחר מכן משתמשים בזהות:
\[
f \cdot g = \frac{1}{4} \big( (f+g)^2 - (f-g)^2 \big)
\]
\end{proofbox}

\begin{thmbox}
\textbf{טענה 9.27: הרכבה עם פונקציה רציפה}

תהי $f : [a, b] \to [c, d]$ אינטגרבילית ותהי $g : [c, d] \to \R$ רציפה. אז $g \circ f$ אינטגרבילית.
\end{thmbox}

%====================================
\subsection{פונקציית רימן}
%====================================

\begin{exbox}
\textbf{פונקציית רימן -- אינטגרבילית עם אינסוף נקודות אי-רציפות}

הפונקציה $R : [0, 1] \to \R$ מוגדרת:
\[
R(x) = \begin{cases} \frac{1}{q}, & x = \frac{p}{q} \text{ בצמצום}, p, q \in \N_+, \gcd(p, q) = 1 \\ 0, & x \notin \Q \text{ או } x = 0 \end{cases}
\]

\textbf{טענה:} $R$ אינטגרבילית ב־$[0, 1]$ ומתקיים $\int_0^1 R(x) \dx = 0$.

\textbf{רעיון ההוכחה:} לכל $\eps > 0$ יש רק מספר \textbf{סופי} של נקודות $x$ עבורן $R(x) \geq \eps$. מכסים אותן בקטעים קטנים, ובשאר $\sup R < \eps$.
\end{exbox}

%====================================
\subsection{תרגילים}
%====================================

\begin{exercisebox}
\textbf{תרגילים:}
\begin{enumerate}
    \item יהי $[a, b]$ קטע ו־$\alpha \in \R$. הראו כי $f(x) = \alpha$ אינטגרבילית ומתקיים $\int_a^b \alpha \dx = \alpha(b-a)$.

    \item תהיינה $f, g$ אינטגרביליות עם $f \geq g$ ו־$f(x_0) > g(x_0)$ בנקודה אחת. האם בהכרח $\int_a^b f > \int_a^b g$?

    \textbf{תשובה:} לא בהכרח! אם $f(x_0) > g(x_0)$ רק בנקודה בודדת, האינטגרל לא משתנה. אבל \textbf{אם $f$ רציפה} ב־$x_0$, אז קיימת סביבה שבה $f > g$, ואז $\int_a^b f > \int_a^b g$.

    \item תהי $f : [-a, a] \to \R$ אינטגרבילית. הוכיחו:
    \begin{itemize}
        \item אם $f$ זוגית: $\int_{-a}^a f = 2\int_0^a f$
        \item אם $f$ אי-זוגית: $\int_{-a}^a f = 0$
    \end{itemize}

    \item תהיינה $f, g$ אינטגרביליות. הוכיחו כי $\max(f, g)$ ו־$\min(f, g)$ אינטגרביליות.

    \textbf{רמז:} $\max(f, g) = \frac{1}{2}(f + g + |f - g|)$, $\min(f, g) = \frac{1}{2}(f + g - |f - g|)$
\end{enumerate}
\end{exercisebox}
  % אינטגרל רימן
% יחידה 3: משפט היסוד של החדו"א
\section{משפט היסוד של החדו"א}

יחידה זו עוסקת בקשר העמוק בין אינטגרציה וגזירה — המשפטים היסודיים של החשבון הדיפרנציאלי והאינטגרלי.

% ===================================
\section{אינטגרביליות מקומית}
% ===================================

\begin{defbox}
\textbf{הגדרה 10.1 (אינטגרביליות מקומית).}
יהי $I \subseteq \R$ קטע ותהי $f : I \to \R$ פונקציה. נאמר כי \textbf{$f$ אינטגרבילית מקומית על $I$} כאשר לכל $a,b \in I$ עם $a < b$ הפונקציה $f$ אינטגרבילית על $[a,b]$.
\end{defbox}

\begin{exbox}
\textbf{דוגמה.}
תהי $f : \R \to \R$ רציפה. אז $f$ אינטגרבילית מקומית על כל קטע, כי פונקציה רציפה בקטע סגור היא אינטגרבילית.
\end{exbox}

\begin{notebox}
\textbf{מוסכמה.}
אם $b < a$ אז נגדיר:
\[
\int_a^b f(x)\dx = -\int_b^a f(x)\dx
\]
\end{notebox}

% ===================================
\section{אינטגרל לא מסוים}
% ===================================

\begin{defbox}
\textbf{הגדרה 10.2 (אינטגרל לא מסוים).}
יהי $I$ קטע ותהי $f : I \to \R$ אינטגרבילית מקומית ב־$I$. פונקציה $F : I \to \R$ נקראת \textbf{אינטגרל לא מסוים של $f$} כאשר קיים $a \in I$ כך שמתקיים:
\[
F(x) = \int_a^x f(t)\dt \quad \text{לכל } x \in I
\]
\end{defbox}

\begin{exercisebox}
\textbf{תרגיל.}
יהיו $F, G : I \to \R$ אינטגרלים לא מסוימים של $f$. הוכיחו כי קיים $c \in \R$ כך ש־$F(x) - G(x) = c$ לכל $x \in I$.
\end{exercisebox}

% ===================================
\section{רציפות האינטגרל הלא מסוים}
% ===================================

\begin{thmbox}
\textbf{טענה 10.3 (רציפות האינטגרל הלא מסוים).}
יהי $I$ קטע, תהי $f : I \to \R$ אינטגרבילית מקומית, ויהי $F : I \to \R$ אינטגרל לא מסוים של $f$.
\begin{enumerate}
    \item $F$ \textbf{רציפה} ב־$I$.
    \item אם בנוסף $f$ \textbf{חסומה} ב־$I$, אז $F$ היא \textbf{פונקציית ליפשיץ} (קיים $M > 0$ כך ש־$|F(x) - F(y)| \le M|x-y|$ לכל $x,y \in I$).
\end{enumerate}
\end{thmbox}

\begin{proofbox}
\textbf{הוכחה.}
יהי $F(x) = \int_a^x f(t)\dt$. יהי $x_0 \in I$.

\textbf{שלב 1:} כיוון ש־$f$ אינטגרבילית מקומית, קיים $\eta > 0$ כך ש־$f$ חסומה ב־$[x_0-\eta, x_0+\eta] \cap I$ על ידי $M > 0$.

\textbf{שלב 2:} לכל $x$ עם $|x - x_0| < \eta$:
\[
|F(x) - F(x_0)| = \abs{\int_{x_0}^x f(t)\dt} \le M|x - x_0|
\]

\textbf{שלב 3:} נבחר $\delta = \min\parens{\eta, \frac{\eps}{M}}$. אז לכל $x$ עם $|x-x_0| < \delta$:
\[
|F(x) - F(x_0)| \le M \cdot |x-x_0| < M \cdot \frac{\eps}{M} = \eps
\]
לכן $F$ רציפה ב־$x_0$. \hfill $\blacksquare$
\end{proofbox}

% ===================================
\section{המשפט היסודי הראשון}
% ===================================

\begin{thmbox}
\textbf{טענה 10.4 (גזירות האינטגרל הלא מסוים).}
יהי $I$ קטע, $x_0 \in I$, $f : I \to \R$ אינטגרבילית מקומית, ו־$F$ אינטגרל לא מסוים של $f$.

\textbf{אם $f$ רציפה ב־$x_0$} אז \textbf{$F$ גזירה ב־$x_0$} ומתקיים:
\[
\boxed{F'(x_0) = f(x_0)}
\]
\end{thmbox}

\begin{proofbox}
\textbf{הוכחה.}
יהי $\eps > 0$. כיוון ש־$f$ רציפה ב־$x_0$, קיים $\delta > 0$ כך שלכל $t$ עם $|t - x_0| < \delta$:
\[
|f(t) - f(x_0)| < \frac{\eps}{2}
\]

נבחן את המנה:
\[
\frac{F(x) - F(x_0)}{x - x_0} - f(x_0) = \frac{1}{x - x_0} \int_{x_0}^x f(t)\dt - f(x_0)
\]

נשתמש בכך ש־$f(x_0) = \frac{1}{x-x_0}\int_{x_0}^x f(x_0)\dt$:
\[
\frac{F(x) - F(x_0)}{x - x_0} - f(x_0) = \frac{1}{x - x_0} \int_{x_0}^x \parens{f(t) - f(x_0)}\dt
\]

לכל $|x - x_0| < \delta$:
\[
\abs{\frac{F(x) - F(x_0)}{x - x_0} - f(x_0)} \le \frac{1}{|x-x_0|} \cdot \frac{\eps}{2} \cdot |x-x_0| = \frac{\eps}{2} < \eps
\]

לכן $\limx{x_0} \frac{F(x) - F(x_0)}{x - x_0} = f(x_0)$, כלומר $F'(x_0) = f(x_0)$. \hfill $\blacksquare$
\end{proofbox}

% ===================================
\section{המשפט היסודי של החשבון הדיפרנציאלי}
% ===================================

\begin{thmbox}
\textbf{משפט 10.5 (המשפט היסודי של החשבון הדיפרנציאלי).}
יהי $I$ קטע ותהי $f : I \to \R$.

\textbf{אם $f$ רציפה} אז \textbf{קיימת ל־$f$ פונקציה קדומה ב־$I$}.
\end{thmbox}

\begin{proofbox}
\textbf{הוכחה.}
נבחר $a \in I$ ונגדיר:
\[
F(x) = \int_a^x f(t)\dt
\]

לפי טענה 10.4, בכל נקודה $x \in I$ (שם $f$ רציפה) מתקיים $F'(x) = f(x)$.

לכן $F$ היא קדומה של $f$ ב־$I$. \hfill $\blacksquare$
\end{proofbox}

\begin{notebox}
\textbf{משמעות המשפט.}
המשפט קושר בין שני מושגים נפרדים לכאורה:
\begin{itemize}
    \item \textbf{גזירה} — מציאת קצב שינוי מקומי
    \item \textbf{אינטגרציה} — מציאת שטח מצטבר
\end{itemize}
המשפט מראה שאינטגרציה היא הפעולה ההפוכה לגזירה!
\end{notebox}

% ===================================
\section{נוסחת ניוטון־לייבניץ (המשפט היסודי השני)}
% ===================================

\begin{thmbox}
\textbf{משפט 10.6 (נוסחת ניוטון־לייבניץ).}
תהי $f : [a,b] \to \R$. \textbf{אם $f$ אינטגרבילית ובעלת קדומה ב־$[a,b]$} אז \textbf{לכל קדומה $F$ של $f$} מתקיים:
\[
\boxed{\int_a^b f(x)\dx = F(b) - F(a)}
\]
\end{thmbox}

\begin{notebox}
\textbf{סימון.}
$F(b) - F(a) = \big[F(x)\big]_a^b$
\end{notebox}

\begin{proofbox}
\textbf{הוכחה (רעיון).}
\textbf{שלב 1:} בוחרים חלוקה $\Pi = \{x_0 = a, x_1, \ldots, x_n = b\}$ עם $\lambda(\Pi) < \delta$ כך שסכום רימן קרוב לאינטגרל.

\textbf{שלב 2:} לפי משפט לגרנז', בכל תת־קטע $[x_i, x_{i+1}]$ קיים $\xi_i \in (x_i, x_{i+1})$ כך ש:
\[
F(x_{i+1}) - F(x_i) = F'(\xi_i)(x_{i+1} - x_i) = f(\xi_i)(x_{i+1} - x_i)
\]

\textbf{שלב 3:} סכימה:
\[
F(b) - F(a) = \sum_{i=0}^{n-1} \big[F(x_{i+1}) - F(x_i)\big] = \sum_{i=0}^{n-1} f(\xi_i)(x_{i+1} - x_i) = S(f, \Pi, \xi)
\]

\textbf{שלב 4:} בגבול $\lambda(\Pi) \to 0$, סכום רימן שואף לאינטגרל:
\[
F(b) - F(a) = \int_a^b f(x)\dx \quad \blacksquare
\]
\end{proofbox}

% ===================================
\section{דוגמאות לחישוב אינטגרלים}
% ===================================

\begin{exbox}
\textbf{דוגמה 1.}
\[
\int_0^1 x\dx = \brackets{\frac{x^2}{2}}_0^1 = \frac{1}{2} - 0 = \frac{1}{2}
\]
\end{exbox}

\begin{exbox}
\textbf{דוגמה 2.} חישוב $\int_1^e \ln x\dx$.

ידוע כי $x\ln x - x$ היא קדומה של $\ln x$ (ניתן לאמת בגזירה). לכן:
\[
\int_1^e \ln x\dx = \big[x\ln x - x\big]_1^e = (e \cdot 1 - e) - (1 \cdot 0 - 1) = 0 - (-1) = 1
\]
\end{exbox}

\begin{exbox}
\textbf{דוגמה 3.} חישוב $\int_0^{\pi/2} \sin^2 x\dx$.

משתמשים בזהות $\sin^2 x = \frac{1 - \cos(2x)}{2}$:
\[
\int_0^{\pi/2} \sin^2 x\dx = \int_0^{\pi/2} \frac{1 - \cos(2x)}{2}\dx = \frac{1}{2}\brackets{x - \frac{\sin(2x)}{2}}_0^{\pi/2} = \frac{1}{2}\parens{\frac{\pi}{2} - 0} = \frac{\pi}{4}
\]
\end{exbox}

% ===================================
\section{גזירה של אינטגרל עם גבולות משתנים}
% ===================================

\begin{thmbox}
\textbf{מסקנה 1 (גבול עליון משתנה).}
אם $f$ רציפה ב־$I$ ו־$a \in I$, אז הפונקציה:
\[
F(x) = \int_a^x f(t)\dt
\]
מקיימת $F'(x) = f(x)$ לכל $x \in I$.
\end{thmbox}

\begin{thmbox}
\textbf{מסקנה 2 (גבולות כלליים — כלל לייבניץ).}
אם $f$ רציפה, ו־$u(x), v(x)$ גזירות, אז:
\[
\boxed{\frac{d}{dx} \int_{u(x)}^{v(x)} f(t)\dt = f(v(x)) \cdot v'(x) - f(u(x)) \cdot u'(x)}
\]
\end{thmbox}

\begin{proofbox}
\textbf{הוכחה.}
נגדיר $G(x) = \int_a^x f(t)\dt$ כך ש־$G'(x) = f(x)$.

אז:
\[
\int_{u(x)}^{v(x)} f(t)\dt = G(v(x)) - G(u(x))
\]

לפי כלל השרשרת:
\[
\frac{d}{dx}\big[G(v(x)) - G(u(x))\big] = G'(v(x)) \cdot v'(x) - G'(u(x)) \cdot u'(x) = f(v(x)) \cdot v'(x) - f(u(x)) \cdot u'(x)
\]
\hfill $\blacksquare$
\end{proofbox}

\begin{exbox}
\textbf{דוגמה.} חשבו $\frac{d}{dx} \int_0^{x^2} e^{-t^2}\dt$.

\textbf{פתרון:} כאן $u(x) = 0$, $v(x) = x^2$, $f(t) = e^{-t^2}$.

לפי כלל לייבניץ:
\[
\frac{d}{dx} \int_0^{x^2} e^{-t^2}\dt = e^{-(x^2)^2} \cdot 2x - e^{-0} \cdot 0 = 2x e^{-x^4}
\]
\end{exbox}

% ===================================
\section{הערות חשובות}
% ===================================

\begin{notebox}
\textbf{הערות 10.7.}
\begin{enumerate}
    \item \textbf{נגזרת של פונקציה גזירה} אינה בהכרח אינטגרבילית רימן.

    \textbf{דוגמה:} הפונקציה $F(x) = x^2\sin\parens{\frac{1}{x^2}}$ עבור $x \neq 0$ ו־$F(0) = 0$ היא גזירה ב־$[0,1]$, אך $F'$ אינה חסומה ולכן אינה אינטגרבילית רימן.

    \item יש דוגמאות לנגזרת \textbf{חסומה} שאינה אינטגרבילית רימן (למשל פונקציית Volterra).
\end{enumerate}
\end{notebox}

% ===================================
\section{תרגילים}
% ===================================

\begin{exercisebox}
\textbf{תרגיל 1.}
חשבו:
\begin{enumerate}
    \item $\int_0^1 \frac{x}{1+x^2}\dx$
    \item $\int_1^4 \frac{1}{\sqrt{x}(1+\sqrt{x})}\dx$
    \item $\int_0^\pi x\sin x\dx$
\end{enumerate}
\end{exercisebox}

\begin{exercisebox}
\textbf{תרגיל 2.}
מצאו את $\limx{0} \frac{\int_0^x \sin(t^2)\dt}{x^3}$.

\textbf{רמז:} השתמשו בכלל לופיטל או בפיתוח טיילור.
\end{exercisebox}

\begin{exercisebox}
\textbf{תרגיל 3.}
תהי $f$ רציפה ב־$\R$. הוכיחו כי אם $\int_0^x f(t)\dt = 0$ לכל $x \in \R$, אז $f \equiv 0$.
\end{exercisebox}

\begin{exercisebox}
\textbf{תרגיל 4.}
תהי $f : [a,b] \to \R$ רציפה. הוכיחו כי:
\[
\limx{0^+} \frac{1}{h} \int_a^{a+h} f(x)\dx = f(a)
\]
\end{exercisebox}
  % המשפט היסודי
% יחידה 4: שיטות אינטגרציה לאינטגרל מסוים
\section{שיטות אינטגרציה לאינטגרל מסוים}

יחידה זו עוסקת בהתאמת שיטות האינטגרציה (אינטגרציה בחלקים ושינוי משתנה) לאינטגרל המסוים, ובמשפט ערך הביניים לאינטגרלים.

% ===================================
\section{אינטגרציה בחלקים לאינטגרל מסוים}
% ===================================

\begin{thmbox}
\textbf{טענה 10.8 (אינטגרציה בחלקים).}
תהיינה $F, g : [a,b] \to \R$ פונקציות גזירות ב־$[a,b]$, ונגזרותיהן $F', g'$ אינטגרביליות רימן ב־$[a,b]$. אז:
\[
\boxed{\int_a^b F'(x) g(x)\dx = F(b)g(b) - F(a)g(a) - \int_a^b F(x) g'(x)\dx}
\]
\end{thmbox}

\begin{proofbox}
\textbf{הוכחה.}
נגדיר $H(x) = F(x) \cdot g(x)$. לפי כלל המכפלה:
\[
H'(x) = F'(x)g(x) + F(x)g'(x)
\]

לכן $F'(x)g(x) = H'(x) - F(x)g'(x)$.

נשלב ונשתמש בנוסחת ניוטון־לייבניץ:
\[
\int_a^b F'(x)g(x)\dx = \int_a^b H'(x)\dx - \int_a^b F(x)g'(x)\dx
\]
\[
= H(b) - H(a) - \int_a^b F(x)g'(x)\dx = F(b)g(b) - F(a)g(a) - \int_a^b F(x)g'(x)\dx
\]
\hfill $\blacksquare$
\end{proofbox}

\begin{notebox}
\textbf{סימון מקוצר.}
\[
\int_a^b u\,dv = [uv]_a^b - \int_a^b v\,du
\]
\end{notebox}

\begin{exbox}
\textbf{דוגמה 1.} חישוב $\int_0^1 x e^x\dx$.

נגדיר $F(x) = e^x$ (כך ש־$F'(x) = e^x$), $g(x) = x$ (כך ש־$g'(x) = 1$).

\textbf{פתרון:}
\[
\int_0^1 x e^x\dx = [x e^x]_0^1 - \int_0^1 e^x \cdot 1\dx = e - [e^x]_0^1 = e - (e - 1) = 1
\]
\end{exbox}

\begin{exbox}
\textbf{דוגמה 2.} חישוב $\int_0^{\pi} x \sin x\dx$.

נגדיר $F(x) = -\cos x$ (כך ש־$F'(x) = \sin x$), $g(x) = x$.

\textbf{פתרון:}
\[
\int_0^{\pi} x \sin x\dx = [-x\cos x]_0^{\pi} - \int_0^{\pi} (-\cos x)\dx = \pi - (-1) - 0 + [\sin x]_0^{\pi} = \pi + 0 = \pi
\]
\end{exbox}

\begin{exbox}
\textbf{דוגמה 3.} חישוב $\int_1^e (\ln x)^2\dx$.

נבצע אינטגרציה בחלקים עם $u = (\ln x)^2$, $dv = dx$.

אז $du = \frac{2\ln x}{x}dx$, $v = x$.

\[
\int_1^e (\ln x)^2\dx = [x(\ln x)^2]_1^e - \int_1^e 2\ln x\dx = e - 2\int_1^e \ln x\dx
\]

מדוגמה קודמת: $\int_1^e \ln x\dx = 1$. לכן:
\[
\int_1^e (\ln x)^2\dx = e - 2 \cdot 1 = e - 2
\]
\end{exbox}

% ===================================
\section{שינוי משתנה — גרסה ראשונה}
% ===================================

\begin{thmbox}
\textbf{טענה 10.9 (שינוי משתנה — גרסה 1).}
תהיינה $f : [a,b] \to \R$ ו־$g : [c,d] \to [a,b]$. נניח כי:
\begin{itemize}
    \item $f$ בעלת קדומה ואינטגרבילית ב־$[a,b]$
    \item $g$ גזירה ב־$[c,d]$
    \item $(f \circ g) \cdot g'$ אינטגרבילית על $[c,d]$
\end{itemize}
אז:
\[
\boxed{\int_c^d (f \circ g)(x) g'(x)\dx = \int_{g(c)}^{g(d)} f(t)\dt}
\]
\end{thmbox}

\begin{notebox}
\textbf{דרך לזכור.}
אם $t = g(x)$, אז $dt = g'(x)dx$. הגבולות משתנים: $x = c \Rightarrow t = g(c)$, $x = d \Rightarrow t = g(d)$.
\end{notebox}

% ===================================
\section{שינוי משתנה — גרסה שנייה}
% ===================================

\begin{thmbox}
\textbf{טענה 10.10 (שינוי משתנה — גרסה 2).}
תהיינה $f : [a,b] \to \R$ ו־$g : [c,d] \to [a,b]$ כך ש:
\begin{itemize}
    \item $f$ אינטגרבילית ב־$[a,b]$
    \item $g$ גזירה והפיכה
    \item $g'$ רציפה
\end{itemize}
אז $(f \circ g) \cdot g'$ אינטגרבילית ומתקיים:
\[
\boxed{\int_a^b f(x)\dx = \int_{g^{-1}(a)}^{g^{-1}(b)} (f \circ g)(t) g'(t)\dt}
\]
\end{thmbox}

\begin{notebox}
\textbf{הערה 10.11.}
מתקיים גם:
\[
\int_{g^{-1}(a)}^{g^{-1}(b)} f(g(t)) |g'(t)|\dt = \int_a^b f(x)\dx
\]
(תלוי בסדר הגבולות ובסימן של $g'$).
\end{notebox}

% ===================================
\section{דוגמאות לשינוי משתנה}
% ===================================

\begin{exbox}
\textbf{דוגמה 1.} חישוב $\int_0^1 \sqrt{1-x^2}\dx$ (שטח רבע מעגל היחידה).

נגדיר $x = \sin t$ על $[0, \frac{\pi}{2}]$. אז:
\begin{itemize}
    \item $dx = \cos t\,dt$
    \item $x = 0 \Rightarrow t = 0$
    \item $x = 1 \Rightarrow t = \frac{\pi}{2}$
\end{itemize}

\textbf{פתרון:}
\[
\int_0^1 \sqrt{1-x^2}\dx = \int_0^{\pi/2} \sqrt{1-\sin^2 t} \cdot \cos t\dt = \int_0^{\pi/2} \cos^2 t\dt
\]

משתמשים בזהות $\cos^2 t = \frac{1 + \cos(2t)}{2}$:
\[
= \int_0^{\pi/2} \frac{1 + \cos(2t)}{2}\dt = \frac{1}{2}\brackets{t + \frac{\sin(2t)}{2}}_0^{\pi/2} = \frac{1}{2}\parens{\frac{\pi}{2} + 0 - 0} = \frac{\pi}{4}
\]
\end{exbox}

\begin{notebox}
\textbf{טעות נפוצה בשינוי משתנה.}

בניסיון לחשב $\int_{-1}^1 \frac{1}{1+x^2}\dx$ עם $g(x) = \frac{1}{x}$:

\textbf{הבעיה:} הפונקציה $g(x) = \frac{1}{x}$ \textbf{אינה מוגדרת} ב־$0 \in [-1,1]$!

יתרה מכך:
\begin{itemize}
    \item $g([-1, 0)) = (-\infty, -1]$
    \item $g((0, 1]) = [1, +\infty)$
\end{itemize}
התחום והטווח אינם תואמים, ואין להפעיל את משפט שינוי המשתנה.

\textbf{הפתרון הנכון:} חישוב ישיר:
\[
\int_{-1}^1 \frac{1}{1+x^2}\dx = [\arctan x]_{-1}^1 = \frac{\pi}{4} - \parens{-\frac{\pi}{4}} = \frac{\pi}{2}
\]
\end{notebox}

\begin{exbox}
\textbf{דוגמה 2.} חישוב $\int_0^{\pi/4} \tan^2 x\dx$.

משתמשים בזהות $\tan^2 x = \sec^2 x - 1$:
\[
\int_0^{\pi/4} \tan^2 x\dx = \int_0^{\pi/4} (\sec^2 x - 1)\dx = [\tan x - x]_0^{\pi/4} = 1 - \frac{\pi}{4}
\]
\end{exbox}

% ===================================
\section{משפט ערך הביניים לאינטגרלים}
% ===================================

\begin{thmbox}
\textbf{טענה 10.12 (משפט ערך הביניים לאינטגרלים).}
תהי $f : [a,b] \to \R$ אינטגרבילית. אז \textbf{קיים $\mu \in [\inf f, \sup f]$} כך ש:
\[
\int_a^b f(x)\dx = \mu(b-a)
\]

\textbf{יתר על כן:} אם $f$ \textbf{רציפה} אז \textbf{קיים $c \in [a,b]$} כך ש:
\[
\boxed{\int_a^b f(x)\dx = f(c)(b-a)}
\]
\end{thmbox}

\begin{proofbox}
\textbf{הוכחה (למקרה הרציף).}
נסמן $m = \inf f$, $M = \sup f$.

מחד:
\[
m(b-a) \le \int_a^b f(x)\dx \le M(b-a)
\]

לכן:
\[
\mu = \frac{\int_a^b f(x)\dx}{b-a} \in [m, M]
\]

אם $f$ רציפה, לפי משפט ערך הביניים של ויירשטראס קיים $c \in [a,b]$ כך ש־$f(c) = \mu$. \hfill $\blacksquare$
\end{proofbox}

\begin{notebox}
\textbf{פרשנות גאומטרית.}
הערך $\mu = \frac{1}{b-a}\int_a^b f(x)\dx$ הוא \textbf{הממוצע} של $f$ על הקטע $[a,b]$.

המשפט אומר שקיימת נקודה $c$ שבה ערך הפונקציה שווה בדיוק לממוצע.
\end{notebox}

% ===================================
\section{משפט ערך הביניים הממושקל}
% ===================================

\begin{thmbox}
\textbf{טענה 10.13 (גרסה ממושקלת).}
תהיינה $f, g : [a,b] \to \R$ אינטגרביליות. נניח כי \textbf{$g \ge 0$}. אז \textbf{קיים $\mu \in [\inf f, \sup f]$} כך ש:
\[
\int_a^b f(x) g(x)\dx = \mu \int_a^b g(x)\dx
\]

\textbf{יתר על כן:} אם \textbf{$f$ רציפה} אז \textbf{קיים $c \in [a,b]$} כך ש:
\[
\boxed{\int_a^b f(x) g(x)\dx = f(c) \int_a^b g(x)\dx}
\]
\end{thmbox}

\begin{exbox}
\textbf{דוגמה.} הוכיחו כי $\frac{2}{3\pi} \le \int_{2\pi}^{3\pi} \frac{\sin x}{x}\dx \le \frac{1}{\pi}$.

\textbf{פתרון:}
נגדיר $f(x) = \frac{1}{x}$, $g(x) = \sin x$ על $[2\pi, 3\pi]$.

שימו לב ש־$\sin x \ge 0$ בקטע $[2\pi, 3\pi]$ (זה לא נכון! $\sin x \ge 0$ רק ב־$[2\pi, 3\pi]$ אם $x \in [2\pi, 3\pi]$... בעצם $\sin(2\pi) = 0$, $\sin(3\pi) = 0$, ובאמצע $\sin x$ עובר דרך ערכים חיוביים ושליליים).

\textbf{תיקון:} נשתמש בטענה 10.13 בקטע $[2\pi, 3\pi]$ כאשר $g(x) = \sin x \ge 0$ עבור $x \in [2\pi, 3\pi]$ (זה נכון כי הקטע $[2\pi, 3\pi]$ מכסה בדיוק חצי מחזור שלם של סינוס מ־0 עד 0 דרך 1).

לפי טענה 10.13 קיים $c \in [2\pi, 3\pi]$ כך ש:
\[
\int_{2\pi}^{3\pi} \frac{\sin x}{x}\dx = \frac{1}{c} \int_{2\pi}^{3\pi} \sin x\dx
\]

נחשב:
\[
\int_{2\pi}^{3\pi} \sin x\dx = [-\cos x]_{2\pi}^{3\pi} = -\cos(3\pi) + \cos(2\pi) = -(-1) + 1 = 2
\]

לכן:
\[
\int_{2\pi}^{3\pi} \frac{\sin x}{x}\dx = \frac{2}{c}
\]

כיוון ש־$c \in [2\pi, 3\pi]$, מתקיים $\frac{1}{3\pi} \le \frac{1}{c} \le \frac{1}{2\pi}$, ומכאן:
\[
\frac{2}{3\pi} \le \int_{2\pi}^{3\pi} \frac{\sin x}{x}\dx \le \frac{2}{2\pi} = \frac{1}{\pi}
\]
\end{exbox}

% ===================================
\section{נוסחת רדוקציה}
% ===================================

\begin{thmbox}
\textbf{נוסחת רדוקציה.}
לכל $n \ge 2$:
\[
\boxed{\int_0^{\pi/2} \sin^n x\dx = \frac{n-1}{n} \int_0^{\pi/2} \sin^{n-2} x\dx}
\]
\end{thmbox}

\begin{proofbox}
\textbf{הוכחה.}
נכתוב:
\[
\int_0^{\pi/2} \sin^n x\dx = \int_0^{\pi/2} \sin^{n-1} x \cdot \sin x\dx
\]

אינטגרציה בחלקים עם $u = \sin^{n-1} x$, $dv = \sin x\dx$:
\[
du = (n-1)\sin^{n-2} x \cos x\dx, \quad v = -\cos x
\]

\[
= \big[-\sin^{n-1} x \cos x\big]_0^{\pi/2} + (n-1)\int_0^{\pi/2} \sin^{n-2} x \cos^2 x\dx
\]

הגבולות מתאפסים. משתמשים ב־$\cos^2 x = 1 - \sin^2 x$:
\[
= (n-1)\int_0^{\pi/2} \sin^{n-2} x\dx - (n-1)\int_0^{\pi/2} \sin^n x\dx
\]

נסמן $I_n = \int_0^{\pi/2} \sin^n x\dx$:
\[
I_n = (n-1)I_{n-2} - (n-1)I_n \quad \Rightarrow \quad nI_n = (n-1)I_{n-2} \quad \Rightarrow \quad I_n = \frac{n-1}{n}I_{n-2}
\]
\hfill $\blacksquare$
\end{proofbox}

% ===================================
\section{תרגילים}
% ===================================

\begin{exercisebox}
\textbf{תרגיל 1.}
חשבו את האינטגרלים הבאים:
\begin{enumerate}
    \item $\int_0^1 x \arctan x\dx$
    \item $\int_0^{\pi/2} \sin^4 x\dx$ (השתמשו בנוסחת הרדוקציה)
    \item $\int_0^1 \frac{x^3}{\sqrt{1+x^2}}\dx$
\end{enumerate}
\end{exercisebox}

\begin{exercisebox}
\textbf{תרגיל 2.}
תהי $f : [a,b] \to \R$ רציפה. הוכיחו כי:
\[
\limx{0^+} \frac{1}{h} \int_a^{a+h} f(x)\dx = f(a)
\]
\end{exercisebox}

\begin{exercisebox}
\textbf{תרגיל 3.}
חשבו $\int_0^1 \frac{\ln(1+x)}{x}\dx$.

\textbf{רמז:} פתחו את $\ln(1+x)$ לטור טיילור ושלבו איבר־איבר.
\end{exercisebox}

\begin{exercisebox}
\textbf{תרגיל 4.}
הוכיחו כי לכל פונקציה רציפה $f : [-a, a] \to \R$:
\begin{enumerate}
    \item אם $f$ זוגית: $\int_{-a}^a f(x)\dx = 2\int_0^a f(x)\dx$
    \item אם $f$ אי־זוגית: $\int_{-a}^a f(x)\dx = 0$
\end{enumerate}
\end{exercisebox}
  % שיטות אינטגרציה
% יחידה 5: אינטגרלים לא אמיתיים ויישומים
\section{אינטגרלים לא אמיתיים ויישומים}

יחידה זו עוסקת באינטגרלים לא אמיתיים (אינטגרלים עם גבולות אינסופיים או עם נקודות סינגולריות), מבחני התכנסות, ויישומים גאומטריים.

% ===================================
\subsection{הגדרת אינטגרל לא אמיתי מסוג ראשון}
% ===================================

\begin{defbox}
\textbf{הגדרה 11.1 (אינטגרל לא אמיתי — גבול אינסופי).}
תהי $f : [a, +\infty) \to \R$ פונקציה \textbf{אינטגרבילית מקומית} ב־$[a, +\infty)$.

נגדיר $F : [a, +\infty) \to \R$ על ידי:
\[
F(x) = \int_a^x f(t)\dt \quad \text{לכל } x \in [a, +\infty)
\]

נאמר כי \textbf{האינטגרל הלא אמיתי} של $f$ בקטע $[a, +\infty)$ \textbf{מתכנס} כאשר קיים וסופי הגבול $\limx{+\infty} F(x)$.

במקרה זה נסמן:
\[
\boxed{\int_a^{+\infty} f(x)\dx = \limx{+\infty} \int_a^x f(t)\dt}
\]

אם הגבול אינו קיים או אינו סופי — נאמר שהאינטגרל \textbf{מתבדר} (או לא מתכנס).
\end{defbox}

\begin{notebox}
\textbf{הערה 11.2.}
באופן דומה מגדירים אינטגרל לא אמיתי על קטעים מהצורה $(-\infty, a]$:
\[
\int_{-\infty}^a f(x)\dx = \limx{-\infty} \int_x^a f(t)\dt
\]
\end{notebox}

% ===================================
\subsection{דוגמאות בסיסיות}
% ===================================

\begin{exbox}
\textbf{דוגמה 1.} $\int_0^{+\infty} \cos x\dx$ — \textbf{לא מתכנס}.

\textbf{פתרון:}
\[
\int_0^x \cos t\dt = \sin x
\]
הגבול $\limx{+\infty} \sin x$ אינו קיים (מתנדנד בין $-1$ ל־$1$).
\end{exbox}

\begin{exbox}
\textbf{דוגמה 2.} $\int_0^{+\infty} e^{\alpha x}\dx$ עבור $\alpha \in \R \setminus \{0\}$.

\textbf{פתרון:}
\[
\int_0^x e^{\alpha t}\dt = \frac{1}{\alpha}(e^{\alpha x} - 1)
\]

הגבול ב־$x \to +\infty$ קיים וסופי \textbf{אם ורק אם $\alpha < 0$}.

במקרה זה:
\[
\int_0^{+\infty} e^{\alpha x}\dx = \frac{1}{\alpha}(0 - 1) = -\frac{1}{\alpha} = \frac{1}{|\alpha|}
\]
\end{exbox}

\begin{exbox}
\textbf{דוגמה 3.} $\int_0^{+\infty} \frac{1}{1+x^2}\dx = \frac{\pi}{2}$.

\textbf{פתרון:}
\[
\int_0^x \frac{1}{1+t^2}\dt = \arctan x
\]
\[
\limx{+\infty} \arctan x = \frac{\pi}{2}
\]
\end{exbox}

\begin{thmbox}
\textbf{דוגמה חשובה — אינטגרל $\int_1^{+\infty} \frac{1}{x^\alpha}\dx$.}

\begin{itemize}
    \item עבור $\alpha = 1$: $\int_1^x \frac{1}{t}\dt = \ln x \to +\infty$ — \textbf{מתבדר}.
    \item עבור $\alpha \neq 1$: $\int_1^x \frac{1}{t^\alpha}\dt = \frac{x^{1-\alpha} - 1}{1-\alpha}$. הגבול קיים וסופי \textbf{אם ורק אם $\alpha > 1$}.
\end{itemize}

\textbf{סיכום:}
\[
\boxed{\int_1^{+\infty} \frac{1}{x^\alpha}\dx \text{ מתכנס } \Leftrightarrow \alpha > 1}
\]
\end{thmbox}

% ===================================
\subsection{אינטגרל לא אמיתי מסוג שני — סינגולריות}
% ===================================

\begin{defbox}
\textbf{הגדרה 11.3 (אינטגרל לא אמיתי — סינגולריות בגבול).}
תהי $f : [a, b) \to \R$ פונקציה אינטגרבילית מקומית ב־$[a, b)$.

נאמר כי \textbf{האינטגרל הלא אמיתי} $\int_a^b f(x)\dx$ \textbf{מתכנס} כאשר קיים וסופי הגבול:
\[
\int_a^b f(x)\dx = \limx{b^-} \int_a^x f(t)\dt
\]
\end{defbox}

\begin{exbox}
\textbf{דוגמה 1.} $\int_0^1 \frac{1}{\sqrt{x}}\dx = 2$ — \textbf{מתכנס}.

\textbf{פתרון:} יש סינגולריות ב־$x = 0$.
\[
\int_0^1 \frac{1}{\sqrt{x}}\dx = \lim_{\eps \to 0^+} \int_\eps^1 x^{-1/2}\dx = \lim_{\eps \to 0^+} [2\sqrt{x}]_\eps^1 = 2 - \lim_{\eps \to 0^+} 2\sqrt{\eps} = 2
\]
\end{exbox}

\begin{exbox}
\textbf{דוגמה 2.} $\int_0^1 \frac{1}{x}\dx$ — \textbf{מתבדר}.

\textbf{פתרון:}
\[
\int_0^1 \frac{1}{x}\dx = \lim_{\eps \to 0^+} \int_\eps^1 \frac{1}{x}\dx = \lim_{\eps \to 0^+} [-\ln\eps] = +\infty
\]
\end{exbox}

\begin{thmbox}
\textbf{דוגמה חשובה — אינטגרל $\int_0^1 \frac{1}{x^\alpha}\dx$.}

\textbf{סיכום:}
\[
\boxed{\int_0^1 \frac{1}{x^\alpha}\dx \text{ מתכנס } \Leftrightarrow \alpha < 1}
\]
\end{thmbox}

% ===================================
\subsection{קריטריון קושי להתכנסות}
% ===================================

\begin{thmbox}
\textbf{קריטריון קושי.}
האינטגרל $\int_a^{+\infty} f(x)\dx$ \textbf{מתכנס} אם ורק אם:

לכל $\eps > 0$ קיים $M \ge a$ כך שלכל $q > p > M$:
\[
\abs{\int_p^q f(x)\dx} < \eps
\]
\end{thmbox}

% ===================================
\subsection{מבחני השוואה}
% ===================================

\begin{thmbox}
\textbf{מבחן ההשוואה.}
תהיינה $f, g : [a, +\infty) \to \R$ אינטגרביליות מקומית עם $0 \le f(x) \le g(x)$ לכל $x \ge a$.
\begin{enumerate}
    \item אם $\int_a^{+\infty} g(x)\dx$ מתכנס — אז $\int_a^{+\infty} f(x)\dx$ מתכנס.
    \item אם $\int_a^{+\infty} f(x)\dx$ מתבדר — אז $\int_a^{+\infty} g(x)\dx$ מתבדר.
\end{enumerate}
\end{thmbox}

\begin{thmbox}
\textbf{מבחן ההשוואה הגבולי.}
תהיינה $f, g : [a, +\infty) \to \R$ אי־שליליות ואינטגרביליות מקומית.

אם $\limx{+\infty} \frac{f(x)}{g(x)} = L \in (0, +\infty)$ — אז:
\[
\int_a^{+\infty} f(x)\dx \text{ מתכנס } \Leftrightarrow \int_a^{+\infty} g(x)\dx \text{ מתכנס}
\]
\end{thmbox}

\begin{exbox}
\textbf{דוגמה.} קבעו האם $\int_1^{+\infty} \frac{1}{x^2 + x}\dx$ מתכנס.

\textbf{פתרון:} נשווה ל־$g(x) = \frac{1}{x^2}$:
\[
\limx{+\infty} \frac{f(x)}{g(x)} = \limx{+\infty} \frac{x^2}{x^2 + x} = \limx{+\infty} \frac{1}{1 + 1/x} = 1 \in (0, +\infty)
\]

כיוון ש־$\int_1^{+\infty} \frac{1}{x^2}\dx$ מתכנס ($\alpha = 2 > 1$), גם $\int_1^{+\infty} \frac{1}{x^2 + x}\dx$ \textbf{מתכנס}.
\end{exbox}

% ===================================
\subsection{התכנסות בהחלט והתכנסות בתנאי}
% ===================================

\begin{defbox}
\textbf{הגדרה 11.11 (התכנסות בהחלט).}
תהי $f : [a, +\infty) \to \R$ אינטגרבילית מקומית.

נאמר כי האינטגרל $\int_a^{+\infty} f(x)\dx$ \textbf{מתכנס בהחלט} כאשר האינטגרל $\int_a^{+\infty} |f(x)|\dx$ מתכנס.
\end{defbox}

\begin{thmbox}
\textbf{טענה 11.12.}
אם האינטגרל \textbf{מתכנס בהחלט} אז הוא \textbf{מתכנס}.
\end{thmbox}

\begin{proofbox}
\textbf{הוכחה.}
מקריטריון קושי: לכל $\eps > 0$ קיים $M$ כך שלכל $q > p > M$:
\[
\abs{\int_p^q f(x)\dx} \le \int_p^q |f(x)|\dx < \eps
\]
לכן $\int_a^{+\infty} f(x)\dx$ מתכנס. \hfill $\blacksquare$
\end{proofbox}

\begin{defbox}
\textbf{הגדרה (התכנסות בתנאי).}
אינטגרל שמתכנס אך \textbf{לא} מתכנס בהחלט נקרא \textbf{מתכנס בתנאי}.
\end{defbox}

\begin{exbox}
\textbf{דוגמה.}
$\int_1^{+\infty} \frac{\sin x}{x}\dx$ — \textbf{מתכנס בתנאי}.

(מוכיחים עם קריטריון דיריכלה; האינטגרל $\int_1^{+\infty} \frac{|\sin x|}{x}\dx$ מתבדר.)
\end{exbox}

% ===================================
\subsection{קריטריון אבל וקריטריון דיריכלה}
% ===================================

\begin{thmbox}
\textbf{טענה 11.20 (קריטריון אבל).}
תהיינה $f, g : [a, +\infty) \to \R$. נניח כי:
\begin{enumerate}
    \item $f$ רציפה ו־$\int_a^{+\infty} f(x)\dx$ מתכנס.
    \item $g$ מונוטונית וחסומה.
\end{enumerate}
אז $\int_a^{+\infty} f(x)g(x)\dx$ \textbf{מתכנס}.
\end{thmbox}

\begin{thmbox}
\textbf{טענה 11.21 (קריטריון דיריכלה).}
תהיינה $f, g : [a, +\infty) \to \R$. נניח כי:
\begin{enumerate}
    \item $f$ רציפה ו־$F(x) = \int_a^x f(t)\dt$ \textbf{חסומה}.
    \item $g$ מונוטונית ו־$\limx{+\infty} g(x) = 0$.
\end{enumerate}
אז $\int_a^{+\infty} f(x)g(x)\dx$ \textbf{מתכנס}.
\end{thmbox}

\begin{exbox}
\textbf{דוגמה — שימוש בקריטריון דיריכלה.}

הוכיחו כי $\int_1^{+\infty} \frac{\sin x}{x}\dx$ מתכנס.

\textbf{פתרון:}
נגדיר $f(x) = \sin x$, $g(x) = \frac{1}{x}$.

\begin{enumerate}
    \item $F(x) = \int_1^x \sin t\dt = -\cos x + \cos 1$ חסומה (ערכים בין $\cos 1 - 1$ ל־$\cos 1 + 1$).
    \item $g(x) = \frac{1}{x}$ יורדת ו־$\limx{+\infty} g(x) = 0$.
\end{enumerate}

לפי קריטריון דיריכלה, האינטגרל מתכנס.
\end{exbox}

% ===================================
\subsection{קשר בין טורים לאינטגרלים}
% ===================================

\begin{thmbox}
\textbf{טענה 11.19 (מבחן האינטגרל).}
תהי $f : [a, +\infty) \to \R$ אינטגרבילית מקומית, \textbf{אי־שלילית ויורדת}.

אז לכל $N \in \N_+$:
\[
\sum_{n=1}^{N} f(a+n) \le \int_a^{a+N} f(x)\dx \le \sum_{n=0}^{N-1} f(a+n)
\]

\textbf{יתר על כן:} האינטגרל $\int_a^{+\infty} f(x)\dx$ מתכנס \textbf{אם ורק אם} הטור $\sum_{n=0}^{\infty} f(a+n)$ מתכנס.
\end{thmbox}

\begin{exbox}
\textbf{יישום — הטור ההרמוני.}

הטור $\sum_{n=1}^{\infty} \frac{1}{n}$ והאינטגרל $\int_1^{+\infty} \frac{1}{x}\dx$ קשורים.

מהמשפט:
\[
\ln(N+1) \le \sum_{n=1}^N \frac{1}{n} \le 1 + \ln N
\]

לכן $\sum_{n=1}^N \frac{1}{n} \sim \ln N$ ובפרט הטור \textbf{מתבדר}.
\end{exbox}

% ===================================
\subsection{יישומים גאומטריים}
% ===================================

\begin{thmbox}
\textbf{שטח בין עקומות.}
עבור $f \ge g$ ב־$[a,b]$, השטח בין $y = f(x)$ ל־$y = g(x)$ הוא:
\[
\boxed{S = \int_a^b \big(f(x) - g(x)\big)\dx}
\]
\end{thmbox}

\begin{exbox}
\textbf{דוגמה.} שטח בין $y = x^2$ ו־$y = x$ ב־$[0, 1]$.

\textbf{פתרון:} בקטע זה $x \ge x^2$, לכן:
\[
S = \int_0^1 (x - x^2)\dx = \brackets{\frac{x^2}{2} - \frac{x^3}{3}}_0^1 = \frac{1}{2} - \frac{1}{3} = \frac{1}{6}
\]
\end{exbox}

\begin{thmbox}
\textbf{נפח גוף סיבוב (שיטת הדיסקים).}
סיבוב השטח מתחת ל־$y = f(x)$ סביב ציר $x$:
\[
\boxed{V = \pi \int_a^b [f(x)]^2\dx}
\]
\end{thmbox}

\begin{exbox}
\textbf{דוגמה.} נפח כדור ברדיוס $R$.

מסובבים את $y = \sqrt{R^2 - x^2}$ (חצי מעגל) סביב ציר $x$:
\[
V = \pi \int_{-R}^R (R^2 - x^2)\dx = \pi\brackets{R^2 x - \frac{x^3}{3}}_{-R}^R
\]
\[
= \pi\parens{R^3 - \frac{R^3}{3}} - \pi\parens{-R^3 + \frac{R^3}{3}} = \pi \cdot \frac{2R^3}{3} + \pi \cdot \frac{2R^3}{3} = \frac{4\pi R^3}{3}
\]
\end{exbox}

\begin{thmbox}
\textbf{אורך קשת.}
לעקומה $y = f(x)$ עם $f'$ רציפה:
\[
\boxed{L = \int_a^b \sqrt{1 + [f'(x)]^2}\dx}
\]
\end{thmbox}

\begin{exbox}
\textbf{דוגמה.} אורך הקשת $y = \frac{x^{3/2}}{3}$ ב־$[0, 4]$.

\textbf{פתרון:}
\[
f'(x) = \frac{1}{2}\sqrt{x}, \quad 1 + [f'(x)]^2 = 1 + \frac{x}{4} = \frac{4+x}{4}
\]
\[
L = \int_0^4 \sqrt{\frac{4+x}{4}}\dx = \int_0^4 \frac{\sqrt{4+x}}{2}\dx
\]

נציב $u = 4 + x$:
\[
= \frac{1}{2} \cdot \frac{2}{3}[(4+x)^{3/2}]_0^4 = \frac{1}{3}(8^{3/2} - 4^{3/2}) = \frac{1}{3}(16\sqrt{2} - 8) = \frac{8(2\sqrt{2} - 1)}{3}
\]
\end{exbox}

\begin{thmbox}
\textbf{שטח פנים של גוף סיבוב.}
סיבוב הקשת $y = f(x)$ סביב ציר $x$:
\[
\boxed{A = 2\pi \int_a^b f(x) \sqrt{1 + [f'(x)]^2}\dx}
\]
\end{thmbox}

% ===================================
\subsection{תרגילים}
% ===================================

\begin{exercisebox}
\textbf{תרגיל 1.}
קבעו התכנסות:
\begin{enumerate}
    \item $\int_1^{+\infty} \frac{\sin x}{x^2}\dx$
    \item $\int_1^{+\infty} \frac{1}{x \ln x}\dx$
    \item $\int_0^1 \frac{1}{\sqrt{x(1-x)}}\dx$
    \item $\int_1^{+\infty} \frac{x}{e^x}\dx$
\end{enumerate}
\end{exercisebox}

\begin{exercisebox}
\textbf{תרגיל 2.}
חשבו את נפח הגוף הנוצר מסיבוב $y = e^{-x}$ סביב ציר $x$ ב־$[0, +\infty)$.
\end{exercisebox}

\begin{exercisebox}
\textbf{תרגיל 3.}
תהי $f : [a, +\infty) \to \R$ אי־שלילית ואינטגרבילית מקומית. הוכיחו כי אם $\int_a^{+\infty} f(x)\dx$ מתכנס וגם קיים $\limx{+\infty} f(x)$, אז הגבול שווה $0$.
\end{exercisebox}

\begin{exercisebox}
\textbf{תרגיל 4.}
תהי $f : [1, +\infty) \to \R$ אי־שלילית, יורדת ואינטגרבילית מקומית. הגדירו:
\[
a_n = \sum_{k=1}^n f(k) - \int_1^n f(x)\dx
\]
הוכיחו כי $\{a_n\}$ מתכנסת.

\textbf{הערה:} זו הדרך להגדיר את \textbf{קבוע אוילר־מסקרוני} $\gamma = \lim_{n \to \infty}\parens{\sum_{k=1}^n \frac{1}{k} - \ln n} \approx 0.5772$.
\end{exercisebox}

\begin{exercisebox}
\textbf{תרגיל 5.}
הוכיחו כי:
\[
\int_0^{+\infty} e^{-x^2}\dx = \frac{\sqrt{\pi}}{2}
\]
\textbf{רמז:} זהו האינטגרל הגאוסיאני. ההוכחה המלאה משתמשת באינטגרל כפול.
\end{exercisebox}
  % אינטגרלים לא אמיתיים

\end{document}
